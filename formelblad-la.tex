%% CHANGES: %<<<
%% IvTj, 2017-03         * Tr2En; --lin.ber, --2:agradare; refold
%% IvTj, 2012-03-19:     * Rättelse för isometriska avb: F(u*v) = F(u)*F(v)
%%                                                   --> u*v = F(u)*F(v)
%% IvTj, 2010-10-29:
%%                       * Byte: kolumn(er) --> kolonn(er)
%%                       * Byte: |A| -->  det(A)
%%                       * Ny def på utveckling av det(A) längs en rad
%%                       * |A-\lambda I| --> det(A-\lambda I)
%%                       * .. nollställena till den karakteristiska ekv
%%                       -->  lösningarna till den karakteristiska ekv
%%
%% Ändring: Ivtj, 2009-02: lagt till direktformeln för ax^2+bx+c=0
%%                         Lagt till indelningslinjer i tabellerna
%%                         "Luftigare" tabeller
%%                         Flyttat om del av "Vektorer" --> sida 1
%%                         Lagt till en till form för kryssprodukten
%%                         Splittrat lin ber/ober.
%%                         Lagt text om A ~ I (radekvivalens med enhetsmatrisen)
%%                         Lagt till Ortogonala matriser
%%                         Ommöblering av huvudsatsen
%%                           Preciserat att A_{n x n}
%%                    Lagt till att A:s både rader och kolonner utgör bas för R^n
%%                           Bytt (in-, sur-, bijektiv) --> omvändbar
%% Ändring: Ivtj, 2008-08: Konverterart t. UTF-8, Tagit bort jth-logot
%%                         No external dependencies: alla LaTeX-def's inbäddade
%%                         Utbyta externdefinierade environments
%%                         Omdef \vec --> vektorer i bold-typpsnitt utan pil
%%                         det A --> det(A), etc
%%                         Krav att egenvektorn u != 0
%%                         Bytt ut definitionen av diagonaliserbar matris,
%%                         ej nödvändigt att kräva skilda egenvärden
%>>>

\documentclass[a4paper]{article}%<<<

\usepackage[utf8]{inputenc}  % <-- IvTj
\usepackage[swedish]{babel}
\usepackage{amsmath,fancyhdr,amssymb,graphicx}
%% \usepackage{mycommands}

\addtolength{\topmargin}{-45mm}
\addtolength{\headheight}{20ex}
\addtolength{\headsep}{-3ex}
\addtolength{\hoffset}{-18mm}
\addtolength{\textheight}{30mm}
\addtolength{\marginparwidth}{-20mm}
\addtolength{\textwidth}{55mm}
\footskip 3ex
\parindent 0 cm

\pagestyle{fancy}

\input Lang.h % En-Sv settings outsourced
%% -- Latex def's:
\let\ent\Leftrightarrow
\def\trevektor[#1,#2,#3]{\begin{pmatrix}#1\cr #2\cr #3\end{pmatrix}}
\def\abs#1{|#1|}
\def\Rone{{\mathbb R}}
\let\R\Rone
\let\iff\Leftrightarrow
\def\vec#1{\mathbf #1} %% <-- omdef \vec -- vektorer i boldface
%%  -- /Latex def's

%% -- Head && foot
\chead{\ifnum\thepage=1 {} \else \Tr{Cheatsheet Linear Algebra}{Formelblad Linjär Algebra}\fi}
\rhead{\ifnum\thepage=1 \textbf{\Tr{Cheatsheet LINEAR ALGEBRA}{Formelblad LINJÄR ALGEBRA}}
               \else sid. \thepage{} av \pageref{fin@lpage}  \fi}
\lfoot{\ifnum\thepage=1\small\today\fi}
\cfoot{}
\rfoot{\ifnum\thepage=1\small File: \texttt{\jobname.pdf}\fi}
%% -- /Head && foot

\def\norm#1{\Tr{\lVert{#1}\rVert}{|#1|}}

\def\bdb{\vec b}
\def\bde{\vec e}
\def\bdu{\vec u}
\def\bdv{\vec v}
\def\bdx{\vec x}
\def\bdzero{\vec 0}
% /preambule %>>>

\begin{document}

\subsubsection*{\Tr{Generic stuff}{Almänna formler}} %<<<

% Standarda värden för sin/cos/tan + symmetrirelationer t höger
{% Local scope def's
\let\F\frac
\newcommand\CC[1]{#1^\circ}
\def\vPad{&&&&&\\*[-10pt]}
\def\myEnd{\\\hline\vPad}
\newcommand\tW{{\sqrt2}}
\newcommand\tH{{\sqrt3}}
$
\begin{array}[m]{|l|l|}
\hline
\multicolumn{2}{|l|}{
  \mbox{ \Tr{The conjugate and the square rules}{Konjugat- och kvadreringsreglerna}: }
} \\
\hline

\rule[-4pt]{0pt}{15pt}
   (a-b)(a+b) = a^2-b^2 & (a\pm b)^2 = a^2\pm2ab+b^2 \\
\hline
\hline
\multicolumn{2}{|l|}{
  \mbox{ \Tr{Roots of a second-degree polynomial}{Rötter till andragradspolynom}: }
} \\
\hline
\rule[-2pt]{0pt}{15pt}
x^2+px+q = 0 & ax^2+bx+c = 0 \\
  \displaystyle
\rule[-10pt]{0pt}{32pt}
  x_{1,2} = -\frac p2\pm\sqrt{\Bigl(\frac p2\Bigr)^2-q}
&
  \displaystyle
   x_{1,2} = -\frac{-b\pm\sqrt{b^2-4ac}}{2a}
\\
\hline
\end{array}
% --
\quad
% --
\begin{array}{|c|c|c|c|c|c|}
\hline
\multicolumn{6}{|l|}{
  \mbox{\Tr{Trigonometric functions of standard angles}{Trigonometriska funktioner av standardvinklar}}
} \\
\hline
\hline\vPad
\mbox{(Std.~\Tr{angles}{vinklar})}^\circ
  & \CC{0} &\CC{30} &\CC{45} &\CC{60} &\CC{90}   % &\CC{120} &\CC{135} &\CC{150} &\CC{180}
 \myEnd
 x \mbox{ (rad)}
        & 0     & \pi/6 & \pi/4 &  \pi/3 & \pi/2 % & 2\pi/3 & 3\pi/4 & 5\pi/6 &\pi
 \\\hline\hline\vPad
 \cos(x)&   1   & \tH/2 & 1/\tW &  1/2  &   0    % & -1/2   & -1/\tW & -\tH/2 & -1
 \myEnd
 \sin(x)&   0   &   1/2 & 1/\tW & \tH/2 &   1    % & \tH/2  &  1/\tW &   1/2  &  0
 \myEnd
 \tan(x)&   0   & 1/\tH &   1   & \tH   &
                                    \dag % \mbox{ej def}
                                                 % &-\tH  &    -1  & -1/\tH &  0
 \\\hline
\end{array}
%%% mer trigonometri; commented out: 2018-08-15:
%%% \begin{array}[m]{|l|l|}
%%% \hline
%%%   \cos(-x) = \cos(x)    & \cos(x\pm\pi) = -\cos(x) \\
%%%   \sin(-x) = -\sin(x)   & \sin(x\pm\pi) = -\sin(x) \\
%%% \hline
%%% \hline
%%%   \sin(\pi-x) = \sin(x) &  \cos(x+2\pi) = \cos(x) \\
%%%   \cos(\pi-x) = -\cos(x)&  \sin(x+2\pi) = \sin(x) \\
%%% \hline
%%% \hline
%%%  \tan(-x) = \tan(x)     & \tan(x\pm\pi) = \tan(x) \\
%%% \hline
%%% \end{array}
$
}%>>>

\subsection*{\Tr{Vectors}{Vektorer}}

% Allmännt %<<<
\Tr{The formulas below are spelled in $\mathbb{R}^3$ for the sake of
  concretion.
With the exception for the cross product, these are readily generalized
for vectors both in the plane as well as in higher dimensions.%
}{%
Formlerna i detta avsnitt utgår från 3-dimensionella vektorer. Men
undantag för formlerna för vektorprodukt samt de formler som rör
planet, är de med självklara modifieringar giltiga även för andra
dimensionstal.}

\bigskip
\Tr{The vectors}{Vektorerna}
$\bdu=\trevektor[u_1,u_2,u_3]$
\Tr{and}{och}
$\bdv=\trevektor[v_1,v_2,v_3]$
\Tr{are given in an ON-basis}{är givna i en ON-bas},
$\theta$
\Tr{is the angle between}{är vinkeln mellan}
$\bdu$
\Tr{and}{och}
$\bdv$
\Tr{and}{och}
$\lambda\in\mathbb R$.
\Tr{Then}{Då är}:
%>>>

% Tabell sum, sträckning, längd %<<<
\medskip
\begin{tabular}{|l|l|}
  \hline  &  \\[-8pt]
  \textbf{Addition: }
  $\bdu+\bdv=\trevektor[u_1+v_1,u_2+v_2,u_3+v_3]=\bdv + \bdu$
  &
  \textbf{Mult.~\Tr{with scalar}{med tal}: }
  $\lambda\bdu=\trevektor[\lambda u_1,\lambda u_2,\lambda u_3] = \bdu\lambda$;
  \; $0\bdu = \bdzero$
  \\[15pt] \hline \multicolumn{2}{|c|}{} \\[-9pt]
  \multicolumn{2}{|c|}{
  \Tr{Linear Combination (LC)}{Linjärkombination (LK)}:
  $
  \displaystyle \sum_{k=1}^n \lambda_k\bdu_k
  =\lambda_1\bdu_1 + \cdots +\lambda_n\bdu_n
  $;
  \hspace{.2em}
  Trivial \Tr{LC}{LK}:
  $\lambda_1 = \cdots = \lambda_n = 0
  \displaystyle
  \;\rightsquigarrow\; \sum_{k=0}^{n} 0\mkern2mu\bdu_k = \mathbf{0}
  $
  }
  \\[15pt] \hline \multicolumn{2}{|c|}{} \\[-9pt]
  \multicolumn{2}{|l|}{\textbf{\Tr{Length}{Längd}/norm}: \;
  $\norm{\bdu}
    =\sqrt{u_1^2+u_2^2+u_3^2}
  = \sqrt{\bdu\circ\bdu}$\,;
    \; $\norm{\lambda\bdu}=\abs \lambda\norm{\bdu}$
  } %multicolumn
  \\[8pt] \hline \multicolumn{2}{|c|}{} \\[-9pt]
  \multicolumn{2}{|l|}{\Tr{Unit vector}{Enhetsvektorn} \/$\bde$\/
  \Tr{pointing in direction}{utmed vektorn} $\bdu$:
  $
  \; \displaystyle\bde = \frac{\bdu}{\norm{\bdu}}
  $
  } %multicolumn
  \\[8pt] \hline
\end{tabular}%>>>

\medskip % Dot product %<<<
\begin{tabular}{|p{0.2\linewidth}|p{0.75\linewidth}|}
  \hline &\\*[-8pt]
  \Tr{Dot product}{Skalärprodukt}:
  &$\bdu\circ\bdv=\norm{\bdu}\norm{\bdv}\cos\theta
  =u_1v_1+u_2v_2+u_3v_3$\\*[2pt]
  &$\bdu\perp\bdv\ent\bdu\circ\bdv=0$ \\[2pt]
  \Tr{Properties}{Räkneregler}:
  &$\bdu\circ\bdv=\bdv\circ \bdu$,
  $(\lambda\bdu)\circ\bdv=\bdu\circ(\lambda\bdv)=\lambda(\bdu\circ\bdv)$\\*[2pt]
  &$\bdu\circ(\bdv+\vec w)=\bdu\circ\bdv+\bdu\circ\vec w$\\*[2pt]
  &$\bdu\circ\bdu=\norm{\bdu}^2$\\[2pt]
  & $\norm{\bdu}=\sqrt{\bdu\circ\bdu}=\sqrt{u_1^2+u_2^2+u_3^2}\ge0$.
  Obs! $\norm{\bdu}=0\iff\bdu=\mathbf 0$
\\*[2pt]\hline&\\*[-8pt]
\Tr{Ortogonal projection}{Projektionsformeln}:
  &
  \Tr{The orthogonal projection}{Den vinkelräta projektionen}
  % $\bdu'$
  \Tr{of}{av}
  $\bdu$
  \Tr{upon}{längs}
  $\bdv$:\,\;
  $\text{proj}_{\bdv}\bdu
  = \bdu'
  = \dfrac{\bdu\circ\bdv}{\norm{\bdv}^2}\bdv$
  \\*[8pt] \hline
\end{tabular}%%>>>

\medskip % Cross product & triple product %<<<
\begin{tabular}{|p{0.2\linewidth}|p{0.75\linewidth}|}
  \hline &\\*[-8pt]
  \Tr{Cross product}{Kryssprodukt}:
  &  %% Tilläg IvTj, 2009-02-13
  $\bdu\times\bdv=\trevektor[u_1,u_2,u_3]\times\trevektor[v_1,v_2,v_3]
  =
  \left(\mkern-12mu
  \begin{array}{r}
      \begin{vmatrix}
        u_2 & v_2\\
        u_3 & v_3\\
      \end{vmatrix}
   \\*[8pt]
  -   \begin{vmatrix}
        u_1 & v_1\\
        u_3 & v_3\\
      \end{vmatrix}
    \\*[8pt]
      \begin{vmatrix}
        u_1 & v_1\\
        u_2 & v_2\\
      \end{vmatrix}
    \end{array}\mkern-3mu\right)
  =\trevektor[u_2v_3-u_3v_2,u_3v_1-u_1v_3,u_1v_2-u_2v_1]$\\*[2pt]
  \Tr{Properties}{Räkneregler}:
  &$\bdu\times\bdv$
  \Tr{is orthogonal to both}{är ortogonal mot både}
  $\bdu$
  \Tr{and}{och}
  $\bdv$\\*[2pt]
  &$\norm{\bdu\times\bdv}=\norm{\bdu}\norm{\bdv}\sin\theta=$
  (\Tr{the area of the parallelogram spanned by}{arean av den parallellogram som spänns upp av}
  $\bdu$ \& $\bdv$)\\*[2pt]
  &$\bdu\times\bdv=-\bdv\times \bdu$\\[2pt]
  & $(\lambda \bdu)\times\bdv=\bdu\times(\lambda \bdv)=\lambda (\bdu\times\bdv)$\\*[2pt]
  &$\bdu\times(\bdv+\vec w)=\bdu\times\bdv+\bdu\times\vec w$
  \\*[2pt] \hline &\\*[-8pt]
  \Tr{Triple product}{Skalär trippelprodukt}:
  &
  $
  \bdu\circ(\bdv\times\vec w)
  = \vec w\circ(\bdu\times\bdv)
  = \bdv\circ(\vec w\times\bdu)
  =\begin{vmatrix}
    u_1&v_1&w_1\\
    u_2&v_2&w_2\\
    u_3&v_3&w_3
  \end{vmatrix}
  $ \\[2pt]
  &
  $
  \hspace{54pt}=\pm$% <-- needed for spacing
  (\Tr{the volume of the parallelepiped spanned by}{volymen av den parallellepiped som spänns upp av}
  $\bdu$, $\bdv$ och $\vec w$)\\
  \hline
  \end{tabular}%>>>

\subsection*{\Tr{Lines and planes}{Linjer och plan}}%<<<

  \begin{tabular}{|p{0.2\linewidth}|p{0.75\linewidth}|}
  \hline
  &\\*[-8pt]
  \Tr{Straight line}{Rät linje}:
  &
  \Tr{A straight line $L$ through the point}{En rät $L$ linje genom punkten}
  $P_0=(x_0,y_0,z_0)$
  \Tr{along the direction vector}{med riktningsvektorn}
  \\[2pt]
  &$\bdv=\trevektor[v_x,v_y,v_z]$
  \Tr{has the parametric representation}{har på parameterform ekvationen}:
  $L\!:\trevektor[x,y,z]=\trevektor[x_0,y_0,z_0]+t\trevektor[v_x,v_y,v_z],\,
  t\in\mathbb R$.
  \\*[14pt] \hline
  \Tr{Equation of a plane}{Planets ekvation}:
  &
  \Tr{The plane through the point}{Ett plan genom punkten}
  $P=(x_0,y_0,z_0)$
  \Tr{orthogonal to}{med normalrikning}:
  $\mathbf n=(a,b,c)^t$:
  \[
  (\mathbf{r} - \mathbf{r}_P)\circ\mathbf{n} = 0
 %%% \iff
 %%% \left[
 %%%    \trevektor[x,y,z]
 %%%   -\trevektor[x_0,y_0,z_0]
 %%% \right]
 %%% \circ
 %%% \trevektor[a,b,c]=0
 \quad \iff \quad
  ax+by+cz=d
  \]
  \Tr{Here}{Här}
  $\mathbf{r}$
  \Tr{is the pos.~vector of an arbitrary pt in the plane}{är ortsvektorn av godt. punkt i planet},
  $\mathbf{r}_P$
  \Tr{-- the one of}{är ortsvektorn av}
  $P$
  \Tr{and}{och}
  $d=\mathbf{r}_p\circ\mathbf{n}$.
  \\*[7pt] % &\\*[-8pt]
  \Tr{In parameter form}{På parameterform}:
  &
  \Tr{The plane through the pt}{Ett plan genom punkten}
  $P=(x_0,y_0,z_0)$
  \Tr{parallel to the linearly independent vectors}{parallellt med de linjärt oberoende vektorerna}
  $\vec v$
  \Tr{and}{och}
  $\vec w$
  \Tr{has a parametric reresentation}{har parametriska framställningen}
  \\
  &
  \vspace{-1em}
  $$
  \vec r = \vec r_P + s\bdv + t\vec w,
 %% \iff
 %% \trevektor[x,y,z]
 %%   = \trevektor[x_0,y_0,z_0]
 %%   +s\trevektor[v_x,v_y,v_z]
 %%   +t\trevektor[w_x,w_y,w_z],
     \; s,t\in\Rone
  $$
  \vspace{-2.2em}
  %%&På parameterfri form är planets ekvation $Ax+By+Cz+D=0$ där $\vec
  %% n=\trevektor[A,B,C]$ är en normalvektor till planet.
  \\*[8pt]
  %En avståndsformel&Avståndet mellan punkten $(x_1,y_1,z_1)$ och planet
  %$Ax+By+Cz+D=0$ är
  %$\dfrac{\abs{Ax_1+By_1+Cz_1+D}}{\sqrt{A^2+B^2+C^2}}$.\\*[8pt]
  \hline
\end{tabular}%>>>

\subsection*{\Tr{Linear independence. Linear equation systems}%<<<
            {Linjärt oberoende. Linjära ekvationssystem}}

\begin{tabular}{|p{0.2\linewidth}|p{0.75\linewidth}|}
  \hline
  &\\*[-8pt]
  \Tr{Linear independence}{Linjärt oberoende}:
  &
  \Tr{The vectors}{Vektorerna}
  $\{\bdu_1,\dots\bdu_n\}$
  \Tr{are linear independent if an only if the homogeneous system}
     {är linjärt oberonde om ekvationssysemet}
  $\lambda_1\bdu_1+\dots\lambda_n\bdu_n=\bdzero$
  \Tr{has only the trivial solution}{har endast den triviala lösningen}
  $\lambda_1=\lambda_2=\cdots\lambda_n=0$
  \\*[2pt]
  \hline
%%%  &\\*[-8pt]
%%%  \Tr{Linear dependence}{Linjärt beroende}:
%%%  &
%%%  %\Tr{The negation of lin.~independence}{Motsatsen till lin.~oberoende}:
%%%  \Tr{The vectors}{Vektorerna}
%%%  $\bdu_1,\dots\bdu_n$
%%%  \Tr{are linear dependent if an only if there exists a non-trivial LC}
%%%     {är linjärt beronde då och endast då det existerar en icke-trivial LK}
%%%  $\lambda_1\bdu_1+\dots\lambda_n\bdu_n=\bdzero$
%%%  (\Tr{i.e. at least one}{dvs minst en} $\lambda_k\neq0$, $1\le k\le n$)
%%%  \\*[2pt] \hline
  &\\*[-8pt]
    \Tr{Elementary Row Operations}{Elementära radoperationer}
    (ERO):
 &
  1) \Tr{Swap two equations}{Byta plats på två ekvationer}; \;
  2) \Tr{Multiply of both sides with a number}{Multiplikation av en ekvation med ett tal} $\lambda\neq0$; \;
  3) \Tr{Add a multiple of an equation to another one}{Addera multipel av en ekvation till en annan}
\\ \hline
  &\\*[-8pt]
  \Tr{Gauss elimination}{Gausseliminering}:
  &
  \Tr{Reduce a linear system through ERO's to echelon form}
  {Att  med hjälp av ERO reducera ett linjärt ekvationsssyem till echelonform}
  \\[2pt] \hline
\end{tabular}%>>>

\subsection*{\Tr{Matrices}{Matriser}} %<<<

\begin{tabular}{|p{0.2\linewidth}|p{0.75\linewidth}|}
  \hline & \\*[-9pt]
  Addition: & $(A+B)+C=A+(B+C)$;\; $A+B=B+A$
  \\*[2pt]\hline & \\*[-9pt]
  Mult.~\Tr{with scalar}{med tal}: & $\lambda(A+B)=\lambda A+\lambda B$
  \\*[2pt]\hline & \\*[-9pt]
  \Tr{Multiplication}{Multiplikation}:
  %% &$AB\ne BA$ (\Tr{in general}{i allmänhet}) \\*[2pt]
  &$(AB)C=A(BC)$; \ $A(B+C)=AB+AC$, \  $(A+B)C=AC+BC$
  \\*[2pt]\hline & \\*[-11pt]
  \Tr{Transposition}{Transponering}:
  &
  \Tr{Rows and columns swap places, e.g.}{Rader och kolonner byter plats, t\,ex},\;
  $ \begin{pmatrix}a_{11}&a_{12}&a_{13}\\a_{21}&a_{22}&a_{23}\end{pmatrix}^T=
    \begin{pmatrix}a_{11}&a_{21}\\a_{12}&a_{22}\\a_{13}&a_{23}\\\end{pmatrix}$\\*[3pt]
  &$(A+B)^T=A^T+B^T$;\; \
  $(AB)^T=B^TA^T$;\; \
  $(A^{-1})^T=(A^T)^{-1}$\\*[2pt]
  & \\*[-9pt]
  \hline
  \Tr{Row equivalence}{Radekvivalens}:
  &
  $A\sim A'$\;
  \Tr{if $A$ can be driven to $A'$ with a finite number of ERO:s}
      {om $A$ kan omformas till $A'$ med ändligt antal ERO:s}
  \\*[2pt] \hline
  \Tr{Rank}{Rang}: &
  $\mbox{rank}(A) = \max$
  \Tr{number of linearly independent column vectors of}{antalet av linjärt oberoende kolonnvektorer av} $A$
  \\ \hline
\end{tabular}%>>>

\subsection*{\Tr{Square matrices}{Kvadratiska matriser}}%<<<
\begin{tabular}{|p{0.2\linewidth}|p{0.75\linewidth}|}
  \hline
  \Tr{The unit matrix}{Identitetsmatrisen}:
  &
  $I= \mbox{diag}(1,1,\dots,1)
  %% \begin{pmatrix}
  %%   1&0&\cdots&0\\
  %%   0&1&\cdots&0\\
  %%   \vdots&\vdots&\ddots&\vdots\\
  %%   0&0&\cdots&1
  %% \end{pmatrix}
  $, \;
  $AI=IA=A$
  \\*[2pt]\hline
  \rule{0pt}{10pt}
  $$
  A_{n\times n}\sim I
  $$
  $$\mbox{rank}(A_{n\times n})=n$$
  &
  \Tr{Often shown via}{Visas ofta genom}
  Gauss-Jordan
  \Tr{elimination for full rank square matrices, forward and backward, when solving a linear system}
  {elimination, ned- och uppåt, i kvadratiska matriser av full rang, när man löser ett linjär system},
  \[
      AX=B \;\rightsquigarrow\; (A|B)\sim\cdots\sim (I|X),
  \]
  \Tr{or finding the inverse via}{eller för att bestämma inversen}:
  \/ $
      (A|I)\sim \cdots\sim (I|A^{-1})
  $
  \\ \hline
  \rule{0pt}{12pt}%
  \Tr{Matrix inverse}{Matrisinvers}:
  &
  $AA^{-1}=A^{-1}A=I$; \quad
  \\[12pt]
  \Tr{Properties}{Räkneregler}:
  &
  $(AB)^{-1}=B^{-1}A^{-1}$; \quad
  $(A^T)^{-1}=(A^{-1})^T$; \quad
  $A^{-n} = (A^{-1})^n = (A^n)^{-1}$
  \\*[7pt]
  \Tr{Orthogonal matrices}{Ortogonala matriser}: & $A^{-1}=A^T$
  \\[3pt]
  $2\times 2$--matri\Tr{x }{s}inverse\Tr{s}{r}:&
  $A_{2\times2}^{-1}
  =\begin{pmatrix}a&b\\c&d\end{pmatrix}^{-1}
  =\dfrac1{\det(A)}\begin{pmatrix}d&-b\\-c&a\end{pmatrix}
  =\dfrac1{ad-bc}\begin{pmatrix}d&-b\\-c&a\end{pmatrix}$
  \\[10pt]
  \hline
\end{tabular}%>>>

\subsection*{Determinant\Tr{s}{er}}%<<<
\begin{tabular}{|p{0.18\linewidth}|p{0.77\linewidth}|}
  \hline
   & \\*[-8pt]
   $2\times2$--determinant\Tr{s}{er}:
  &
  $\det\begin{pmatrix}a&b\\c&d\end{pmatrix}
  =\begin{vmatrix}a&b\\c&d\end{vmatrix}=ad-bc$.
  \\*[9pt] \hline & \\*[-15pt]
  \vfill
  \Tr{Develop after row}{Utveckla efter rad} $i$
  &
  $$
  \det(A_{n\times n})
  = a_{i1}(-1)^{i+1}M_{i1}
  + \cdots + a_{ij}(-1)^{i+j}M_{ij}
  + \cdots + a_{in}(-1)^{i+n}M_{in},
 $$
 $M_{ij}$
 \Tr{is the determinent obtained by excluding row}{är determinanten som fås genom att stryka rad}
 $i$
 \Tr{and column}{och kolonn}
 $j$
 \Tr{in}{i}
 $\det(A)$.
  \\*[5pt] \hline & \\*[-8pt]
  \Tr{Properties}{Räkneregler}:
  &
  $\det(A^T)=\det(A)$
  (%
  \Tr{i.e., everything allowed row-wise is also allowed column-wise}
  {dvs, allt som är tillåtet att göra radvis fungerar även kolonnvis});
  \\*[2pt]
  &
  $\det(AB)=\det(A)\det(B)$,
  $\det(A^{-1})=1/\det(A)$
  \\*[2pt]
  &
  \parbox{10em}{%
    \Tr{Factor out a multiple of a row or column}
    {Multipel av en rad eller kolonn kan brytas ut}:
}
  \ $
  \begin{vmatrix}
    a&\lambda b&c\\d&\lambda e&f\\g&\lambda h&i
  \end{vmatrix}
  =\lambda
  \begin{vmatrix}
    a&b&c\\d&e&f\\g&h&i
  \end{vmatrix},\quad
  \begin{vmatrix}
    a&b&c\\d&e&f\\\lambda g&\lambda h&\lambda i
  \end{vmatrix}
  =\lambda
  \begin{vmatrix}
    a&b&c\\d&e&f\\g&h&i
  \end{vmatrix}.
  $\\*[15pt]
  &
  \Tr{Swapping of  two rows or columns reverses the sign of the determinant}
  {Om två rader eller två kolonner byter plats skiftar determinanten tecken.}
  \\*[2pt]
  &
  \Tr{Adding a multiple of a row to another row does not change the determinant}
  {En multipel av en rad kan läggas till en annan rad utan att determinanten ändras}
  \\*[2pt]
  &
  \Tr{Adding a multiple of a column to another column does not change the determinant}
  {En multipel av en kolonn kan läggas till en annan utan att determinanten ändras}
\\
  \hline
\end{tabular}%>>>

\subsection*{\Tr{Bases and changes of bases}{Baser och basbyten}}%<<<
\begin{tabular}{|p{0.2\linewidth}|p{0.75\linewidth}|}
  \hline & \\*[-9pt]
  \Tr{Bases}{Baser}:
  &
  \Tr{Any set of $n$ linearly independent vectors}{En uppsättning av $n$ linjärt oberoende vektorer}
  $\underline{\vec e}=\{\vec e_1,\dots,\vec e_n\}$
  \Tr{is a basis for}{bildar en bas för}
  $\R^n$.
  \Tr{Any vector}{Varje vektor}
  $\bdb\in\R^n$
  \Tr{can be written as a unique LC in this basis}{kan skrivas entydigt som LK i denna bas}:
  $\bdb
  = \underline{\vec e}X
  =x_1\vec e_1\dots x_n\vec e_n$.
 %% \Tr{The coefficients}{Koefficienterna}
 %% $x_1,\dots,x_n$
 %% \Tr{are}{är}
 %% $\bdb$%
 %% \Tr{'s coordinates in base}{:s koordinater i basen}
 %%  $\underline{\vec e}$.
  \\*[2pt] \hline\\*[-9pt]
  \Tr{Change of basis}{Basbyten}:
  &
  \Tr{Let}{Låt}
  $\bdb
  = \underline{\vec e}X
  = \underline{\vec f}Y
  $
  \Tr{has coordinates}{har koordinaterna}
  $x_1,\dots,x_n$
  \Tr{in base}{i basen}
  $\underline{\vec e}=\{\vec e_1,\dots,\vec e_n\}$
  \Tr{and}{och}
  \Tr{coordinates}{koordinaterna}
  $y_1,\dots,y_n$
  \Tr{in base}{i basen}
  $\underline{\vec f}=\{\vec f_1,\dots,\vec f_n\}$,
  \Tr{where}{där}
  $
  \underline{\vec f}
  = \underline{\vec e}P
  $.
  \Tr{Then}{Då är}
  $$
  \trevektor[x_1,\vdots,x_n]
  = \sum_{k=1}^{n} y_k\vec f_k
  = P\trevektor[y_1,\vdots,y_n]
  \;\ent\;
  X = PY
  ,\quad
  P=
  \begin{pmatrix}
    |&&|\\
    \vec f_1&\dots&\vec f_n\\
    |&&|
  \end{pmatrix},\!
  $$
  \Tr{where the columns of}{där kolonnerna i}
  $P$
  \Tr{are populated by the components of}{består av komponenterna till}
  $\vec f_1,\dots,\vec f_n$
  \Tr{in base}{i bas}
  $
  \underline{\vec e}.
  $
  \\*[2pt]  \hline
\end{tabular}%>>>

\subsection*{\Tr{Linear maps}{Linjära avbildningar}}%<<<

\begin{tabular}{|p{0.17\linewidth}|p{0.78\linewidth}|}
  \hline & \\*[-9pt]
  Definition:
  &
  \begin{tabular}[t]{ll}
  $F\!:\mathbb{R}^n\to\mathbb{R}^n$--lin\Tr{ear}{jär}
  &
  $\ent F(\bdu+\bdv)=F(\bdu)+F(\bdv)$s
  \Tr{and}{och}
  $F(\lambda\bdu)=\lambda F(\bdu)$.
  \\
 &
 $\ent F(\bdu)=A\bdu$,  $A_{n\times n} = (F(\vec e_1),\ldots, F(\vec e_n))$
  \end{tabular}
  \\*[2pt] \hline & \\*[-9pt]
  \Tr{Map composition}{Sammansättning}:
  &
  \Tr{If}{Om}
  $F(\bdu)=A\bdu$
  \Tr{and}{och}
  $G(\bdu)=B\bdu$
  $\;\rightsquigarrow\;$
  $
  (F\circ G)(\bdu)
  =
  F\bigl(G(\bdu)\bigr)
  =F\bigl(B\bdu\bigr)
  =AB\bdu
  $.
  \\*[2pt] \hline & \\*[-9pt]
  \Tr{Isometric maps}{Isometriska}:
  &
  $F:\R^n\to\R^n,F(\bdu)=A\bdu$
  \Tr{is isometric}{är isometrisk}
  $\ent \norm{F(\bdu)}=\norm{\bdu}$
  \Tr{for all}{för alla}
  $\bdu\in\R^n\ent$\\
  \Tr{}{avbildningar}
  &
  $\bdu\circ\bdv=F(\bdu)\circ
  F(\bdv)$
  \Tr{for all}{för alla}
  $\bdu$
  \Tr{and}{och}
  $\bdv\in\R^n\ent A$
  \Tr{is an ON-matrix}{är en ON-matris}
  $\ent A^{-1}=A^T$ \\
  \hline\\*[-9pt]
  \Tr{Change of basis}{Basbyte}:
  &
  %% Om $F$ har
  %% avbildningsmatris $A$ i basen $\vec e_1,\dots,\vec e_n$ och
  %% avbildningsmatris $A'$ i basen $\vec f_1,\dots,\vec f_n$, gäller
  %% $A'=P^{-1}AP$, där $P$ är basbytesmatrisen definierad ovan.\\
  $
    F(\bdx) = \underline{\vec e}AX = \underline{\vec f}(PAP^{-1})Y
  $,
  \Tr{with}{med}
  $
  \underline{\vec e}=\{\vec e_1,\ldots,\vec e_n\},\;
  \underline{\vec f}=\{\vec f_1,\ldots,\vec f_n\},\;
  P=(\vec f_1,\ldots,\vec f_n)$ % \Tr{as above}{som ovan}
  \\
 \hline
\end{tabular}%>>>

\subsection*{\Tr{Egenvalues och egenvectors}{Egenvärden och egenvektorer}}%<<<
\label{subsection*}
\begin{tabular}{|p{0.2\linewidth}|p{0.75\linewidth}|}
  \hline\\*[-9pt]
  Definition:
  &
  \Tr{The matrix}{Matrisen}
  $A$
  \Tr{has eigenvector}{har egenvektorn}
  $\bdu\neq\vec0$
  \Tr{to the eigenvalue}{med egenvärdet}
  $\lambda$
  \Tr{if}{om}
  $A\bdu=\lambda\bdu$
  \\*[2pt] \hline\\*[-9pt]
  \Tr{The eigenvalues}{Egenvärdena}
  $\lambda_k$
   &
   \Tr{are roots to the chacteristic equation}
   {bestäms som lösningarna till den karakteristiska ekvationen}
   $\det(A-\lambda I)=0$\\
   \Tr{The eigenvectors}{Egenvektorerna}
  $\bdu_k$
  &
  \Tr{are solutions to the homogeneous system}{bestäms ur det homogena systemet}
  $(A-\lambda I)\bdu=\mathbf{0}$
  \\*[2pt] \hline\\*[-9pt]
%%   Diagonalisering&Om $n\times n$-matrisen $A$ har $n$ olika egenvärden
%%   $\lambda_1,\dots,\lambda_n$ med motsvarande egenvektorer
%%   $\bdu_1,\dots,\bdu_n$, är $A=PDP^{-1}$ där
  \Tr{Diagonalization}{Diagonalisering}
&
\Tr{If the}{Om}
$n\times n$-\Tr{matrix}{matrisen}
$A$
\Tr{has}{har} $n$
\Tr{linearly independent eigenvectors}{st linjärt oberoende egenvektorer}
$\bdu_1,\dots,\bdu_n$
\Tr{to the eigenvalues}{med motsvarande egenvärden}
$\lambda_1,\dots,\lambda_n$,
\Tr{then}{då är}:
\vspace{-10pt}
  $$
A=PDP^{-1}\;
\mbox{ \Tr{with}{där} }\;
   P=\begin{pmatrix}
    |&&|\\
    \bdu_1&\dots&\bdu_n\\
    |&&|
  \end{pmatrix}
  \mbox{ \ \Tr{and}{och} \ \ }
  D = \begin{pmatrix}
    \lambda_1&\dots&0\\
    \vdots&\ddots&\vdots\\
    0&\dots&\lambda_n
  \end{pmatrix}
  $$
  \vspace{-12pt}
  \\
  \hline
\end{tabular}%>>>

\subsection*{\Tr{The Great Unifying Theorem for square matrices} {Huvudsatsen för kvadratiska matriser}:}%<<<

\Tr{Let $A$ be a $n\times n$ matrix}{Låt $A$ vara en $n\times n$ matris}.
\medskip

%%%% \Tr{The GUT below rules them all, in the sense that}{Huvudsatsen är vattendelare i följande bemärkelse}:
%%%% \begin{itemize}
%%%%   \item
%%%%    \Tr{Any square matrix}
%%%%       {Varje kvadratisk matris}
%%%%    $A_{n\times n}$
%%%%    \Tr{is either GUT-compliant or not}{antingen uppfyller Huvudsatsen eller gör inte det}.
%%%%   \item
%%%%    \Tr{If}{Ifall} $A$
%%%%    \Tr{satisfies any of the conditions of the GUT, it compulsory satisfies them all}
%%%%       {uppfyller någon av Huvudsatsens krav, den upfyller obligatoriskt alla de övriga}.
%%%%   \item
%%%%       \Tr{Equivalently, if $A$ breaks any of the conditions of the GUT, it breaks them all}
%%%%       {Ekvivalent, skulle $A$ bryta mot en av Huvudsatsens krav, den bryter även mot alla de övriga}.
%%%% \end{itemize}
%%%% \medskip

%% Låt $A$ vara en $n\times n$--matris.\\*[10pt]
\begin{tabular}{|p{0.475\linewidth}|p{0.475\linewidth}|}
  \hline
  &\\*[-10pt]
%  Låt $A$ vara en $n\times n$--matris.\hfill\break
  \textbf{\Tr{The GUT equivalent statements}{Huvudsatsens ekvivalenta påståenden}}: &
  \textbf{\Tr{Its converse}{Dess motsats}}:
  \\
  \hline
  \\*[-24pt]
  \begin{itemize}\addtolength{\itemsep}{-5pt}
    \item $A\sim I$ ($A$ \Tr{is row-equivalent to}{är radekvivalent med} $I$)
  \item $\mbox{rank}(A)=n$
  \item $\det(A)\ne0$
  \item $A$ \Tr{has a unique inverse}{har entydig invers} \/ $A^{-1}$
  \item \Tr{The homogeneous system}{Det homogena systemet}
    $A\vec x=\vec 0$
    \Tr{has only the trivial solution}{har endast den triviala lösningen}
    $\vec x=\vec 0$
  \item \Tr{The inhomogeneous system}{Det inhomogena systemet}
    $A\vec x=\vec b$
    \Tr{is uniquely solvable for any right-hand side}
       {har en entydig lösning för varje val av högerledet} $\vec b$
  \item \Tr{The rows of $A$ are linearly independent}{$A$:s rader är linjärt oberoende}
  \item \Tr{The rows of $A$ form a basis for}        {$A$:s rader utgör en bas för} $\mathbb{R}^n$
  \item \Tr{The columns of $A$ are linearly independent}{$A$:s kolonner är linjärt oberoende}
  \item \Tr{The columns of $A$ form a basis for}        {$A$:s kolonner utgör en bas för} $\mathbb{R}^n$
  \item \Tr{All eigenvalues of $A$ are different from}{$A$:s egenvärden är alla skilda från} $0$
  \item \Tr{The linear map}{Avbildningen}
        $F(\vec x)=A\vec x$
        \Tr{is invertible}{är omvändbar}
  \end{itemize}
  \vspace{-16pt}
  &
  \begin{itemize}\addtolength{\itemsep}{-5pt}
    \item $A\not\sim I$ ($A$ \Tr{is not row-equivalent to}{är ej radekvivalent med} $I$)
  \item $\mbox{rank}(A)<n$
  \item $\det(A)=0$
  \item $A$ \Tr{is singular (not invertible)}{är singulär (ej inverterbar)}
  \item \Tr{The homogeneous system}{Det homogena systemet}
    $A\vec x=\vec 0$
    \Tr{has infinitely many solutions}{har oändligt många lösningar}
  \item \Tr{The inhomogeneous system}{Det inhomogena systemet}
    $A\vec x=\vec b$
    \Tr{is either inconsistent or has infinitely many solutions}
       {är antingen ölosbart eller har oändligt många lösningar}
  \item \Tr{The rows of $A$ are linearly dependent}   {$A$:s rader är linjärt beroende}
  \item \Tr{The rows of $A$ do not span}              {$A$:s rader spänner inte upp} $\mathbb{R}^n$
  \item \Tr{The columns of $A$ are linearly dependent}{$A$:s kolonner är linjärt beroende}
  \item \Tr{The columns of $A$ do not span}           {$A$:s kolonner spänner inte upp} $\mathbb{R}^n$
  \item $\lambda=0$ \Tr{is an eigenvalue of}{är ett egenvärde till} $A$
  \item \Tr{The linear map}{Avbildningen}
        $F(\vec x)=A\vec x$
        \Tr{is not invertible}{är inte omvändbar}
  \end{itemize}
  \vspace{-16pt}
  \\
  \hline
\end{tabular}%>>>

\subsection*{\Tr{Least squares solution to an overdetermined system}%<<<
                {Minsta kvadratlösning till ett överdeterminerat system}\/ $A_{m\times n} X_n=B_m$, $m>n$}

\Tr{In case the columns of the matrix}{Om kolonnvektorerna av matrisen}
$A$
\Tr{are linearly independent, the unique solution of the square system}
  {är linjärt oberoende, den entydiga lösningen till det kvadratiska systemet}
\[
  A^TAX = A^TB
\]
\Tr{minimizes the error}{minimerar felet} \/ $|AX-B|^2$.%>>>

\label{fin@lpage}

\end{document}

% vim: foldmethod=marker
