%% CHANGES:
%% IvTj, 2012-03-19:     * Rättelse för isometriska avb: F(u*v) = F(u)*F(v)
%%                                                   --> u*v = F(u)*F(v)
%% IvTj, 2010-10-29: 
%%                       * Byte: kolumn(er) --> kolonn(er)
%%                       * Byte: |A| -->  det(A)
%%                       * Ny def på utveckling av det(A) längs en rad
%%                       * |A-\lambda I| --> det(A-\lambda I)
%%                       * .. nollställena till den karakteristiska ekv
%%                       -->  lösningarna till den karakteristiska ekv
%%                        
%% Ändring: Ivtj, 2009-02: lagt till direktformeln för ax^2+bx+c=0
%%                         Lagt till indelningslinjer i tabellerna 
%%                         "Luftigare" tabeller 
%%                         Flyttat om del av "Vektorer" --> sida 1
%%                         Lagt till en till form för kryssprodukten
%%                         Splittrat lin ber/ober.
%%                         Lagt text om A ~ I (radekvivalens med enhetsmatrisen)
%%                         Lagt till Ortogonala matriser
%%                         Ommöblering av huvudsatsen
%%                           Preciserat att A_{n x n}
%%                    Lagt till att A:s både rader och kolonner utgör bas för R^n
%%                           Bytt (in-, sur-, bijektiv) --> omvändbar
%% Ändring: Ivtj, 2008-08: Konverterart t. UTF-8, Tagit bort jth-logot
%%                         No external dependencies: alla LaTeX-def's inbäddade
%%                         Utbyta externdefinierade environments
%%                         Omdef \vec --> vektorer i bold-typpsnitt utan pil
%%                         det A --> det(A), etc
%%                         Krav att egenvektorn u != 0
%%                         Bytt ut definitionen av diagonaliserbar matris,
%%                         ej nödvändigt att kräva skilda egenvärden

\documentclass[a4paper]{article}
%% \usepackage[T1]{fontenc}
%% \usepackage[latin1]{inputenc}
\usepackage[utf8]{inputenc}  % <-- IvTj
\usepackage[swedish]{babel}
\usepackage{amsmath,fancyhdr,amssymb,graphicx} 
%% \usepackage{mycommands}

\addtolength{\topmargin}{-45mm}
\addtolength{\headheight}{20ex}
\addtolength{\headsep}{-3ex}
\addtolength{\hoffset}{-18mm}
\addtolength{\textheight}{30mm}
\addtolength{\marginparwidth}{-20mm}
\addtolength{\textwidth}{55mm}
\footskip 3ex
\parindent 0 cm

\pagestyle{fancy}

%% -- Latex def's:
\let\ent\Leftrightarrow
\def\trevektor[#1,#2,#3]{\begin{pmatrix}#1\cr #2\cr #3\end{pmatrix}}
\def\abs#1{|#1|}
\def\Rone{{\mathbb R}}
\let\R\Rone
\def\vec#1{\mathbf #1} %% <-- omdef \vec -- vektorer i boldface
%%  -- /Latex def's 

%% -- Head && foot
\chead{\ifnum\thepage=1 {} \else Formelblad Linjär Algebra\fi}
\rhead{\ifnum\thepage=1 \textbf{FORMELBLAD Linjär Algebra}
               \else sid. \thepage{} av \pageref{fin@lpage}  \fi}
\lfoot{\ifnum\thepage=1\small\today\fi}
\cfoot{}
\rfoot{\ifnum\thepage=1\small File: \texttt{\jobname.pdf}\fi}


%% -- /Head && foot
\begin{document}

\subsection*{Andragradsekvationer}

Ekvationen $x^2+px+q=0$ har lösningarna
%% \begin{equation*}
  $x_{1,2}=-\frac p2\pm\sqrt{\left(\frac p2\right)^2-q}$\,.
%% \end{equation*}
Då är $x^2+px+q=(x-x_1)(x-x_2)$.

\medskip %% Added by IvTj, 2009-02-13:  
Ekvationen $ax^2+bx+c=0$ har lösningarna
$x_{1,2}=\dfrac{-b\pm\sqrt{b^2-4ac}}{2a}$\,.
Då är $ax^2+bx+c=a(x-x_1)(x-x_2)$.


\subsection*{Trigonometri}
\begin{equation*}
  \begin{array}{|ll|ccccccccc|}
    \hline
    &&&&&&&&&&\\*[-7pt]
    \text{Vinkel }x&(\text{grader})&
    0^\circ&30^\circ&45^\circ&60^\circ&90^\circ&
    120^\circ&135^\circ&150^\circ&180^\circ\\
    &(\text{radianer})&
    0&\dfrac\pi6&\dfrac\pi4&\dfrac\pi3&\dfrac\pi2&
    \dfrac{2\pi}3&\dfrac{3\pi}4&\dfrac{5\pi}6&\pi\\*[3pt]
    &&&&&&&&&&\\*[-7pt]
    \hline
    &&&&&&&&&&\\*[-7pt]
    &\sin x&0&\dfrac12&\dfrac1{\sqrt2}&\dfrac{\sqrt3}2&1&
    \dfrac{\sqrt3}2&\dfrac1{\sqrt2}&\dfrac12&0\\*[3pt]
    &&&&&&&&&&\\*[-7pt]
    &\cos x&1&\dfrac{\sqrt3}2&\dfrac1{\sqrt2}&\dfrac12&0&
    -\dfrac12&-\dfrac1{\sqrt2}&-\dfrac{\sqrt3}2&-1\\*[3pt]
    &&&&&&&&&&\\*[-7pt]
    &\tan x&0&\dfrac1{\sqrt3}&1&\sqrt3&\mbox{odef}&
    -\sqrt3&-1&-\dfrac1{\sqrt3}&0\\*[10pt]
    \hline
  \end{array}
\end{equation*}

\begin{equation*}
  \begin{array}{|l|l|}
    \hline 
    &\\*[-7pt]
    \sin^2x+\cos^2x=1  &\tan x=\dfrac{\sin x}{\cos x}\\ 
    & \\ 
    \sin(-x)=-\sin x &\cos(-x)=\cos x \\ 
    & \\ 
    \sin(\pi-x)=\sin x=\cos(\frac{\pi}{2}-x) &
    \cos(\pi-x)=-\cos x=-\sin(\frac{\pi}{2}-x) \\ 
    & \\ 
    \sin(x\pm y)=\sin x\cos y\pm\sin y\cos x &
    \cos(x\pm y)=\cos x\cos y\mp\sin x\sin y \\ 
    & \\ 
    \sin 2x=2\sin x\cos x&\cos 2x=2\cos^2x-1=1-2\sin^2x \\ 
    & \\
    \sin x\sin y=\frac{1}{2}(\cos(x-y)-\cos(x+y))
    &\sin x\cos y=\frac{1}{2}(\sin(x-y)+\sin(x+y)) \\
    & \\
    \cos x\cos y=\frac{1}{2}(\cos(x-y)+\cos(x+y))\hspace{7 mm}
    &\tan(x\pm y)=\dfrac{\tan x\pm\tan y}{1\mp\tan x\tan y}\\*[10pt]
    \hline
  \end{array} 
\end{equation*}

%% \newpage
\subsection*{Vektorer}

Formlerna i detta avsnitt utgår från 3-dimensionella vektorer. Men
undantag för formlerna för vektorprodukt samt de formler som rör
planet, är de med självklara modifieringar giltiga även för andra
dimensionstal. 

\bigskip
Antag att vektorerna $\vec u=\trevektor[u_1,u_2,u_3]$ och $\vec
v=\trevektor[v_1,v_2,v_3]$ i en ON-bas, $\theta$ är vinkeln mellan
$\vec u$ och $\vec v$ samt $\lambda\in\mathbb R$. Då är:
%% är ett godtyckligt reellt tal. Då gäller:

\bigskip
\begin{tabular}{|p{0.25\linewidth}|p{0.7\linewidth}|}
  \hline &\\*[-8pt]
  Addition &
  $\vec u+\vec v=\trevektor[u_1+v_1,u_2+v_2,u_3+v_3]=\vec v + \vec u$
  \\*[15pt] \hline &\\*[-8pt]
  Multiplikation med tal&
  $\lambda\vec u=\trevektor[\lambda u_1,\lambda u_2,\lambda u_3] = \vec u\lambda$
  \\*[15pt] \hline &\\*[-8pt]
  Längd, norm
  &
  $\abs{\vec u}=\sqrt{u_1^2+u_2^2+u_3^2}$\\*[2pt]
  &$\abs{\lambda\vec u}=\abs \lambda\abs{\vec u}$\\*[0pt]
  \hline
\end{tabular}

\newpage

\subsection*{Vektorer (forts.)}

\begin{tabular}{|p{0.2\linewidth}|p{0.75\linewidth}|}
  \hline &\\*[-8pt]
  Skalärprodukt&$\vec u\cdot\vec v=\abs{\vec u}\abs{\vec v}\cos\theta
  =u_1v_1+u_2v_2+u_3v_3$\\*[2pt]
  Räkneregler:
  &$\vec u\cdot\vec v=\vec v\cdot \vec u$, 
  $(\lambda\vec u)\cdot\vec v=\vec u\cdot(\lambda\vec v)=\lambda(\vec u\cdot\vec v)$\\*[2pt]
  &$\vec u\cdot(\vec v+\vec w)=\vec u\cdot\vec v+\vec u\cdot\vec w$\\*[2pt]
  &$\vec u\cdot\vec u=\abs{\vec u}^2$;
  $|\vec u|=\sqrt{\abs{\vec u}^2}=\sqrt{u_1^2+u_2^2+u_3^2}\ge0$ 
  (med $|\vec u|=0\iff\vec u=\mathbf 0$)\\*[2pt]
  &$\vec u$ och $\vec v$ ortogonala $\ent\vec u\cdot\vec v=0$
\\*[2pt]\hline&\\*[-8pt]
  Projektions\-formeln&Den vinkelräta projektionen $\vec u'$ av
  $\vec u$ på $\vec v$ ges av:
  $\vec u'=\dfrac{\vec u\cdot\vec v}{\abs{\vec v}^2}\vec v$.
  \\*[8pt] \hline &\\*[-10pt]
  Vektor-/kryssprodukt&  %% Tilläg IvTj, 2009-02-13
  $\vec u\times\vec v=\trevektor[u_1,u_2,u_3]\times\trevektor[v_1,v_2,v_3]
  =
  \left(\begin{array}{c}
      \begin{vmatrix}
        u_2 & v_2\\
        u_3 & v_3\\
      \end{vmatrix} 
   \\*[8pt] 
  -   \begin{vmatrix}
        u_1 & v_1\\
        u_3 & v_3\\
      \end{vmatrix} 
    \\*[8pt]
      \begin{vmatrix}
        u_1 & v_1\\
        u_2 & v_2\\
      \end{vmatrix} 
    \end{array}\right) 
  =\trevektor[u_2v_3-u_3v_2,u_3v_1-u_1v_3,u_1v_2-u_2v_1]$\\*[2pt]
  Räkneregler:
  &$\vec u\times\vec v$ är ortogonal mot både $\vec u$ och $\vec v$.\\*[2pt]
  &$\abs{\vec u\times\vec v}=\abs{\vec u}\abs{\vec v}\sin\theta$ anger
  arean av den parallellogram som spänns upp av $\vec u$ \& $\vec v$\\*[2pt]
  &$\vec u\times\vec v=-\vec v\times \vec u$, 
  $(\lambda \vec u)\times\vec v=\vec u\times(\lambda \vec v)=\lambda (\vec u\times\vec v)$\\*[2pt]
  &$\vec u\times(\vec v+\vec w)=\vec u\times\vec v+\vec u\times\vec w$
  \\*[2pt] \hline &\\*[-8pt]
  Skalär trippelprodukt
  &
  $
  \vec u\cdot(\vec v\times\vec w)=
  \begin{vmatrix}
    u_1&v_1&w_1\\
    u_2&v_2&w_2\\
    u_3&v_3&w_3
  \end{vmatrix}
  = \vec w\cdot(\vec u\times\vec v)
  = \vec v\cdot(\vec w\times\vec u)
  = \pm $ volymen av den parallellepiped som spänns upp av vektorerna
  $\vec u$, $\vec v$ och $\vec w$.\\*[0pt]
  \hline
  &\\*[-8pt]
  Linjärt oberoende & Vektorerna $\vec u_1,\dots\vec u_n$ är linjärt 
  oberonde då och endast då det homogena ekvationssysemet
  $\lambda_1\vec u_1+\dots\lambda_n\vec u_n=\vec 0$
  har endast den triviala lösningen
  $\lambda_1=\lambda_2=\cdots\lambda_n=0$.
  \\*[2pt]
  \hline
  &\\*[-8pt]
  Linjärt beroende&Motsatsen till lin. oberoende: 
  Vektorerna $\vec u_1,\dots\vec u_n$ är
  linjärt beronde $\ent$ Det finns skalärer $\lambda_1,\dots\lambda_n$,
  ej alla 0 sådana att $\lambda_1\vec u_1+\dots\lambda_n\vec u_n=\vec
  0$.
  \\*[2pt] \hline &\\*[-8pt]
  Räta linjens ekvation&En rät linje genom punkten $P_0=(x_0,y_0,z_0)$ med
  riktningsvektorn\\
  &$\vec v=\trevektor[v_x,v_y,v_z]$ har på parameterform ekvationen 
  $\trevektor[x,y,z]=\trevektor[x_0,y_0,z_0]+t\trevektor[v_x,v_y,v_z],\,
  t\in\mathbb R$.
  \\*[14pt] \hline &\\*[-8pt]
  Planets ekvation&Ett plan genom punkten $P=(x_0,y_0,z_0)$ med normalrikning
  $\mathbf n=(a,b,c)$ ges av
  \[
  (\mathbf{r} - \mathbf{r}_P)\cdot\mathbf{n} = 0
  \iff
  \left[
     \trevektor[x,y,z]
    -\trevektor[x_0,y_0,z_0]
  \right]
  \cdot
  \trevektor[a,b,c]=0
  \iff
  ax+by+cz=d,
  \]
  där 
  $\mathbf{r}$ är ortsvektorn av godt. punkt i planet,
  $\mathbf{r}_P$ är ortsvektorn av $P$ och
  $d=\mathbf{r}_p\cdot\mathbf{n}$.
  \\*[14pt] %% &\\*[-8pt]
  På parameterform:
  &Ett plan genom punkten $P=(x_0,y_0,z_0)$ utmed (de
  linjärt oberoende) vektorerna $\vec v=(v_x,v_y,v_z)^t$ 
  och $\vec w=(w_x,w_y,w_z)^t$ har parametriska framställningen\\
  &
  $$
  \vec r = \vec r_P + s\vec v + t\vec w
  \iff
  \trevektor[x,y,z]
    = \trevektor[x_0,y_0,z_0]
    +s\trevektor[v_x,v_y,v_z]
    +t\trevektor[w_x,w_y,w_z],
     \,s,t\in\Rone.
  $$
  %%&På parameterfri form är planets ekvation $Ax+By+Cz+D=0$ där $\vec
  %% n=\trevektor[A,B,C]$ är en normalvektor till planet.
  \\*[8pt]
  %En avståndsformel&Avståndet mellan punkten $(x_1,y_1,z_1)$ och planet
  %$Ax+By+Cz+D=0$ är
  %$\dfrac{\abs{Ax_1+By_1+Cz_1+D}}{\sqrt{A^2+B^2+C^2}}$.\\*[8pt]
  \hline
\end{tabular}
\newpage

\subsection*{Linjära ekvationssystem}
\begin{tabular}{|p{0.2\linewidth}|p{0.75\linewidth}|}
  \hline
  Gausseliminering
  &
  Syfte: Att med hjälp av radoperationer ställa systemet på
  trappstegsform.\\
  &
  Tillåtna radoperationer, som inte ändrar på systemets lösningsmängd, är:
  \vspace{-5pt}
  \begin{enumerate}
      \addtolength{\itemsep}{-5pt}
  \item 
    Platsbyte på ekvationer.
  \item 
    Multiplikation av en ekvation med ett tal.
  \item 
    Addition av en multipel av en ekvation till en annan.
  \end{enumerate}
  \vspace{-15pt}
  \\*[2pt]\hline\\*[-9pt]
  Radekvivalens med $I$ & Om en kvadratisk martis kan med tillåtna radoperationer formas om
  till enhetsmatris, kallas $A$ radekvivalent med enhetsmatrisen $I$ (betecknas
  med $A\sim I$).
  \\*[2pt]
  Om $A\sim I$
  & Gausselimination och bakåtelimination vid lösningen av matrisekvationen $AX=B$ resulterar i
  $$
  (A|B)\sim\cdots\sim (I|X).
  $$
  \vspace{-15pt}
  \\
  \hline
\end{tabular}

\subsection*{Matriser}

\begin{tabular}{|p{0.2\linewidth}|p{0.75\linewidth}|}
  \hline & \\*[-9pt]
  Addition & $A+B=B+A$, $(A+B)+C=A+(B+C)$
  \\*[2pt]\hline & \\*[-9pt]
  Mult. med tal & $\lambda(A+B)=\lambda A+\lambda B$
  \\*[2pt]\hline & \\*[-9pt]
  Multiplikation&$AB\ne BA$ (i allmänhet) 
  \\*[2pt]
  &$(AB)C=A(BC)$, \ $A(B+C)=AB+AC$, \  $(A+B)C=AC+BC$
  \\*[2pt]\hline & \\*[-9pt]
  Enhetsmatrisen&$I=
  \begin{pmatrix}
    1&0&\cdots&0\\
    0&1&\cdots&0\\
    \vdots&\vdots&\ddots&\vdots\\
    0&0&\cdots&1
  \end{pmatrix}
  $, $AI=IA=A$
  \\*[24pt]\hline & \\*[-9pt]
  Invers &$AA^{-1}=A^{-1}A=I$, $(AB)^{-1}=B^{-1}A^{-1}$\\*[3pt]
  Jacobis metod:& Med radoperationer överförs systemet $(A|I)\sim \cdots\sim (I|A^{-1})$.
  \\*[3pt]
  Ortogonala matriser & $A^{-1}=A^T$.
  \\*[3pt]
  $2\times 2$--matriser:&
  $A_{2\times2}^{-1}
  =\begin{pmatrix}a&b\\c&d\end{pmatrix}^{-1}
  =\dfrac1{\det(A)}\begin{pmatrix}d&-b\\-c&a\end{pmatrix}
  =\dfrac1{ad-bc}\begin{pmatrix}d&-b\\-c&a\end{pmatrix}$
  \\*[9pt]\hline  & \\*[-7pt]
  Transponering &
  Rader och kolonner byter plats, t\,ex, 
  $ \begin{pmatrix}a_{11}&a_{12}&a_{13}\\a_{21}&a_{22}&a_{23}\end{pmatrix}^T=
    \begin{pmatrix}a_{11}&a_{21}\\a_{12}&a_{22}\\a_{13}&a_{23}\\\end{pmatrix}$\\*[3pt]
  &$(A+B)^T=A^T+B^T$, \ 
  $(AB)^T=B^TA^T$, \ 
  $(A^{-1})^T=(A^T)^{-1}$\\*[2pt]
  \hline
\end{tabular}
\subsection*{Determinanter}
\begin{tabular}{|p{0.2\linewidth}|p{0.75\linewidth}|}
  \hline
   & \\*[-8pt]
  $2\times2$--matriser &Om $A=\begin{pmatrix}a&b\\c&d\end{pmatrix}$ är 
  $\det(A)=\begin{vmatrix}a&b\\c&d\end{vmatrix}=ad-bc$.
  \\*[9pt] \hline & \\*[-15pt]
  \vfill
  Utveckling efter rad $i$ &
  $$
  \det(A_{n\times n})
  = a_{i1}(-1)^{i+1}D_{i1}
  + \cdots + a_{ij}(-1)^{i+j}D_{ij}
  + \cdots + a_{in}(-1)^{i+n}D_{in},
 $$ 
 där
 $D_{ij}$ är determinanten som fås 
 genom att i $\det(A)$ stryka rad $i$ och kolonn $j$.
  \\*[5pt] \hline & \\*[-8pt]
%   \\*[9pt] \hline & \\*[-8pt]
%   Utveckling efter rad &
%   Exempel (utv. längs rad 1): \  
%   $
%   \begin{vmatrix}
%     a&b&c\\d&e&f\\g&h&i
%   \end{vmatrix}=a
%   \begin{vmatrix}
%     e&f\\
%     h&i
%   \end{vmatrix}
%   -b
%   \begin{vmatrix}
%     d&f\\
%     g&i
%   \end{vmatrix}
%   +c
%   \begin{vmatrix}
%     d&e\\
%     g&h
%   \end{vmatrix}
%   $
%  \\*[15pt] \hline & \\*[-8pt]
  Räkneregler &
  $\det(A^T)=\det(A)$ (dvs, allt som är tillåtet att göra 
                         radvis fungerar även kolonnvis);
  \\*[2pt]
  &
  $\det(AB)=\det(A)\det(B)$, 
  $\det(A^{-1})=1/\det(A)$
  \\*[2pt]
  &
  \parbox{10em}{Multipel av en rad eller kolonn kan bruytas ut:}
  \ $
  \begin{vmatrix}
    a&\lambda b&c\\d&\lambda e&f\\g&\lambda h&i
  \end{vmatrix}
  =\lambda 
  \begin{vmatrix}
    a&b&c\\d&e&f\\g&h&i
  \end{vmatrix},\quad
  \begin{vmatrix}
    a&b&c\\d&e&f\\\lambda g&\lambda h&\lambda i
  \end{vmatrix}
  =\lambda 
  \begin{vmatrix}
    a&b&c\\d&e&f\\g&h&i
  \end{vmatrix}.
  $\\*[15pt]
  &Om två rader eller två kolonner byter plats skiftar determinanten
  tecken.
  \\*[2pt]
  &En multipel av en rad kan läggas till en annan rad utan att determinanten ändras.
  \\*[2pt]
  &En multipel av en kolonn kan läggas till en annan utan att determinanten ändras.
\\
  \hline
\end{tabular}

\subsection*{Linjära avbildningar}

\begin{tabular}{|p{0.2\linewidth}|p{0.75\linewidth}|}
  \hline\\*[-9pt]
  Definition&$F$ är en linjär avbildning $\ent F(\vec u+\vec
  v)=F(\vec u)+F(\vec v)$ och $F(\lambda\vec u)=\lambda F(\vec u)$.\\
  &$F$ är en linjär avbildning $\ent F(\vec u)=A\vec u$ där $A$ är en
  konstant matris.
  \\*[2pt] \hline\\*[-9pt]
  Sammansättning&Om $F(\vec u)=A\vec u$ och
  $G(\vec u)=B\vec u$ är
  $G\bigl(F(\vec u)\bigr)
  =G\bigl(A\vec u\bigr)
  =BA\vec u
  $.
  \\*[2pt] \hline\\*[-9pt]
  Isometriska&$F:\R^n\to\R^n,F(\vec u)=A\vec u$ är isometrisk $\ent
  \abs{F(\vec u)}=\abs{\vec u}$ för alla $\vec u\in\R^n\ent$\\
  avbildningar&$\vec u\cdot\vec v=F(\vec u)\cdot
  F(\vec v)$ för alla $\vec u$ och $\vec v\in\R^n\ent A$ är en ON-matris 
  $\ent$\\ 
  &$A^{-1}=A^T\ent A$:s kolonner (rader) utgör en ON-bas för $\R^n$.\\
  \hline
\end{tabular}

\subsection*{Baser och basbyten}
\begin{tabular}{|p{0.2\linewidth}|p{0.75\linewidth}|}
  \hline\\*[-9pt]
  Baser&En uppsättning av $n$ linjärt oberoende vektorer $\vec e_1,\dots,\vec
  e_n$ bildar en \emph{bas} för $\R^n$. Till varje $\vec u\in\R^n$ finns
  då reella tal $x_1,\dots,x_n$ så att $\vec u=x_1\vec e_1\dots x_n\vec
  e_n$. Talen $x_1,\dots,x_n$ är $\vec u$:s koordinater i basen 
  $\vec e_1,\dots,\vec e_n$.
  \\*[2pt] \hline\\*[-9pt]
  Basbyten&Om $u$ har koordinaterna $x_1,\dots,x_n$ i basen $\vec e_1,\dots,\vec
  e_n$ och koordinaterna $y_1,\dots,y_n$ i basen $\vec f_1,\dots,\vec
  f_n$, är $\trevektor[x_1,\vdots,x_n]
           =P\trevektor[y_1,\vdots,y_n]$ där
  basbytesmatrisen $P=
  \begin{pmatrix}
    |&&|\\
    \vec f_1&\dots&\vec f_n\\
    |&&|
  \end{pmatrix}$ med $\vec f_1,\dots,\vec f_n$ uttryckta i basen
  $\vec e_1,\dots,\vec e_n$.
  \\*[2pt] \hline\\*[-9pt]
  Avbildningsmatris vid basbyte&Om den linjära avbildningen $F$ har
  avbildningsmatris $A$ i basen $\vec e_1,\dots,\vec e_n$ och
  avbildningsmatris $A'$ i basen $\vec f_1,\dots,\vec f_n$, gäller
  $A'=P^{-1}AP$, där $P$ är basbytesmatrisen definierad ovan.\\
  \hline
\end{tabular}

\subsection*{Egenvärden och egenvektorer}
\label{subsection*}
\begin{tabular}{|p{0.2\linewidth}|p{0.75\linewidth}|}
  \hline\\*[-9pt]
  Definition&Matrisen $A$ har \emph{egenvektorn} $\vec u\neq\vec0$ med
  \emph{egenvärdet} $\lambda$ om
  $A\vec u=\lambda\vec u$.
  \\*[2pt] \hline\\*[-9pt]
  Egenvärdena $\lambda_k$ 
   & bestäms som lösningarna till den karakteristiska ekvationen 
   $\det(A-\lambda I)=0$.\\
  Egenvektorerna $\vec u_k$& bestäms ur det homogena systemet $(A-\lambda I)\vec u=\mathbf{0}$.
  \\*[2pt] \hline\\*[-9pt]
%%   Diagonalisering&Om $n\times n$-matrisen $A$ har $n$ olika egenvärden
%%   $\lambda_1,\dots,\lambda_n$ med motsvarande egenvektorer
%%   $\vec u_1,\dots,\vec u_n$, är $A=PDP^{-1}$ där
Diagonalisering&Om $n\times n$-matrisen $A$ har $n$ st linjärt oberoende
egenvektorer $\vec u_1,\dots,\vec u_n$ med motsvarande egenvärden
$\lambda_1,\dots,\lambda_n$, är $A=PDP^{-1}$ där
\vspace{-10pt}
  $$ P=
  \begin{pmatrix}
    |&&|\\
    \vec u_1&\dots&\vec u_n\\
    |&&|
  \end{pmatrix}
  \mbox{ \ och \ \ }
  D = \begin{pmatrix}
    \lambda_1&\dots&0\\
    \vdots&\ddots&\vdots\\
    0&\dots&\lambda_n
  \end{pmatrix}.
  $$
  \vspace{-12pt}
  \\
  \hline
\end{tabular}

\subsection*{Huvudsatsen}

%% Låt $A$ vara en $n\times n$--matris.\\*[10pt]
\begin{tabular}{|p{0.475\linewidth}|p{0.475\linewidth}|}
  \hline
  &\\*[-10pt]
%  Låt $A$ vara en $n\times n$--matris.\hfill\break
  Ekvivalenta påståenden för en $n\times n$--matris $A$:
& % Låt $A$ vara en $n\times n$--matris.\hfill\break
  %% Följande påståenden är ekvivalenta:
  Ekvivalenta påståenden för en $n\times n$--matris $A$:
  \\
  \hline
  \\*[-24pt]
  \begin{itemize}\addtolength{\itemsep}{-5pt}
  \item $\det(A)\ne0$
  \item $A$ är inverterbar
  \item $A\sim I$ (dvs $A$ är radekvivalent med $I$)
  \item Ekvationssystemet $A\vec x=\vec 0$ har endast den triviala
    lösningen $\vec x=\vec 0$
  \item Ekvationssystemet $A\vec x=\vec b$ har en entydig lösning för
    varje val av högerledet $\vec b$
  \item $A$:s rader är linjärt oberoende
  \item $A$:s rader utgör en bas för $\mathbb{R}^n$
  \item $A$:s kolonner är linjärt oberoende
  \item $A$:s kolonner utgör en bas för $\mathbb{R}^n$
  \item Avbildningen $F(\vec x)=A\vec x$ är omvändbar % injektiv, surjektiv och inverterbar.
  \item $A$:s egenvärden är alla skilda från $0$
  \end{itemize}
  \vspace{-16pt}
  &
  \begin{itemize}\addtolength{\itemsep}{-5pt}
  \item $\det(A)=0$
  \item $A$ är inte inverterbar
  \item $A$ är inte radekvivalent med $I$
  \item Ekvationssystemet $A\vec x=\vec 0$ har oändligt många lösningar
  \item Ekvationssystemet $A\vec x=\vec b$ saknar lösning eller också
    har det oändligt många lösningar
  \item $A$:s rader är linjärt beroende
  \item $A$:s rader utgör inte bas för $\mathbb{R}^n$
  \item $A$:s kolonner är linjärt beroende
  \item $A$:s kolonner utgör inte bas för $\mathbb{R}^n$
  \item Avbildningen $F(\vec x)=A\vec x$ är inte omvändbar % varken injektiv, surjektiv eller inverterbar.
  \item $0$ är ett egenvärde till $A$
  \end{itemize}
  \vspace{-16pt}
  \\
  \hline
\end{tabular}

\label{fin@lpage}


\end{document}
