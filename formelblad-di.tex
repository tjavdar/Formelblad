% vim: foldmethod=marker:spelllang=sv:spell
% ------------------------------------------------------------------
% Formelblad Matematiska verktyg
% Version 0.5 IvTj, 2011-01-13:


\documentclass{article}

\usepackage[utf8]{inputenc}
\usepackage{a4,graphicx,amssymb}
\usepackage{array} % For fancy tables
%%% instead of \usepackage{fullpage} <<<
\topmargin 0pt
\advance \topmargin by -\headheight
\advance \topmargin by -\headsep
\textheight 8.9in
\oddsidemargin 0pt
\evensidemargin \oddsidemargin
\marginparwidth 0.5in
\textwidth 6.5in
% >>>

\pagestyle{headings}
\textheight=720pt\parindent=0pt


%% <<< Former \input\jobname.h ----------------

%-------------
%- LaTeX def's
%-------------

\let\UP\nearrow
\let\DN\searrow
\everymath{\displaystyle}

%--------------------------- Facit macros ------

\def\contentsname{} % Erase it
\def\Facit#1{\par

\addtocontents{toc}{{\small\protect\item[\textbf{\theenumi.}]#1\par\smallskip\hrule\par}}}
\def\BeginUppg{\par\addtocontents{toc}{\par\protect\begin{enumerate}}
               \par\begin{enumerate}}
\def\EndUppg{\addtocontents{toc}{\protect\end{enumerate}}\par\end{enumerate}}

\newcommand{\mymatrix}[2]{\left(\begin{array}{#1} #2\end{array}\right)}
\newcommand{\mydet}[2]{\left|\begin{array}{#1} #2\end{array}\right|}

\newcommand\dd[2]{\frac{\partial #1}{\partial #2}}
\newcommand\lvec[1]{\stackrel{\longrightarrow}{#1}}

\newcommand\tightSvar[1]{\fbox{\textbf{A:} #1}}
\let\Svar\tightSvar

%--------------------------- end Facit macros ------

\makeatletter
\renewcommand{\@oddhead}{}
\renewcommand{\@evenhead}{}
\renewcommand{\@oddfoot}
{\ifnum\thepage=1
  \today\hfill file:~\small\texttt{\jobname.pdf}
\else
  Formelblad Trans \& Stat \hfill s.~\thepage{} av \pageref{LastPageNo}
\fi}

  \renewcommand{\@evenfoot}{\small
        \texttt{\jobname.pdf},\hfill}

\makeatother

%% restore good old TeX \eqalign:
\makeatletter

\def\eqalign#1{\null\,\vcenter{\openup\jot\m@th
  \ialign{\strut\hfil$\displaystyle{##}$&$\displaystyle{{}##}$\hfil
      \crcr#1\crcr}}\,}

\def\iint{\mathop{\relax\protect
    \noexpand\intop\mkern-9mu\noexpand\intop}\displaylimits}

\def\iiint{\mathop{\relax\protect
    \noexpand\intop\mkern-9mu\noexpand\intop\mkern-9mu\noexpand\intop}\displaylimits}


\makeatother


\def\ppmatrix{\protect\pmatrix}
\def\pcases{\protect\cases}
\newcommand\pdet[1]{\protect\left|\protect\matrix{#1}\protect\right|}
\newcommand\Ordo{\mathcal O}
\let\ergo\Longrightarrow
\let\vaxer\nearrow
\let\avtar\searrow
\newcommand\conj[1]{{\overline #1}}
\let\ob\conj
\newcommand\binom[2]{{#1\choose #2}}

%% Boldface things %%

\newcommand\bdF{\mathbf F}
\newcommand\bdE{\mathbf E}
\newcommand\bdC{\mathbf C}
\newcommand\bdS{\mathbf S}
\newcommand\bdG{\mathbf G}
\newcommand\bdN{\mathbf N}

\newcommand\bda{\mathbf a}
\newcommand\bdb{\mathbf b}
\newcommand\bdc{\mathbf c}
\newcommand\bde{\mathbf e}
\newcommand\bdf{\mathbf f}
\newcommand\bdg{\mathbf g}
\newcommand\bdh{\mathbf h}
\newcommand\bdi{\mathbf i}
\newcommand\bdj{\mathbf j}
\newcommand\bdk{\mathbf k}
\newcommand\bdm{\mathbf m}
\newcommand\bdn{\mathbf n}
\newcommand\bdp{\mathbf p}
\newcommand\bdq{\mathbf q}
\newcommand\bdr{\mathbf r}
\newcommand\bdu{\mathbf u}
\newcommand\bdv{\mathbf v}
\newcommand\bdx{\mathbf x}
\newcommand\bdy{\mathbf y}
\newcommand\bdw{\mathbf w}
\newcommand\bdzero{\mathbf 0}

% - sets -
\def\Rone{{\mathbb R}}
\def\Cone{{\mathbb C}}
\def\Zone{{\mathbb Z}}

% - operators -

\def\norm#1{{\Vert #1\Vert}}
\def\SP#1{\langle #1\rangle}
\def\Ordo{\mathcal O}
\def\Fourier#1{\mathcal F\left\{#1\right\}}
\def\Laplace#1{\mathcal L\left\{#1\right\}}
\def\invFourier#1{\mathcal F^{-1}\left\{#1\right\}}
\def\invLaplace#1{\mathcal L^{-1}\left\{#1\right\}}
\let\invFF\invFourier
\let\FF\Fourier
\let\invLL\invLaplace
\let\LL\Laplace

% - förkortnngar

\newcommand\mha{med hjälp av }
\newcommand\bis{^{\prime\prime}}
\newcommand\triss{^{'''}}
\newcommand\PartInt[2]{\left\lceil\matrix{\mbox{\small #1}\cr
                                       \mbox{\small #2}}\right\rceil}
\def\tfrac{\textstyle\frac}

%adjust row height in tables #1 - overall height #2 - depth:
\newcommand\TblHeight[2]{\lower#2em\vbox to#1em{\hsize=0pt}}
\def\tvavektor[#1,#2]{\protect\pmatrix{#1\cr #2}}
\def\trevektor[#1,#2,#3]{\protect\pmatrix{#1\cr #2\cr #3}}
\let\ob\overline
\let\iff\Leftrightarrow


%% End \input\jobname.h fold >>> -


\let\ergo\Rightarrow

\begin{document}


\subsection*{Formelblad Diskret matematik}
\hrule
\bigskip

\subsubsection*{Binomialsatsen. Geometrisk summa. Eulers polyederformel för plana grafer $G(V,E)$} %<<<
\vspace{-2em}
\[
  \begin{array}[t]{|l|l|l|}
\hline
  (a+b)^n = \sum_{k=0}^{n} \binom{n}{k} a^{n-k}b^k &
  \sum_{k=0}^{n} x^k = 1 + x  + \cdots +x^n = \frac{x^{n+1}-1}{x-1},\quad x \neq 1 &
  %\sum_{k=0}^{\infty} z^k = \frac{1}{1-z},\quad |z| < 1
  |V|-|E|+ \#(\mbox{faces}) = 2
\\\hline
  \end{array}
\]%>>>

\subsubsection*{Arithmetic mod $n$ %<<<
\textnormal{(i tabellen nedan är alla tal hela, $m>1,\,n>1$, och $p$ är ett primtal):}}
\vspace{-2.2em}
\[
  \begin{array}[t]{|*{4}{l|}}
    \hline \rule{0pt}{12pt}
 %   \Zone = \{0,\pm1,\pm2,\cdots\}
      [a]_n=[b]_n\iff a+tn = b,\, t\in \Zone
    & [a+b]_n = [[a]_n + [b]_n]_n
    & [ab]_n = [[a]_n[b]_n]_n
    \cr\hline \rule{0pt}{12pt}
      \Zone_n=\{[0]_n,\cdots,[n-1]_n\}
    & a\perp b \iff \gcd(a,b)=1
    & U_n=\{x\in\Zone_n\!: x\perp n\} % ,\, |U_n|=\phi(n)
    \\[1pt] \hline
      \rule{0pt}{16pt}
      \phi(n) =  |U_n| = n\prod_{p|n}\Bigl(1-\frac1p\Bigr)
      & \phi(p^k) = (p-1)p^{k-1}
    & m\perp n\ergo \phi(mn)=\phi(m)\phi(n)
    \\ \hline
    \multicolumn{3}{|l|}{\rule{0pt}{12pt}
      \begin{array}[m]{l rl|rl}
        &\mbox{Fermats lilla sats:} & a\perp p\ergo [a^{p-1}]_p=[1]_p
        &\mbox{Eulers sats:}        & a\perp n\ergo [a^{\phi(n)}]_n=[1]_n
      \end{array}
    }
    \\[2pt] \hline
  \end{array}
\]%>>>

\subsubsection*{De grundläggande logiska konnektiven och deras sanningsvärdestabeller}%<<<

\begin{tabular}[m]{|l*{3}{|c}|}%<<<
\cline{2-4}
\multicolumn{1}{c|}{} & \textbf{och} &\textbf{eller} & \textbf{inte}\\
\hline
Engelska/Python &   \textbf{and}    &   \textbf{or}     &   \textbf{not}  \\
\hline
  Logiksymbol      & $\wedge$ & $\vee$ & $\neg$ \\
\hline
  Mängdsymbol      & $\cap$   & $\cup$ &\raise5pt\hbox{$\rule{8pt}{0.72pt}$} \\
\hline
Boolesk symbol   & $\cdot$  &    $+$   & \raise5pt\hbox{$\rule{8pt}{0.72pt}$} \\
\hline
C/C++ m.fl.      & \&\& & $||$ & ! \\
\hline
 & & & \\[-8pt]
 \raise5pt\hbox{Gate (en)\,/\,grind (sv)}
  & \lower0pt\hbox{\includegraphics[scale=0.7]{Figs/gate_and}}
  & \lower0pt\hbox{\includegraphics[scale=0.7]{Figs/gate_or}}
  & \lower0pt\hbox{\includegraphics[scale=0.7]{Figs/gate_not}}
     \\
\hline
\end{tabular}%>>>
\hfil
\begin{tabular}[m]{|*{7}{>{$}c<{$}|}}%<<<
  \hline
    p & q & \neg p & p \wedge q & p \vee q & p \to q & p \leftrightarrow q\\
  \hline
    0 & 0 &    1   &       0    &     0    &    1    &         1     \\
  \hline
    0 & 1 &    1   &       0    &     1    &    1    &         0     \\
  \hline
    1 & 0 &    0   &       0    &     1    &    0    &         0     \\
  \hline
    1 & 1 &    0   &       1    &     1    &    1    &         1     \\
  \hline
  \multicolumn{7}{l}{\rule{0pt}{12pt}\textbf{Ekvivalenser}:} \\
  \multicolumn{7}{l}{
              $E_{10}$: $p\to q \,\iff\, \neg p \vee q \,\iff\, \neg q\to \neg  p$
                     } \\
  \multicolumn{7}{l}{$E_{11}$: $p \leftrightarrow q
                      \,\iff\,
                      (p\to q) \wedge ( q \to p )
  $}
\end{tabular}%>>>
%>>>

\subsubsection*{Ekvivalenser i mängdläran, logiken och i den booleska algebran}%<<<
\vspace{-1em}
\begin{tabular}[t]{|l|*{3}{>{$}c<{$}|}}
\hline
\textbf{Lagar}
 & \textbf{Mängdlära} & \textbf{Satslogik} &\textbf{Boolesk algebra} \\
\hline
$E_{0}$:
Dubbelnegation \rule{0pt}{10pt}
               &  \scriptstyle  \ob{(\ob P)} = P
               &  \neg (\neg p) \iff p & \ob{\ob p} = p \\
\hline
$E_{1}$:
Idempotens
                & \scriptstyle  P\cup P = P
                & p\vee p \iff p & p + p = p\\
                & \scriptstyle  P\cap P = P
                & p\wedge p \iff p & p^2 = p\\
\hline
$E_{2}$:
Invers
                & \scriptstyle  P\cup \ob P =\mkern2mu\mathcal U
                & p\vee\neg p \iff 1   & p + \ob p = 1\\
                & \scriptstyle  P\cap \ob P =\varnothing
                & p\wedge\neg p \iff 0 & p\,\ob p = 0\\
\hline
$E_{3}$:
Dominans
                & \scriptstyle  P \cup \mkern2mu\mathcal U = \mkern2mu\mathcal U
                & p \vee 1 \iff 1 & p + 1 = 1 \\
                & \scriptstyle  P \cap \varnothing = \varnothing
                & p\wedge0 \iff 0 & p \cdot 0  = 0 \\
\hline
$E_{4}$:
Identitet
                & \scriptstyle  P \cup \varnothing = P
                & p \vee 0 \iff p  & p + 0 = p \\
                & \scriptstyle  P \cap \mkern2mu\mathcal U = P
                & p\wedge 1 \iff p & p\cdot 1 = p\\
\hline
$E_{5}$:
Kommutativa
                  & \scriptstyle  P\cup Q=Q\cup P
                  & p \vee q \iff q \vee p & p + q = q + p \\
                  & \scriptstyle  P\cap Q=Q\cap P
                  & p \wedge q \iff q \wedge p & pq = qp\\
\hline
$E_{6}$:
Associativa
                  & \scriptstyle  (P\cup Q)\cup R = P\cup(Q\cup R)
                  & (p \vee q) \vee r \iff p \vee (q \vee r)
                  & (p + q) + r = p + (q + r) \\
                  & \scriptstyle  (P\cap Q)\cap R = P\cap(Q\cap R)
                  & (p \wedge q)\wedge r \iff p \wedge (q\wedge r)
                  & (pq)r = p(qr)\\
\hline
$E_{7}$:
de Morgan \rule{0pt}{10pt}
                  & \scriptstyle  \ob{P\cup Q} = \ob P \cap \ob Q
                  & \neg ( p \vee q ) \iff \neg p \wedge \neg q
                  & \ob {p + q}  = \ob p\,\ob q \\
                  & \scriptstyle  \ob{P\cap Q} = \ob P \cup \ob Q
                  & \neg ( p \wedge q ) \iff \neg p \vee \neg q
                  & \ob {pq}  = \ob p+\ob q \\
\hline
$E_{8}$:
Distributiva
                  & \scriptstyle (P\cup Q)\cap R = (P\cap R)\cup(Q\cap R)
                  & p \vee (q \wedge r ) \iff (p \vee q) \wedge (p \vee r)
                  & p + qr = (p + q)(p + r) \\
                  & \scriptstyle (P\cap Q)\cup R = (P\cup R)\cap(Q\cup R)
                  & p \wedge (q\vee r)\iff (p \wedge q) \vee (p\wedge r)
                  & p(q+r) = pq+pr\\
\hline
$E_{9}$:
Absorbering
                  & \scriptstyle  P\cup (P\cap Q)=P
                  & p \vee (p \wedge q ) \iff p
                  & p + pq = p \\
                  & \scriptstyle  P\cap (P\cup Q)=P
                  & p \wedge (p\vee q)\iff p
                  & p(p+q) = p\\
\hline
\end{tabular}
%>>>

\subsubsection*{Inferenslagar (Logiska implikationer)}%<<<
\vspace{-2em}
\[
\begin{array}[t]{|*{2}{rrcl|}}
  \hline \rule{0pt}{11pt}
    L_0\!: & \lnot p\to F_0 &\ergo& p
  & L_4\!: & p &\ergo& p\lor q \\
    L_1\!: & p\land (p\to q) &\ergo& q
  & L_5\!: & p\land q &\ergo& p \\
    L_2\!: & (p\to q)\land\lnot q &\ergo& \lnot p
  & L_6\!: & (p\to q)\land(q\to r) &\ergo&  (p\to r) \\
    L_3\!: & \lnot p \land (p\lor q) &\ergo& q
  & L_7\!: & (p\to r) \land (q\to r)\land (p\lor q) &\ergo&  r
  \\[1pt]
  \hline
\end{array}
\quad
\begin{array}[t]{|l|}
\hline \rule{0pt}{12pt}
  \textbf{Predikatlogik} \\[4pt]
  \neg [\forall x\,P(x)]  \iff   \exists x\,\neg  P(x)  \\[4pt]
  \neg [\exists x\,P(x)]  \iff   \forall x\,\neg  P(x)
\\[2pt]
\hline
\end{array}
\]
%>>>

%% \subsubsection*{Inferenslagar (Logiska implikationer)} % variant <= ht-20 %<<<
%% \vspace{-1em}
%% \begin{array}[t]{|*{2}{rrcl|}}
%%   \hline \rule{0pt}{11pt}
%%     L_0\!: & p\land (p\to q) &\ergo& q
%%   & L_5\!: & p &\ergo& p\lor q \\
%%     L_1\!: & (p\to q)\land\lnot q &\ergo& \lnot p
%%   & L_6\!: & (p\to q) \land (r\to s) \land (p\lor r) &\ergo& q\lor s \\
%%     L_2\!: & (p\to q)\land(q\to r) &\ergo&  (p\to r)
%%   & L_7\!: & (p\to q) \land (r\to s) \land (\lnot q\lor \lnot s) &\ergo& \lnot p\lor \lnot r \\
%%     L_3\!: & \lnot p \land (p\lor q) &\ergo& q
%%   & L_8\!: & \lnot p\to F_0 &\ergo& p \\
%%     L_4\!: & p\land q &\ergo& p
%%   & L_9\!: & (p\to r) \land (q\to r) &\ergo& (p\lor q)\to r \\[1pt]
%%   \hline
%% \end{array}
%% %>>>

%%% \subsubsection*{Inferenslagar (Logiska implikationer)}%<<<
%%% \begin{tabular}[t]{*{4}{>{$}l<{$}}l}
%%%   L_0
%%%   & p\land (p\to q) &\ergo& q
%%%   & \emph{Modus Ponens}
%%%   \cr
%%%   L_1
%%%   & (p\to q)\land\lnot q &\ergo& \lnot p
%%%   & \emph{Modus Tollens}
%%%   \cr
%%%   L_2
%%%   & (p\to q)\land(q\to r) &\ergo&  (p\to r)
%%%   & \emph{Hypotetisk syllogism}
%%%   \cr
%%%   L_3
%%%   & \lnot p \land (p\lor q) &\ergo& q
%%%   & \emph{Disjunktiv syllogism}
%%%   \cr
%%%   L_4
%%%   & p\land q &\ergo& p
%%%   & \emph{Konunktiv förenkling}
%%%   \cr
%%%   L_5
%%%   & p &\ergo& p\lor q
%%%   & \emph{Disjunktiv förstärkning}
%%%   \cr
%%%   L_6
%%%   & (p\to q) \land (r\to s) \land (p\lor r) &\ergo& q\lor s
%%%   & \emph{Konstruktiv dilemma}
%%%   \cr
%%%   L_7
%%%   & (p\to q) \land (r\to s) \land (\lnot q\lor \lnot s) &\ergo& \lnot p\lor \lnot r
%%%   & \emph{Destruktiv dilemma}
%%%   \cr
%%%   L_8
%%%   & \lnot p\to F_0 &\ergo& p
%%%   & \emph{Reductio ad absurdum} (motsägelse)
%%%   \cr
%%%   L_9
%%%   & (p\to r) \land (q\to r) &\ergo& (p\lor q)\to r
%%%   & \emph{Tudelad bevis} (by cases)
%%%   \cr
%%% \end{tabular}%>>>

\end{document}
