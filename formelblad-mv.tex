% vim: foldmethod=marker:spelllang=sv:spell
% ------------------------------------------------------------------
% Formelblad Matematiska verktyg 
% Version 0.5 IvTj, 2011-01-13: 


\documentclass{article}

\usepackage[utf8]{inputenc}
\usepackage{a4,graphicx,amssymb}
\usepackage{array} % For fancy tables
%%% instead of \usepackage{fullpage} <<<
\topmargin 0pt
\advance \topmargin by -\headheight
\advance \topmargin by -\headsep
\textheight 8.9in
\oddsidemargin 0pt
\evensidemargin \oddsidemargin
\marginparwidth 0.5in
\textwidth 6.5in
% >>>

\pagestyle{headings}
%%\everymath{\displaystyle}
%% \everymath{\textstyle}
\textheight=720pt\parindent=0pt


%% <<< Former \input\jobname.h ----------------

%-------------
%- LaTeX def's
%-------------

\let\UP\nearrow
\let\DN\searrow
\everymath{\displaystyle}

%--------------------------- Facit macros ------

\def\contentsname{} % Erase it
\def\Facit#1{\par

\addtocontents{toc}{{\small\protect\item[\textbf{\theenumi.}]#1\par\smallskip\hrule\par}}}
\def\BeginUppg{\par\addtocontents{toc}{\par\protect\begin{enumerate}}
               \par\begin{enumerate}}
\def\EndUppg{\addtocontents{toc}{\protect\end{enumerate}}\par\end{enumerate}}

\newcommand{\mymatrix}[2]{\left(\begin{array}{#1} #2\end{array}\right)}
\newcommand{\mydet}[2]{\left|\begin{array}{#1} #2\end{array}\right|}

\newcommand\dd[2]{\frac{\partial #1}{\partial #2}}
\newcommand\lvec[1]{\stackrel{\longrightarrow}{#1}}

\newcommand\tightSvar[1]{\fbox{\textbf{A:} #1}}
\let\Svar\tightSvar

%--------------------------- end Facit macros ------

\makeatletter
\renewcommand{\@oddhead}{}
\renewcommand{\@evenhead}{}
\renewcommand{\@oddfoot}
{\ifnum\thepage=1 
  \today\hfill file:~\small\texttt{\jobname.pdf}
\else
  Formelblad Trans \& Stat \hfill s.~\thepage{} av \pageref{LastPageNo} 
\fi}
      
  \renewcommand{\@evenfoot}{\small
        \texttt{\jobname.pdf},\hfill}

\makeatother

%% restore good old TeX \eqalign:
\makeatletter

\def\eqalign#1{\null\,\vcenter{\openup\jot\m@th
  \ialign{\strut\hfil$\displaystyle{##}$&$\displaystyle{{}##}$\hfil
      \crcr#1\crcr}}\,}

\def\iint{\mathop{\relax\protect
    \noexpand\intop\mkern-9mu\noexpand\intop}\displaylimits}

\def\iiint{\mathop{\relax\protect
    \noexpand\intop\mkern-9mu\noexpand\intop\mkern-9mu\noexpand\intop}\displaylimits}


\makeatother


\def\ppmatrix{\protect\pmatrix}
\def\pcases{\protect\cases}
\newcommand\pdet[1]{\protect\left|\protect\matrix{#1}\protect\right|}
\newcommand\Ordo{\mathcal O}
\let\ergo\Longrightarrow
\let\vaxer\nearrow
\let\avtar\searrow
\newcommand\conj[1]{{\overline #1}}
\let\ob\conj
\newcommand\binom[2]{{#1\choose #2}}

%% Boldface things %%

\newcommand\bdF{\mathbf F}
\newcommand\bdE{\mathbf E}
\newcommand\bdC{\mathbf C}
\newcommand\bdS{\mathbf S}
\newcommand\bdG{\mathbf G}
\newcommand\bdN{\mathbf N}

\newcommand\bda{\mathbf a}
\newcommand\bdb{\mathbf b}
\newcommand\bdc{\mathbf c}
\newcommand\bde{\mathbf e}
\newcommand\bdf{\mathbf f}
\newcommand\bdg{\mathbf g}
\newcommand\bdh{\mathbf h}
\newcommand\bdi{\mathbf i}
\newcommand\bdj{\mathbf j}
\newcommand\bdk{\mathbf k}
\newcommand\bdm{\mathbf m}
\newcommand\bdn{\mathbf n}
\newcommand\bdp{\mathbf p}
\newcommand\bdq{\mathbf q}
\newcommand\bdr{\mathbf r}
\newcommand\bdu{\mathbf u}
\newcommand\bdv{\mathbf v}
\newcommand\bdx{\mathbf x}
\newcommand\bdy{\mathbf y}
\newcommand\bdw{\mathbf w}
\newcommand\bdzero{\mathbf 0}

% - sets -
\def\Rone{{\mathbb R}}
\def\Cone{{\mathbb C}}
\def\Zone{{\mathbb Z}}

% - operators -

\def\norm#1{{\Vert #1\Vert}}
\def\SP#1{\langle #1\rangle}
\def\Ordo{\mathcal O}
\def\Fourier#1{\mathcal F\left\{#1\right\}}
\def\Laplace#1{\mathcal L\left\{#1\right\}}
\def\invFourier#1{\mathcal F^{-1}\left\{#1\right\}}
\def\invLaplace#1{\mathcal L^{-1}\left\{#1\right\}}
\let\invFF\invFourier
\let\FF\Fourier
\let\invLL\invLaplace
\let\LL\Laplace

% - förkortnngar

\newcommand\mha{med hjälp av }
\newcommand\bis{^{\prime\prime}}
\newcommand\triss{^{'''}}
\newcommand\PartInt[2]{\left\lceil\matrix{\mbox{\small #1}\cr
                                       \mbox{\small #2}}\right\rceil}
\def\tfrac{\textstyle\frac}

%adjust row height in tables #1 - overall height #2 - depth:
\newcommand\TblHeight[2]{\lower#2em\vbox to#1em{\hsize=0pt}}
\def\tvavektor[#1,#2]{\protect\pmatrix{#1\cr #2}}
\def\trevektor[#1,#2,#3]{\protect\pmatrix{#1\cr #2\cr #3}}
\let\ob\overline
\let\iff\Leftrightarrow


%% End \input\jobname.h fold >>> -


\begin{document}


\section*{Matematiska verktyg, formelblad} 
\hrule
\bigskip

\subsection*{De grundläggande logiska konnektiven och deras sanningsvärdestabeller}%<<<

\begin{tabular}[m]{|l*{3}{|c}|}%<<<
\cline{2-4}
\multicolumn{1}{c|}{} & \textbf{och} &\textbf{eller} & \textbf{inte}\\
\hline
  Engelska &   AND    &   OR     &   NOT  \\
\hline
  Logiksymbol      & $\wedge$ & $\vee$   & $\neg$ \\
\hline
  Mängdsymbol      & $\cap$   & $\cup$   & ${}^c$ \\
\hline
Boolesk symbol   & $\cdot$  &    $+$   & \raise5pt\hbox{$-$} \\
\hline
  C-program        & \&\& & $||$ & ! \\
\hline
 & & & \\[-8pt]
 \raise5pt\hbox{Gate (en)\,/\,grind (sv)}
  & \lower0pt\hbox{\includegraphics[scale=0.7]{Figs/gate_and}}
  & \lower0pt\hbox{\includegraphics[scale=0.7]{Figs/gate_or}} 
  & \lower0pt\hbox{\includegraphics[scale=0.7]{Figs/gate_not}}
     \\
\hline
\end{tabular}%>>>
\hfil
\begin{tabular}[m]{|*{7}{>{$}c<{$}|}}%<<<
  \hline
    p & q & \neg p & p \wedge q & p \vee q & p \to q & p \leftrightarrow q\\
  \hline
    0 & 0 &    1   &       0    &     0    &    1    &         1     \\
  \hline
    0 & 1 &    1   &       0    &     1    &    1    &         0     \\
  \hline
    1 & 0 &    0   &       0    &     1    &    0    &         0     \\
  \hline
    1 & 1 &    0   &       1    &     1    &    1    &         1     \\
  \hline
\end{tabular}%>>>
%>>>

\subsection*{Räkneregler i satslogiken och i booleska algebran}%<<<
\begin{tabular}[t]{|l|*{2}{>{$}c<{$}|}}
\cline{2-3}
  \multicolumn{1}{c|}{} & \textbf{Satslogik} &\textbf{Boolesk algebra} \\
\hline
1. \textbf{Dubbel negation} 
               &  \neg (\neg p) \iff p & \ob{\ob p} = p \\
\hline
2. \textbf{Idempotenslagar} 
                & p\vee p \iff p & p + p = p\\
                & p\wedge p \iff p & p^2 = p\\
\hline
3. \textbf{Inverslagar}     
                & p\vee\neg p \iff 1   & p + \ob p = 1\\
                & p\wedge\neg p \iff 0 & p\,\ob p = 0\\
\hline
4. \textbf{Dominanslagar}   
                & p \vee 1 \iff 1 & p + 1 = 1 \\
                & p\wedge0 \iff 0 & p \cdot 0  = 0 \\
\hline
5. \textbf{Identitetslagar} 
                & p \vee 0 \iff p  & p + 0 = p \\
                & p\wedge 1 \iff p & p\cdot 1 = p\\
\hline
6. \textbf{Kommutativa lagar} 
                  & p \vee q \iff q \vee p & p + q = q + p \\
                  & p \wedge q \iff q \wedge p & pq = qp\\
\hline
7. \textbf{Associativa lagar} 
                  & (p \vee q) \vee r \iff p \vee (q \vee r) 
                  & (p + q) + r = p + (q + r) \\
                  & (p \wedge q)\wedge r \iff p \wedge (q\wedge r) 
                  & (pq)r = p(qr)\\
\hline
8. \textbf{de Morgan lagarna} 
                  & \neg ( p \vee q ) \iff \neg p \wedge \neg q 
                  & \ob {p + q}  = \ob p\,\ob q \\
                  & \neg ( p \wedge q ) \iff \neg p \vee \neg q 
                  & \ob {pq}  = \ob p+\ob q \\
\hline
9. \textbf{Distributiva lagar}
                  & p \vee (q \wedge r ) \iff (p \vee q) \wedge (p \vee r) 
                  & p + qr = (p + q)(p + r) \\
                  & p \wedge (q\vee r)\iff (p \wedge q) \vee (p\wedge r) 
                  & p(q+r) = pq+pr\\
\hline
10. \textbf{Absorptionslagar}
                  & p \vee (p \wedge q ) \iff p 
                  & p + pq = p \\
                  & p \wedge (p\vee q)\iff p
                  & p(p+q) = p\\
\hline
\end{tabular}
%>>>

\subsection*{Omskrivning för konnektiven implikation och ekvivalens} %<<<

\begin{tabular}{|l|*{2}{>{$}c<{$}|}}
  \hline
  \textbf{Implikation} &    p\to q
                       &  \neg p \vee q \iff \neg q \to \neg p \\
  \hline
  \textbf{Ekvivalens} &  p\leftrightarrow q 
                      & 
                      (p\wedge q) \vee (\neg p\wedge \neg q) 
                      \iff 
                      (p\to q) \wedge ( q \to p ) 
                      \\
  \hline
\end{tabular}
%>>>

\end{document}
