% Funklära formelblad, startat 2013-08-19 as cherrypick från formelbald-an
% 2015-01-19:  C-tal borkommenterats, trig tabellerna kortade med symmetrirelationer till höger

\voffset-1.5cm
\documentclass{article}
\usepackage[utf8]{inputenc} % <-- IvtTj: Changed encoding to UTF-8
\usepackage{graphicx,a4wide,amsfonts,amsmath,fancyhdr}
\everymath{\displaystyle}   % <-- Ändring IvTj

\parindent=0pt

% Macros <<<
\addtolength{\topmargin}{-35mm}
\addtolength{\headheight}{25ex}
\addtolength{\headsep}{-3ex}
\addtolength{\textheight}{20mm}
\addtolength{\textwidth}{12mm}
%\footheight{ 0ex}
\footskip 3ex
\pagestyle{fancy}

\begin{document}

%% <-- 2009-01-21: ändring IvTj, nya headers/footers
%\lhead{\ifnum\thepage=1  \includegraphics[scale=1.3]{jth-logo.eps}\fi} % <-- Ändring IvTj
\chead{\ifnum\thepage=1 {} \else Formelblad Linjär Algebra \& Funktionslära\fi}
\rhead{\ifnum\thepage=1 \textbf{FORMELBLAD LINJÄR ALGEBRA \& FUNKTIONSLÄRA}
          %\parbox[b]{8.2em}{\textbf{FORMELBLAD}\\Matematisk analys}
               \else sid. \thepage{} av \pageref{LastPageNo}  \fi}

\lfoot{\ifnum\thepage=1\small\today\fi}
\cfoot{}
\rfoot{\ifnum\thepage=1\small File: \texttt{\jobname.pdf}\fi}
\parindent 0 cm
\normalsize
%% \begin{flushright}\ifnum\thepage=1 
%% %{\today} 
%% \fi\end{flushright}



\let\ent\Leftrightarrow
\def\trevektor[#1,#2,#3]{\begin{pmatrix}#1\cr #2\cr #3\end{pmatrix}}
\def\abs#1{|#1|}
\def\Rone{{\mathbb R}}
\def\bdzero{{\mathbf 0}}
\def\vec#1{\mathbf #1} %% <-- omdef \vec -- vektorer i boldface

\def\bdb{{\mathbf b}}
\def\bde{{\mathbf e}}
\def\bdf{{\mathbf f}}
\def\bdn{{\mathbf n}}
\def\bdr{{\mathbf r}}
\def\bdr{{\mathbf r}}
\def\bdu{{\mathbf u}}
\def\bdv{{\mathbf v}}
\def\bdw{{\mathbf w}}
\def\bdx{{\mathbf x}}

\let\iff\Leftrightarrow % <-- ändring IvTj
\let\ob\overline        % <-- ändring IvTj
\def\EndRow{\\*[3pt]}

%\psfig{figure=CTH-logo-4cm.eps%,width=19mm}
 %\psfig{figure=k-ing.eps,width=29mm}} <-- ändring IvTj
%>>>

\vspace{-1.5cm}

%%% \subsection*{Funktionslära:}

%% \subsubsection*{Några beteckningar, konjugat- och binomialformlerna}%<<<
%% 
%% 
%% \begin{tabular}[m]{|c|c|}
%%   \hline
%% Fakultet: $n! = \begin{cases}
%%                 \hfil1,& n=0\\
%%                 1\cdot2\cdot3\cdots\cdot n,& n=1,2,\dots
%%                 \end{cases}$
%% &
%% Binomiala koefficienter:
%%     $\binom nk = \frac{n!}{k!(n-k)!}$\\
%%   \hline
%% \end{tabular}
%% 
%% \medskip
%% 
%% \begin{tabular}[m]{|c|c|c|}
%%       \hline
%%       &&\\*[-11pt]
%%     $(a+b)(a-b) = a^2-b^2$
%%     &
%%     $(a \pm b)^2 = a^2 \pm 2ab+b^2$
%%     &
%%     $(a+b)^n = 
%%     \sum_{k=0}^n\binom nk a^{n-k}b^k$\\
%%       \hline
%% \end{tabular}
%% % 
%% % Pascals triangel:
%% % \begin{verbatim}
%% %       1 
%% %     1 2 1
%% %    1 3 3 1
%% %   1 4 6 4 1
%% % \end{verbatim}
%% %>>>

%% \subsubsection*{Komplexa tal} % Kommenterats bort 2015-01-19: %<<<
%%
%% $
%% \begin{array}{|l|l|}
%%   \hline & \\*[-8pt]
%%   z=x+iy    =|z|\,e^{ i\varphi}=|z|(\cos\varphi+i\sin\varphi) &
%%  x=\text{Re}(z)\in\Rone,\;
%%  y=\text{Im}(z)\in\Rone,\;
%%  |z| = |\ob z| = \sqrt{x^2+y^2}
%%   \EndRow
%%   \ob z=x-iy=|z|\,e^{-i\varphi}=|z|(\cos\varphi-i\sin\varphi) &
%%   \varphi = \arg (z),
%%   \;
%%   %\text{ där }
%%   \cos\varphi = x/|z|,
%%   %\text{ och }
%%   \;
%%   \sin\varphi = y/|z|  ;
%%   \;
%%   \arg(\ob z) = -\varphi
%%   \EndRow
%%   \ob{\ob z} = z ;
%%   \;\,
%%   \ob{zw} = \ob z\,\,\ob w ;
%%   \;\,
%%   \ob{z / w} = \ob z/\ob w ;
%%   \;\,
%%   z\ob z = |z|^2
%%   &
%%   |zw| = |z|\,|w| ;
%%   \quad
%%   |z/w| = |z|/|w| ;
%%   \quad
%%   |z+w| \le |z| + |w|
%%   \EndRow   \hline
%% \end{array}
%% $
%% %>>>

\subsubsection*{Kvadratkomplettering, faktorisering och nollställen till ett andragradspolynom:}%<<<
   \[
z^2+pz+q
  =\left(z+\frac{p}{2}\right)^2-\frac{p^2}{4}+q
  =(z-z_1)(z-z_2)
  ,\quad
  z_{1,2}=-\frac{p}{2}\pm\sqrt{\frac{p^2}{4}-q}.
\]

\vspace{-1em}
% Andragradsekvationer:
% % \subsubsection*{Andragradsekvationer}
% $x^2+px+q=(x-x_1)(x-x_2)=0\iff 
% x_{1,2}
% =\begin{cases}
%   -\frac{p}{2}\pm\sqrt{\frac{p^2}{4}-q},&\frac{p^2}{4}-q\geq 0
%   \\*[9pt]
%   -\frac{p}{2}\pm i\sqrt{q-\frac{p^2}{4}},&\frac{p^2}{4}-q< 0
% \end{cases}
% $
%>>>

\subsubsection*{Polynom och deras nollställen} %<<<
%% <-- änding IvTj 
Polynom av grad $n$:\, $P_n(z)
   =a_nz^n+a_{n-1}z^{n-1}+\dots+a_1z+a_0
   =\sum_{k=0}^n a_kz^k$, 
   $a_n\neq 0$.

Faktorsatsen: $P_n(z_0)=0 \iff P_n(z) = (z-z_0)Q_{n-1}(z)$, där $Q_{n-1}$ är
polynom av grad $n-1$.

%% \bigskip %  bortkommenterats 2015-01-19:
%%   Polynom med reella koefficienter: 
%%   Om $P_n(z_0)=0\iff P_n(\ob z_0)=0$, $\text{Im}(z)\neq0$, och därmed
%%   $$
%%   P_n(z)
%%    =(z-z_0)(z-\ob z_0)Q_{n-2}(z)
%%    =(z^2-2\text{Re}(z_0)z +|z_0|^2)Q_{n-2}(z).
%%   $$
%>>>

\subsubsection*{Potenser, exponentialfunktioner, logaritmer}%<<<


$\begin{array}{|l|l|l|l|l|}\hline
  &&&&\\*[-11pt]
  a^0=1 &a^xa^y=a^{x+y} &
  %\frac{a^x}{a^y}=a^{x-y}&(a^x)^y=a^{xy} <-- ändring IvTj
  a^x/a^y=a^{x-y}&(a^x)^y=a^{xy} 
  & (ab)^x=a^xb^x 
\EndRow
  \ln(1)=0 &\ln(xy)=\ln(x)+\ln(y) &\ln\Bigl(\frac{x}{y}\Bigr)
  =\ln(x)-\ln(y) & 
  \ln\Bigl(\frac1x\Bigr) = -\ln(x) &
  \ln(x^y)=y\ln(x) 
\EndRow 
  e^{\ln(x)}=x & \ln(e^x)=x & \log_a(x)
  =\frac{\ln(x)}{\ln(a)} & a^x=e^{x\ln(a)} & 
  (\log_ba)(\log_ab) = 1
\EndRow
\hline\end{array}$%>>>

\subsubsection*{De trigonometriska funktionerna}%<<<

% Standarda värden för sin/cos/tan + symmetrirelationer t höger %<<<
{% Local scope def's
\let\F\frac
\newcommand\CC[1]{#1^\circ}
\def\vPad{&&&&&\\*[-10pt]}
\def\myEnd{\\\hline\vPad}
\newcommand\tW{{\sqrt2}}
\newcommand\tH{{\sqrt3}}
$
\begin{array}{|c|c|c|c|c|c|}
\hline\vPad
\mbox{(Std.~vinklar)}^\circ
  & \CC{0} &\CC{30} &\CC{45} &\CC{60} &\CC{90}   % &\CC{120} &\CC{135} &\CC{150} &\CC{180}
 \myEnd
 x \mbox{ (\textbf{rad})}
        & 0     & \pi/6 & \pi/4 &  \pi/3 & \pi/2 % & 2\pi/3 & 3\pi/4 & 5\pi/6 &\pi
 \\\hline\hline\vPad
 \cos(x)&   1   & \tH/2 & 1/\tW &  1/2  &   0    % & -1/2   & -1/\tW & -\tH/2 & -1
 \myEnd
 \sin(x)&   0   &   1/2 & 1/\tW & \tH/2 &   1    % & \tH/2  &  1/\tW &   1/2  &  0
 \myEnd
 \tan(x)&   0   & 1/\tH &   1   & \tH   &
                                    \mbox{ej def}
                                                 % &-\tH  &    -1  & -1/\tH &  0
 \\\hline
\end{array}
\quad
\begin{array}[m]{|l|l|}
\hline
  \cos(-x) = \cos(x)    & \cos(x\pm\pi) = -\cos(x) \\
  \sin(-x) = -\sin(x)   & \sin(x\pm\pi) = -\sin(x) \\
\hline
\hline
  \sin(\pi-x) = \sin(x) &  \cos(x+2\pi) = \cos(x) \\
  \cos(\pi-x) = -\cos(x)&  \sin(x+2\pi) = \sin(x) \\
\hline
\hline
 \tan(-x) = \tan(x)     & \tan(x\pm\pi) = \tan(x) \\
\hline
\end{array}
$
}%>>>

% Trig.~ekv: %<<<
\medskip
\textbf{Trig.~ekv}:
\raise8pt\hbox{
$
\begin{array}[t]{|c|c|c|}
  \hline
%  \multicolumn{3}{|c|}{\mbox{Trigonometriska ekvationer}}\\ % <-- Ändring IvTj
%  \cline{1-3}
  \sin(x)=c\,, \;|c|\leq 1,      \mbox{ löses av:} &
  \cos(x)=c\,, \;|c|\leq 1,      \mbox{ löses av:} &
  \tan(x)=c\,, \;c\in\mathbb{R}, \mbox{ löses av:}
\\[2pt]
x=
\begin{cases}
     x_0+2n\pi,   & n\in\mathbb{Z}\cr
    \pi-x_0+2k\pi,& k\in\mathbb{Z}
\end{cases}
&
x=\pm x_0+n2\pi\,,\;n\in\mathbb{Z}
&x=x_0+n\pi\,,\;n\in\mathbb{Z}
\\[11pt]
\hline
\end{array}
$} % raised hbox
%>>>

% Trigrormlerna%<<<
\medskip
$\begin{array}{|l|l|}\hline &\\*[-11pt] % <-- ändring IvTj
 %\sin(-x)=-\sin(x) &\cos(-x)=\cos(x) \EndRow
 \cos^2(x)+\sin^2(x)=1  &
 \cos(2x)=\cos^2(x)-\sin^2(x) \EndRow %% <-- ändring IvTj
 \sin(2x)=2\sin(x)\cos(x)&\cos(2x)=2\cos^2(x)-1=1-2\sin^2(x) \EndRow
 \sin^2(x) = (1-\cos(2x))/2 &
 \cos^2(x) = (1+\cos(2x))/2 \EndRow % <-- ändring IvTj
 \sin(x)=\sin(\pi-x)=\cos(\frac{\pi}{2}-x) &
 \cos(x)=\sin(\frac{\pi}{2}-x)=-\cos(\pi-x)\EndRow
 % \sin(x\pm\pi) = -\sin(x) & \cos(x\pm\pi) = -\cos(x) \EndRow % <-- ändring IvTj
 \sin(x\pm y)=\sin(x)\cos(y)\pm\sin(y)\cos(x) &
 \cos(x\pm y)=\cos(x)\cos(y)\mp\sin(x)\sin(y)\EndRow
% \sin(2x)=2\sin(x)\cos(x)&\cos(2x)=2\cos^2(x)-1=1-2\sin^2 x \EndRow
 \sin(x)\sin(y)=\frac{1}{2}(\cos(x-y)-\cos(x+y)) &
 \sin(x)\cos(y)=\frac{1}{2}(\sin(x-y)+\sin(x+y)) \EndRow
 \cos(x)\cos(y)=\frac{1}{2}(\cos(x-y)+\cos(x+y)) &
 \tan(x\pm y)=\dfrac{\tan(x)\pm\tan(y)}{1\mp\tan(x)\tan(y)}\\*[8pt]
 \cline{1-2}\multicolumn{2}{|c|}{     % <-- ändring IvTj
   a\cos x+b\sin x=A\sin(x+\varphi) = A\cos(x-\psi),\;
              A=\sqrt{a^2+b^2}
     \mbox{ \ och \ }
              \left\{ \begin{array}{l}
              \sin\varphi=\cos\psi=a/A \\
              \cos\varphi=\sin\psi=b/A
\end{array}\right.}
 \EndRow\hline\end{array}$ %>>>

%>>>

\subsubsection*{De inversa trigonometriska funktionerna}%<<<

$\begin{array}{|lll|}\hline && \\*[-9pt] 
  y=\sin(x),\hspace{2 mm} x\in[-\frac{\pi}{2},\frac{\pi}{2}]&
  \iff &
  x=\arcsin(y),\hspace{2 mm} y\in[-1,1] \EndRow 
  y=\cos(x),\hspace{2 mm} x\in[0,\pi]&
  \iff &
  x=\arccos(y),\hspace{2 mm} y\in[-1,1] \EndRow 
  y=\tan(x),\hspace{2 mm} x\in(-\frac{\pi}{2},\frac{\pi}{2})&
  \iff &
  x=\arctan(y), \hspace{2 mm} y\in\mathbb{R}\\*[6pt]\hline
\end{array}$  

\vspace{1 mm}

$\begin{array}{|l|l|l|}\hline & & \\*[-11pt]
\arcsin(-x)=-\arcsin(x) & \arccos(-x)=\pi-\arccos(x) & 
  \arctan(x)+\arctan(\tfrac1x)=\frac\pi2, \; x>0 \EndRow % <-- ändring IvTj
\arctan(-x)=-\arctan(x) &\arccos(x)=\frac{\pi}{2}-\arcsin(x) & 
  \arctan(x)+\arctan(\tfrac1x)=-\frac\pi2, \; x < 0 \\*[6pt] % <-- ändring IvTj
\hline
\end{array}$%>>>

\vspace{2mm}\hrule

%%% \subsection*{Linjär algebra}

\subsubsection*{Vektorer}%<<<

Formlerna i detta avsnitt är för vektorer i $\Rone^3$. Med
undantag för formlerna för vektorprodukt samt de formler som rör
planet, gäller de med självklara modifieringar även i $\Rone^n$.
Nedan är vektorerna $\vec u=(u_1,u_2,u_3)^t$ och $\vec
v=(v_1,v_2,v_3)^t$ givna i en ON-bas, $\theta$ är vinkeln mellan
$\vec u$ och $\vec v$ samt $\lambda\in\mathbb R$. Då gäller:
%% är ett godtyckligt reellt tal. Då gäller:

% Tabell sum, sträckning, längd%<<<
\medskip
\begin{tabular}{|l|l|}
  \hline  &  \\[-8pt]
  \textbf{Addition: } 
  $\bdu+\bdv=\trevektor[u_1+v_1,u_2+v_2,u_3+v_3]=\bdv + \bdu$.
  &
  \textbf{Mult.~med tal: }
  $\lambda\bdu=\trevektor[\lambda u_1,\lambda u_2,\lambda u_3] = \bdu\lambda$.
  \\[15pt] \hline \multicolumn{2}{|c|}{} \\[-9pt]
  \multicolumn{2}{|l|}{\textbf{Längd, norm}: \;
  $\abs{\bdu}
    =\sqrt{u_1^2+u_2^2+u_3^2} 
  = \sqrt{\bdu\bullet\bdu}$\,; 
    \; $\abs{\lambda\bdu}=\abs \lambda\abs{\bdu}$.
  } %multicolumn
  \\[8pt] \hline \multicolumn{2}{|c|}{} \\[-9pt]
  \multicolumn{2}{|l|}{Enhetsvektorn \/$\bde$\/ 
  med sammma pekrikitning som en given vektor $\bdu$:
  $
  \; \bde = \frac{\bdu}{|\bdu|}\cdot
  $
  } %multicolumn
  \\[8pt] \hline
\end{tabular}%>>>

\smallskip

% Tabell Skalärprodukt, RR, Vektorprodukt %<<<
\begin{tabular}{|p{0.2\linewidth}|p{0.745\linewidth}|}
  \hline &\\*[-8pt]
  \textbf{Skalärprodukt}:
  &
  $\bdu\bullet\bdv=\abs{\bdu}\abs{\bdv}\cos\theta = u_1v_1+u_2v_2+u_3v_3$
  \\ &\\*[-8pt]
  \textbf{Räkneregler}:
  &
  $\bdu\bullet\bdv=\bdv\bullet \bdu$, 
  $(\lambda\bdu)\bullet\bdv=\bdu\bullet(\lambda\bdv)=\lambda(\bdu\bullet\bdv)$
  \\*[2pt]
  &$\bdu\bullet(\bdv+\bdw)=\bdu\bullet\bdv+\bdu\bullet\bdw$
  \\*[2pt]
  &$\bdu\bullet\bdu=\abs{\bdu}^2$;
  $|\bdu|=\sqrt{\abs{\bdu}^2}=\sqrt{u_1^2+u_2^2+u_3^2}\ge0$ 
  (med $|\bdu|=0\iff\bdu=\mathbf 0$)
  \\*[2pt]
  &$\bdu$ och $\bdv$ ortogonala $\ent\bdu\bullet\bdv=0$ eller är någon av vektorerna nollvektorn.
\\*[2pt]\hline&\\*[-8pt]
  \textbf{Proj.~formeln}:
  &
  Den vinkelräta projektionen $\bdu'$ av
  $\bdu$ på $\bdv$ är:
  $\bdu'=\dfrac{\bdu\bullet\bdv}{\abs{\bdv}^2}\bdv = (\bdu\bullet\bde)\,\bde$, 
  $\bde=\frac{\bdv}{|\bdv|}\cdot$
  \\*[8pt] \hline
  \textbf{Kryssprodukt}:
  &  %% Tilläg IvTj, 2009-02-13
  $\bdu\times\bdv=\trevektor[u_1,u_2,u_3]\times\trevektor[v_1,v_2,v_3]
  =
  \left(\begin{array}{r}
      \begin{vmatrix}
        u_2 & v_2\\
        u_3 & v_3\\
      \end{vmatrix} 
   \\*[8pt] 
  -   \begin{vmatrix}
        u_1 & v_1\\
        u_3 & v_3\\
      \end{vmatrix} 
    \\*[8pt]
      \begin{vmatrix}
        u_1 & v_1\\
        u_2 & v_2\\
      \end{vmatrix} 
    \end{array}\right) 
  =\trevektor[u_2v_3-u_3v_2,u_3v_1-u_1v_3,u_1v_2-u_2v_1]$.
  \\*[8pt] 
  \textbf{Räkneregler}:
  &
  $\bdu\times\bdv$ är ortogonal mot både $\bdu$ och $\bdv$.
  \\*[2pt]
  &$\abs{\bdu\times\bdv}=\abs{\bdu}\abs{\bdv}\sin\theta$ 
  ger
  arean av parallellogramen som $\bdu$ och $\bdv$ spänner.
  \\*[2pt]
  &$\bdu\times\bdv=-\bdv\times \bdu$, 
  $(\lambda \bdu)\times\bdv=\bdu\times(\lambda \bdv)=\lambda (\bdu\times\bdv)$, 
  $\bdu\times(\bdv+\bdw)=\bdu\times\bdv+\bdu\times\bdw$.
  \\*[2pt] \hline & \\*[-10pt]
  \textbf{Trippelprodukt}:
  &
  $
  \bdu\bullet(\bdv\times\bdw)=
  \begin{vmatrix}
    u_1&v_1&w_1\\
    u_2&v_2&w_2\\
    u_3&v_3&w_3
  \end{vmatrix}
  = \bdw\bullet(\bdu\times\bdv)
  = \bdv\bullet(\bdw\times\bdu)
  = \pm $ volymen av den parallellepiped som spänns upp av vektorerna
  $\bdu$, $\bdv$ och $\bdw$.\\*[0pt]
  \hline
  &
  \\*[-8pt]
  \textbf{Lin.~oberoende}: 
  & 
  Vektorerna $\bdu_1,\dots\bdu_n$ är linjärt 
  oberonde då och endast då det homogena ekvationssysemet
  $\lambda_1\bdu_1+\dots\lambda_n\bdu_n=\bdzero$
  har endast den triviala lösningen
  $\lambda_1=\lambda_2=\cdots\lambda_n=0$.
  \\*[2pt]
  \hline
  &\\*[-8pt]
  \textbf{Lin.~beroende}: 
  &
  Motsatsen till lin. oberoende: 
  Vektorerna $\bdu_1,\dots\bdu_n$ är
  linjärt beroende $\ent$ Det finns skalärer $\lambda_1,\dots\lambda_n$,
  ej alla 0 sådana att $\lambda_1\bdu_1+\dots\lambda_n\bdu_n=\vec
  0$.
  \\*[2pt] \hline &\\*[-8pt]
  \textbf{Räta linjer}: 
  &
  En rät linje genom punkten $P_0=(x_0,y_0,z_0)$ med
  riktningsvektorn\\
  &$\bdv=\trevektor[v_x,v_y,v_z]$ har på parameterform ekvationen 
  $\trevektor[x,y,z]=\trevektor[x_0,y_0,z_0]+t\trevektor[v_x,v_y,v_z],\,
  t\in\mathbb R$.
  \\*[14pt] \hline
\end{tabular}%>>>

% Trippelprodukt,lin (o)ber, linjens + planets ekv. %<<<
\begin{tabular}{|p{0.2\linewidth}|p{0.745\linewidth}|}
  \hline & \\*[-8pt]
  \textbf{Plan}:
  &
  Ekvation för planet genom punkt $(x_0,y_0,z_0)$  vinkelrätt mot
  $\mathbf n=(a,b,c)^t$:
  \[
  (\mathbf{r} - \mathbf{r}_0)\cdot\mathbf{n} = 0
  \iff
  \left[
     \trevektor[x,y,z]
    -\trevektor[x_0,y_0,z_0]
  \right]
  \cdot
  \trevektor[a,b,c]=0
  \iff
  \begin{cases}
  ax+by+cz=d,\cr
  \hspace{3em}  d=\bdn\bullet\bdr_0.
  \end{cases}
  \]
  \\*[-12pt] %% &\\*[-8pt]
  På parameterform:
  &Plan genom punkten $P$ med ortsvektor $\bdr_0$ utmed vektorerna $\bdu$ 
  och $\bdv$:
  \\
  &
  \hfil
  $
  \bdr = \bdr_0 + s\bdv + t\bdw, \;s,t\in\Rone.
  $
  \\
  \hline
\end{tabular}%>>>
%>>>

\subsubsection*{Linjära ekvationssystem:}%<<<
\begin{tabular}{|p{0.2\linewidth}|p{0.745\linewidth}|}
  \hline
  \textbf{Gausseliminering}:
  &
  Syfte: Att med hjälp av radoperationer ställa systemet på
  trappstegsform.
  \\
  &
  Tillåtna radoperationer, som inte ändrar på systemets lösningsmängd, är:
  \vspace{-7pt}
  \begin{itemize}
      \addtolength{\itemsep}{-7pt}
  \item Platsbyte på ekvationer;
  \item Multiplikation av en ekvation med ett tal;
  \item Addition av en multipel av en ekvation till en annan.
  \end{itemize}
  \\*[-13pt]\hline & \\*[-9pt]
  $A\sim I$:
  &
  Om en kvadratisk martis $A$ kan med radoperationer formas om
  till $I$, kallas den \textbf{radekvivalent med enhetsmatrisen} (betecknas: $A\sim I$).
  Lösningen av en matrisekvation $AX=B$ resulterar då i: 
  $
  (A|B)\sim\cdots\sim (I|X).
  $
  \\\hline
\end{tabular}%>>>

\subsubsection*{Matriser}%<<<

\begin{tabular}{|p{0.2\linewidth}|p{0.745\linewidth}|}
  \hline & \\*[-9pt]
  \textbf{Addition}:
  & 
  $A+B=B+A$; \; $(A+B)+C=A+(B+C)$. 
  \\*[2pt]\hline & \\*[-9pt]
  \textbf{Mult. med tal}:
  &
  $\lambda A = A\lambda = (\lambda a_{ij})$ (termvis mult.~av alla termer); \;
  $\lambda(A+B)=\lambda A+\lambda B$.
  \\*[2pt]\hline & \\*[-9pt]
  \textbf{Multiplikation}:
  &
  $AB\ne BA$ (i allmänhet);
  \\*[2pt]
  &$(AB)C=A(BC)$, \ $A(B+C)=AB+AC$, \  $(A+B)C=AC+BC$.
  \\*[2pt]\hline & \\*[-9pt]
  \textbf{Transponering}: &
  Rader och kolonner byter plats, t\,ex, 
  $ \begin{pmatrix}
    a_{11}&a_{12}&a_{13}\\
    a_{21}&a_{22}&a_{23}
  \end{pmatrix}^t
  =
  \begin{pmatrix}
    a_{11}&a_{21}\\a_{12}&a_{22}\\
    a_{13}&a_{23}
  \end{pmatrix}$;
  \\*[16pt]
  &$(A+B)^t=A^t+B^t$, \ 
  $(AB)^t=B^tA^t$ (Obs.~ordningen!), \ 
  $(A^{-1})^t=(A^t)^{-1}$.
  \\*[2pt]\hline  & \\*[-7pt]
  \textbf{Enhetsmatrisen} $I$:
  &
  $I=\mbox{diag}[1,\dots,1]
  $ (diagonalmatris med 1 på huvuddiagonalen), $AI=IA=A$.
%%  \begin{pmatrix}
%%    1&0&\cdots&0\\
%%    0&1&\cdots&0\\
%%    \vdots&\vdots&\ddots&\vdots\\
%%    0&0&\cdots&1
%%  \end{pmatrix}
  \\*[2pt]\hline & \\*[-9pt]
\textbf{Matrisinvers}:
 &
  $AA^{-1}=A^{-1}A=I$, $(AB)^{-1}=B^{-1}A^{-1}$.\\*[3pt]
  Jacobis metod:
  &
  Med radoperationer överförs systemet $(A|I)\sim \cdots\sim (I|A^{-1})$.
  \\[3pt]
  Ortogonala matriser: & $A^{-1}=A^t$.
  \\
  $2\times 2$--matriser:&
  $A_{2\times2}^{-1}
  =\begin{pmatrix}
    a&b\\
    c&d
  \end{pmatrix}^{-1}
  =\dfrac1{\det(A)}
  \begin{pmatrix}
    d&-b\\
    -c&a
  \end{pmatrix}
  =\dfrac1{ad-bc}
  \begin{pmatrix}
    d&-b\\
    -c&a
  \end{pmatrix}$.
  \\*[9pt]\hline\end{tabular}%>>>

\subsubsection*{Determinanter}%<<<
\begin{tabular}{|p{0.2\linewidth}|p{0.745\linewidth}|}
  \hline
   & \\*[-8pt]
   $2\times2$--\textbf{matriser}: &Om $A=\begin{pmatrix}a&b\\c&d\end{pmatrix}$ är 
  $\det(A)=\begin{vmatrix}a&b\\c&d\end{vmatrix}=ad-bc$.
  \\*[9pt] \hline & \\*[-9pt]
  \textbf{Utv.~efter rad} $i$: &
 Om $D_{ij}$ är det.~som fås 
 genom att i $\det(A)$ stryka rad $i$ och kolonn $j$, så är:
  $$
  \det(A_{n\times n})
  = a_{i1}(-1)^{i+1}D_{i1}
  + \cdots + a_{ij}(-1)^{i+j}D_{ij}
  + \cdots + a_{in}(-1)^{i+n}D_{in}.
 $$ 
  \\*[-20pt] \hline & \\*[-8pt]
  \textbf{Räkneregler}: &
  $\det(A^t)=\det(A)$ (allt som är tillåtet radvis fungerar även kolonnvis);
  \\*[2pt]
  &
  $\det(AB)=\det(A)\det(B)$, och i synnerhet, 
  $\det(A^{-1})=1/\det(A)$;
  \\*[2pt]
  &
  Bryt ut tal:
  \ $
  \begin{vmatrix}
    a&\lambda b&c\\d&\lambda e&f\\g&\lambda h&i
  \end{vmatrix}
  =\lambda 
  \begin{vmatrix}
    a&b&c\\d&e&f\\g&h&i
  \end{vmatrix},\quad
  \begin{vmatrix}
    a&b&c\\d&e&f\\\lambda g&\lambda h&\lambda i
  \end{vmatrix}
  =\lambda 
  \begin{vmatrix}
    a&b&c\\d&e&f\\g&h&i
  \end{vmatrix};
  $\\*[15pt]
  &Om två rader eller två kolonner byter plats skiftar determinanten
  tecken.
  \\*[2pt]
  &
  En multipel av en rad kan läggas till en annan rad utan att determinanten ändras.
  Fungerar även kolonnvis, eftersom $\det(A)=\det(A^t)$.
\\
  \hline
\end{tabular}%>>>

\subsection*{Linjära avbildningar}%<<<

\begin{tabular}{|p{0.2\linewidth}|p{0.745\linewidth}|}
  \hline & \\*[-9pt]
  \textbf{Definition}
  &
  $F$ är en linjär avbildning $\ent F(\bdu+\vec
  v)=F(\bdu)+F(\bdv)$ och $F(\lambda\bdu)=\lambda F(\bdu)$.\\
  &$F$ är en linjär avbildning $\ent F(\bdu)=A\bdu$ där $A$ är en
  konstant matris.
  \\*[2pt] \hline & \\*[-9pt]
  \textbf{Sammansättning}:
  &
  Om $F(\bdu)=A\bdu$ och
  $G(\bdu)=B\bdu$ är
  $G\bigl(F(\bdu)\bigr)
  =G\bigl(A\bdu\bigr)
  =BA\bdu
  $.
  \\*[2pt] \hline & \\*[-9pt]
  \textbf{Isometriska}
  & Låt $F(\bdu)=A\bdu$. De följande poståenden är ekvivalenta: \{$F$ är isometrisk\}\\
  \textbf{avbildningar}:
  &
   ${}\ent\{\abs{F(\bdu)}=\abs{\bdu}$ för alla $\bdu\in\Rone^n$\}
   ${}\ent \{A$ är en ON-matris\}
  \\
  &
  ${}\ent\{\bdu\cdot\bdv=F(\bdu)\cdot F(\bdv)$ för alla $\bdu$ och $\bdv\in\Rone^n\}$
  \\ 
  &
  ${}\ent\{A^{-1}=A^t\ent A$:s kolonner (rader) utgör en ON-bas för $\Rone^n$\}.\\
  \hline
\end{tabular}%>>>

\subsection*{Baser och basbyten}%<<<
\begin{tabular}{|p{0.2\linewidth}|p{0.745\linewidth}|}
  \hline & \\*[-9pt]
  \textbf{Baser}:
  &
  En uppsättning av $n$ linjärt oberoende vektorer $\bde_1,\dots,\vec
  e_n$ bildar en bas för $\Rone^n$. Till varje $\bdu\in\Rone^n$ finns
  då reella tal $x_1,\dots,x_n$ så att $\bdu=x_1\bde_1\dots x_n\vec
  e_n$. Talen $x_1,\dots,x_n$ är $\bdu$:s koordinater i basen 
  $\bde_1,\dots,\bde_n$.
  \\*[2pt] \hline & \\*[-9pt]
  \textbf{Basbyten}:
  &
  Om vektorn $\bdu=\underline{\bde}X_\bde = \underline{\bdf}Y_\bdf$ har koordinaterna 
  $X_\bde=(x_1,\dots,x_n)^t$ i basen 
  $\underline{\bde}=(\bde_1,\dots,\bde_n)$
  och koordinaterna 
  $Y_{\bdf}=(y_1,\dots,y_n)^t$ i basen 
  $\underline{\bdf}=(\bdf_1,\dots,\bdf_n)$, 
  då är 
  $
  X_\bde = P Y_\bdf
  $,
  där $P$ består kolonnvis av de nya basvektorerna 
  $(\bdf_1,\dots,\bdf_n)$.
 % $\trevektor[x_1,\vdots,x_n]
 %          =P\trevektor[y_1,\vdots,y_n]$ där
 % basbytesmatrisen $P=
 % \begin{pmatrix}
 %   |&&|\\
 %   \bdf_1&\dots&\bdf_n\\
 %   |&&|
 % \end{pmatrix}$ med $\bdf_1,\dots,\bdf_n$ uttryckta i basen
 % $\bde_1,\dots,\bde_n$.
  \\*[2pt] \hline & \\*[-9pt]
  Avbildningsmatriser 
  &
  Om den linjära avbildningen $F$ har
  avbildningsmatris $A_\bde$ i basen $\underline{\bde}$ och
  \\
  i olika baser:
  &
  avbildningsmatris $A_\bdf$ i basen $\underline{\bdf}$, då gäller:
  $
  A_\bdf=P^{-1}A_eP\,\Leftrightarrow\, A_\bde = PA_\bdf P^{-1}
  $.
  \\
  \hline
\end{tabular}%>>>

\subsection*{Egenvärden och egenvektorer}%<<<
\label{subsection*}
\begin{tabular}{|p{0.2\linewidth}|p{0.745\linewidth}|}
  \hline & \\*[-9pt]
  \textbf{Definition:}
  &
  Matrisen $A$ har egenvektorn $\bdu\neq\vec0$ med
  egenvärdet $\lambda$ om
  $A\bdu=\lambda\bdu$.
  \\*[2pt] %\hline & \\*[-9pt]
  Egenvärdena $\lambda_k$ 
   & bestäms som lösningarna till den karakteristiska ekvationen 
   $\det(A-\lambda I)=0$.\\
  Egenvektorerna $\bdu_k$& bestäms ur det homogena systemet $(A-\lambda I)\bdu=\mathbf{0}$.
  \\*[2pt] \hline & \\*[-9pt]
  \textbf{Diagonalisering}:
&
Om $n\times n$-matrisen $A$ har $n$ st linjärt oberoende
egenvektorer $\bdu_1,\dots,\bdu_n$ med motsvarande egenvärden
$\lambda_1,\dots,\lambda_n$, är $A=PDP^{-1}$ med
\vspace{-10pt}
  $$ P=
  \begin{pmatrix}
    |&&|\\
    \bdu_1&\dots&\bdu_n\\
    |&&|
  \end{pmatrix}
  \mbox{ \ och \ \ }
  D = \begin{pmatrix}
    \lambda_1&\dots&0\\
    \vdots&\ddots&\vdots\\
    0&\dots&\lambda_n
  \end{pmatrix}.
  $$
  \vspace{-12pt}
  \\
  \hline
\end{tabular}%>>>

\subsection*{Huvudsatsen (vänster spalt) och dess negation (höger spalt)}%<<<

%% Låt $A$ vara en $n\times n$--matris.\\*[10pt]
\begin{tabular}{|p{0.473\linewidth}|p{0.473\linewidth}|}
  \hline
  &\\*[-10pt]
%  Låt $A$ vara en $n\times n$--matris.\hfill\break
  Ekvivalenta påståenden för en $n\times n$--matris $A$:
& % Låt $A$ vara en $n\times n$--matris.\hfill\break
  %% Följande påståenden är ekvivalenta:
  Ekvivalenta påståenden för en $n\times n$--matris $A$:
  \\
  \hline
  \\*[-24pt]
  \begin{itemize}\addtolength{\itemsep}{-5pt}
  \item $\det(A)\ne0$
  \item $A$ är inverterbar
  \item $A\sim I$ (dvs $A$ är radekvivalent med $I$)
  \item Ekvationssystemet $A\bdx=\bdzero$ har endast den triviala
    lösningen $\bdx=\bdzero$
  \item Ekvationssystemet $A\bdx=\bdb$ har en entydig lösning för
    varje val av högerledet $\bdb$
  \item $A$:s rader är linjärt oberoende
  \item $A$:s rader utgör en bas för $\mathbb{R}^n$
  \item $A$:s kolonner är linjärt oberoende
  \item $A$:s kolonner utgör en bas för $\mathbb{R}^n$
  \item Avbildningen $F(\bdx)=A\bdx$ är omvändbar % injektiv, surjektiv och inverterbar.
  \item $A$:s egenvärden är alla skilda från $0$
  \end{itemize}
  \vspace{-16pt}
  &
  \begin{itemize}\addtolength{\itemsep}{-5pt}
  \item $\det(A)=0$
  \item $A$ är inte inverterbar
  \item $A$ är inte radekvivalent med $I$
  \item Ekvationssystemet $A\bdx=\bdzero$ har oändligt många lösningar
  \item Ekvationssystemet $A\bdx=\bdb$ saknar lösning eller också
    har det oändligt många lösningar
  \item $A$:s rader är linjärt beroende
  \item $A$:s rader utgör inte bas för $\mathbb{R}^n$
  \item $A$:s kolonner är linjärt beroende
  \item $A$:s kolonner utgör inte bas för $\mathbb{R}^n$
  \item Avbildningen $F(\bdx)=A\bdx$ är inte omvändbar % varken injektiv, surjektiv eller inverterbar.
  \item $0$ är ett egenvärde till $A$
  \end{itemize}
  \vspace{-16pt}
  \\
  \hline
\end{tabular}%>>>

\label{LastPageNo}
\end{document}


% vim: foldmethod=marker
