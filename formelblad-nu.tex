\documentclass[a4paper]{article}

\usepackage[utf8]{inputenc}\usepackage{mathtools,graphicx,amssymb,array}

% Custom page dimensions + header/footer defs <<<
\advance \topmargin by -4.5\headheight
\advance \textheight by 125pt
\oddsidemargin -10pt
\evensidemargin \oddsidemargin
\marginparwidth 0.5in
\textwidth 6.6in
\parindent=0pt
\advance\parskip by 1pt

\pagestyle{headings}

\def\Nu{\textsc{Formula Sheet\hfill Numerical Methods}}
\def\Opt{\textsc{Formula Sheet\hfill Optimization}}
\def\NuOpt{\textsc{Formula Sheet\hfill Numerical Methods / Optimization}}

\makeatletter
\renewcommand{\@oddhead}{%
\ifnum\thepage=1
  \Nu\hfill
\else
  \ifnum\thepage=4
    \NuOpt
  \else
    \ifnum\thepage>4
      \Opt
    \else
      \Nu
    \fi
  \fi
  \hfill\thepage{}
\fi
}
\renewcommand{\@evenhead}{}
\renewcommand{\@oddfoot}
{\ifnum\thepage=1
  \today\hfill file:~\small\texttt{\jobname.pdf}
\else
\fi}

\renewcommand{\@evenfoot}{\small \texttt{\jobname.pdf},\hfill}
\makeatother
%>>>

\input ../bz/StdLaTeXdef.h \everymath{\displaystyle}

\begin{document}

\subsubsection*{General Calculus} % <<<

  \begin{itemize}

  \item Falling factorial: $x^{\scriptsize\b k}=x(x-1)\cdots(x-k+1)$. Thus,
    the binomial coefficient:
    $
      \binom{n}{k}
      = \frac{n^{\scriptsize\b k}}{k!}
      = \frac{n!}{k!(n-k)!}
      \cdot
    $

  \item The $k$:th falling power of a $x$ at step $h$:
    $
     x^{\scriptsize\b kh} =
     x(x-h)(x-2h)\cdots (x-(k-1)h)
    $.

  \item Mean value theorem for a $C^{k+1}$ $f\!:\Rone\to\Rone$:
    $
    f(a+h)
    =\sum_{i=0}^n\frac{f^{(j)}(a)}{j!}h^j
             + f^{(n+1)}(\xi),
             \; \xi=a+th,
             \; 0<t<1
    $.

  \item Taylor poly of order $k$ of a $C^{k+1}$-function $f\!:\!\Rone^n\to\Rone$:
    $f(\bdx+\bdh)=\sum_{j=0}^k
    (\partial_1+\cdots+\partial_n)^j f(\bda)+\Ordo(\Norm{\bdh}^{k+1})$.

  \item Linearization of a $C^2$-vector field $\bdf\!:\Rone^n\to\Rone^n$
     around a point $\bdx$:
   \[
     \bdf(\bdx+\bdh)
     =
     f(\bdx)
     +J_{\bdf}(x)\bdh + \Ordo(\Norm{\bdh}^2),
     \quad
     J_{\bdf}(x)
     \defeq\frac{d\bdf(x)}{d\bdx}
     =
     \begin{bmatrix}
       - & \nabla f_1(\bdx) & - \\
         & \vdots & \\
       - & \nabla f_n(\bdx) & - \\
     \end{bmatrix}
     =
     \begin{bmatrix}
       (f_1)'_{x_1} &\cdots&(f_1)'_{x_n} \\
       \vdots & \ddots & \vdots \\
       (f_n)'_{x_1} &\cdots&(f_n)'_{x_n} \\
     \end{bmatrix}
      \]
\end{itemize}

% >>>

\subsubsection*{General Linear Algebra}% <<<

\begin{itemize}

  \item Norm of a $n\times n$-matrix $A$ as a linear map $\Rone^n\to\Rone^n$:
    $\Norm{A}
    =\max_{\Norm{\bdx}=1} \Norm{A\bdx}
    =\max_{\bdx\neq\bdzero} \frac{\Norm{A\bdx}}{\Norm{\bdx}}\cdot
    $

  \item If $A=A^T$ is a symmetric $n\times n$ matrix,
    then all its eigenvalues are real.
    If
    $\lambda_i\neq \lambda_j$,
    $A\bdv_i=\lambda_i\bdv_i$
    and
    $A\bdv_j=\lambda_j\bdv_j$
    then
    $\bdv_i\perp\bdv_j$.
  \item
    The Spectral theorem: The matrix $A=A^T$ is symmetric
    if and only if is orthogonally diagonalizable:
    $A=VDV^T$,
    where
    $D=\diag[\lambda_1,\cdots,\lambda_n]$
    and
    $VV^T=I$.

  \item
    $A\succeq0$ is
    \textbf{positive definite}, if
    $\Angled{\bdx,A\bdx}=\bdx^T\!A\bdx\ge0$, $\forall\bdx\in\Rone^n$.
    $A\preceq0$ is negative definite if $-A\succeq0$.

  \item $A$ is \textbf{spd} (symmetric positive definite) if $A=A^T\succeq0$.
    $A_{n\times n}\succ 0$ if and only if \/ $\frac{A+A^T}{2}$ \/ is \textbf{spd}.

  \item (Sylvester criterion): $A\succeq 0$ if \textbf{all} its major minors are non-negative:
    $\det A[1:i,1:i]\ge 0$, $i=1,2,\ldots,n$.

\end{itemize}% >>>

\subsubsection*{Roots of functions of one variable}% <<<

\begin{itemize}
  \item Let $\lim_{k\to\infty}a_k\to a$
    and $\epsilon_k=|a-a_k|$.
    Covergence is of order $\alpha$ if
    $
    \epsilon_{k+1}\le M\epsilon^\alpha,
    $
    $k>N$.
    If
    $
    \Bigl\{\mkern-6mu
    \begin{array}{l}
      \alpha=1\!:~\text{lin.~conv.}, \\[-2pt]
      \alpha=2\!:~\text{quadratic}. \\
    \end{array}
    \Bigr.
    $

  \item Fix point $x^*$ of $\phi(x)$:
    $x^*=\phi(x^*)$.
    If $|\phi'(x^*)|<1$ near $x^*$, then
    $x^*$ is an attractor:
    $x_{k+1}=\phi(x_k)\to x^*$
    for $x_0$ near $x^*$.
    Rate of convergence: $\epsilon_{k+1}=|\phi'(\xi_k)|\epsilon_k$,
    $\xi_k$ between $x_k$ and $x_{k+1}$.
    If $|\phi'(x^*)|>1$, then $x^*$ is a repeller.

  \item Bisect to bracket a root $x^*\in(a,b)$, $f(a)f(b)<0$:
     For $m=\frac{a+b}{2}$, if $f(m)f(a)<0$ then $b:=m$, else $a:=m$.
     Convergence rate: $\epsilon_{k+1}\simeq \epsilon_k/2$.


  \item Newton-Raphson's method for $f(x^*)=0$: If $x_0$ is close to $x^*$,
    $x_{k+1}=x_k - \frac{f(x_k)}{f'(x_k)}\to x^*$.
    If $f'(x^*)\neq0$ and $f''$ is continuous, the rate of convergence is quadratic:
     $\epsilon_{k+1}\simeq C\epsilon_k^2$. For $x_k$ close to $x^*$,
     $C\simeq |\frac{f''(x^*)}{2f'(x^*)}|\cdot$

  \item Secant method for $f(x^*)=0$. For $x_0$ close to $x^*$:
    $x_{k+1}=x_k - \frac{f(x_k)(x_k-x_{k-1})}{f(x_k)-f(x_{k-1})}\to x^*$,
    % Convergence rate:
     $\epsilon_{k+1}\simeq C\epsilon_k^{1.618}.$

  \item Method independent estimate for the root $x^*$ of $f(x)=0$:
    $\epsilon_k=|x_k-x^*|\le \frac{|f(x_k)|+\delta}{M}$. \\
    Here
    $M\!=\!\min_{x\,\mbox{\tiny near}\,x^*}|f'(x)|$ and $\delta$
    is the rounding error in computing $f$.

\end{itemize}
% >>>

\subsubsection*{Solutions of square non-linear systems, or, roots of vector fields $\Rone^n\to\Rone^n$}% <<<


\begin{itemize}

  \item Fix point $\bdx^*$
    of a field $\bdphi$:
    $\bdphi(\bdx^*)=\bdx^*$.
    If
    $
      \Norm {J_{\bdphi}(\bdx)} \le m < 1
    $
    near $\bdx^*$,
    then $\bdx^*$ is an attractor point, i.e., \\
    $\bdx^{[i+1]}=\bdphi(\bdx^{[i]})\to\bdx^*$, $i\to\infty$, for
    $\bdx^{[0]}$ near $\bdx^*$.

  \item Newton-Raphson's iteration scheme to find the root $\bdx^*$ of
    the system \/
        $
          \bdf(\bdx) = \bdzero
          \,\iff
          \begin{cases}
            f_1(x_1,\ldots,x_n) = 0, \\[-5pt]
            \hspace{3.3em}\vdots \\[-9pt]
            f_n(x_1,\ldots,x_n) = 0. \\
          \end{cases}
        $
        \[
        \text{Starting at $\bdx^{[0]}$ near the root $\bdx^*$:}
        \begin{cases}
          \bdx^{[i+1]}
          =
          \bdx^{[i]}
          -
          J_{\bdf}^{-1}(\bdx^{[i]})\bdf(\bdx^{[i]}) \to \bdx^*, \\
        \hfill\text{as }~i\to\infty,
        \end{cases}
        \iff
          \begin{cases}
            \bdx^{[i+1]}
            =
            \bdx^{[i]}+\bdh^{[i]}, \\ %~\text{where}\\
            J_{\bdf}(\bdx^{[i]})\bdh^{[i]}=-\bdf(\bdx^{[i]}).
          \end{cases}
        \]
  % for all starting
\end{itemize}
% >>>

\subsubsection*{Polynomials: $p_n(x) = a_nx^n+\cdots+a_1x+a_0$}%<<<

\begin{itemize}
  \item Euklides division: $p_n(x)=(x-x_0)(b_{n-1}x^{n-1}+\cdots+b_1x+b_0)+r$,\; $r=p_n(x_0)$.
    The coefficients $b_k$ can be obtained by synthetic division (Horner's rule) below:
  \item Horner's rule:\;
 $
 \begin{array}{c|*{6}{c}}
    & a_n     & a_{n-1} & \cdots & a_2& a_1& a_0 \\ \hline
   \rule[0pt]{0pt}{11pt}
   x& b_{n-1} & b_{n-2} & \cdots & b_1& b_0& p_n(x) \\
 \end{array}
 \quad
    b_{k-1}
    =
    \begin{cases}
      \hfil a_n ,& k=n \\
      b_{k} + a_{k-1}x ,& k < n.\\
    \end{cases}
 $

\item $k$ recursions of Horner at the same $x$
 yield the derivatives as a last coefficient:
 $b_0^{(k)}=\frac{D^kp_n(x)}{k!}$.

% Commented utvidgad Horner:<<<
%%%   \item Horner's rule:\;
%%%  \(
%%%  \begin{array}[t]{c|*{6}{c}}
%%%     x & a_n     & a_{n-1} & \cdots & a_2& a_1& a_0 \\ \hline
%%%    \rule[0pt]{0pt}{11pt}
%%%    x& b_{n-1} & b_{n-2} & \cdots & b_1& b_0& p_n(x) \\
%%%    \rule[0pt]{0pt}{11pt}
%%%    x& b'_{n-2} & b'_{n-3} & \cdots & b'_0& p_n'(x) \\
%%%    \rule[0pt]{0pt}{12pt}
%%%    x& b''_{n-3} & b''_{n-4} & \cdots & \frac{p_n''(x)}{2!} \\
%%%    \vdots & \vdots \\
%%%    x& b^{(n-1)}_1 & \frac{p^{(n-1)}(x)}{(n-1)!} \\
%%%    x& \frac{p^{(n)}(x)}{n!} \\
%%%  \end{array}
%%%     \begin{array}[t]{l}
%%%       \\[3em]
%%%     b_{n-1} = a_n,\\
%%%     b_{k-1} = xb_{k}+a_{k-1}, \; k < n.\\
%%%       \\[3pt]
%%%       \text{The coefficients on each row} \\
%%%       \text{are computed recursively} \\
%%%       \text{from the ones} \\
%%%       \text{on the previous one}. \\
%%%     \end{array}
%%%  \)% >>>

\end{itemize}%>>>

\subsubsection*{The Divided diffences (DD) $[\bdx\!:\!\bdf]$, $\bdx=\{x_i\}_0^n$, $\bdf=\{f_i\}_0^n=\{f(x_i)\}_0^n$} %<<<

{\footnotesize
\[
  \begin{array}[m]{c|c*{6}{l}}
    x_0 & {f_0}
        & {[f_0\!:\!f_1]}
        & {[f_0\!:\!f_2]}
        & \cdots
        & {[f_0\!:\!f_{n-1}]}
        & {[f_0\!:\!f_{n}]} \\
    x_1 & f_1 & [f_1\!:\!f_2] & [f_1\!:\!f_3]
        & \hfill\cdots & [f_1\!:\!f_{n}] \\
        & \vdots & \\
    x_i & f_i & \hfil\cdots &  [f_i\!:\!f_{i+j}]\\
        &   \vdots & \\
    x_{n-1} & f_{n-1} & [f_{n-1}\!:\!f_n] \\
    x_{n}   & f_{n} \\
  \end{array}
  \quad
      [f_i:f_{i+j}]
      =
      \begin{cases}
       \hfill f_i\hfill , & j=0, \\[4pt]
        \frac{[f_{i+1}\!:\!f_{i+j}]-[f_i\!:\!f_{i+j-1}]}{x_{i+j}-x_i},
        & j \ge 1.
      \end{cases}
\]
}

%>>>

\subsubsection*{Weighted Divided diffences (wDD) $\wDD[\bddelta:\bdf]\!:$} %<<<
{\footnotesize
\[
  \begin{array}[m]{c|c*{5}{l}}
    \delta_0 & {f_0}
             & {\wDD[f_0:f_1]}
             & \cdots
             & {\wDD[f_0:f_{n-1}]}
             & {\wDD[f_0:f_{n}]} \\
    \delta_1 & f_1 & \wDD[f_1:f_2]
                 & \cdots & \wDD[f_1:f_{n}] \\
                 & \vdots & \\
    \delta_{n-1} & f_{n-1} & \wDD[f_{n-1}:f_n] \\
    \delta_{n}   & f_{n} \\
  \end{array}
  \quad
\begin{aligned}
  \wDD[f_i:f_{i+j}]
  =
  \begin{cases}
    \hfill f_i\hfill , & j=0
    \\[7pt]
  \frac{1}{\delta_i - \delta_{i+j}}
  \begin{vmatrix}
   \delta_i     & \wDD[f_i:f_{i+j-1}] \\
   \delta_{i+j} & \wDD[f_{i+1}:f_{i+j}],
  \end{vmatrix},
    & j \ge 1.
  \end{cases}
\end{aligned}
\]
}
Scaling scaling the \textbf{first} data vector does not change the scheme:
$
  \wDD[\bddelta:\bdf]
  =\wDD[\lambda\bddelta:\bdf]
$ for any $\lambda\neq0$.

%>>>

\subsubsection*{Interpolation Polynomials (IP)} %<<<
\begin{itemize}

  \item For $x_0<\cdots<x_n$,
    there exists a unique IP $p_n(x)$ of $\deg p_n \le n$
    satisfying $p_n(x_i)=f_i$,
    $i=0,1,\ldots,n$.

  \item Lagrangian form:
      \(
      \displaystyle
        p_n(x) = \sum_{i=0}^{n} f_iL_i(x), \;
        L_i(x) = \prod_{j\neq i}\frac{x-x_j}{x_i-x_j};
        \quad L_i(x_j) =
        \begin{cases}
          1, i=j, \\
          0, i\neq j.
        \end{cases}
      \)

 \item Newton form:
    $
    \displaystyle
      p_n(x) = \sum_{i=0}^{n} c_iN_i(x)
    $,
    where
    $
    c_i = {[f_0\!:\!f_i]}
% \begin{cases}
%   \hfill y_0\hfill,\; i=0, \\
%   \frac{y_i-p_{i-1}(x_i)}{N_i(x_i)},\; i>0.
% \end{cases}
    $
    (DD-scheme),
    and
    $
    N_i(x) =
    \begin{cases}
      \hfill1\hfill,\; i=0, \\
      (x-x_0)\ldots(x-x_{i-1}),\; i>0.
    \end{cases}
    $

    \item Evaluation of the IP at a given point $x$ using wDD (Neville scheme):
      $p_n(x)=\wDD[f_0:f_n]$, with $\delta_i=x-x_i$.

    \item Pointwise error:
    If $f\in C^{n+1}$ and $\{f_i\}_0^n=\{x_i\}_0^n$,
      then \/
    $
      f(x)-p_n(x)
    = \frac{f^{(n+1)}(\xi)}{(n+1)!}N_{n+1}(x)
    %= [f_0\!:\!f_n]N_{n+1}(x)
    $,
    $x_0<\xi<x_n$.


\end{itemize}
%>>>

\subsubsection*{Differentiation as a linear map, discretization  and numerical approximation of of derivatives}% <<<

\begin{itemize}

   \item $\calL$ is a linear map acting on the vector space of some class of functions if
     $\calL (\alpha f+\beta g)=\alpha\calL f+\beta\calL g$.
     Powers: $\calL^n\defeq\calL\calL^{n-1}$,
     $\calL^0\!f=I\!f=f$ (identity).

  \item Differentiation $D\!:\!f\to f'$ is linear.
  Polynomial action, e.g.: $(D^2+6D+9I)f=f''+6f'+f=(D+3I)^2f$.

  \item The step operator: $T_h\!:f(x)\to f(x+h)$,
    $T_hT_{-h}=I$. Powers: $T^n_hf(x)=T_{nh}f(x)=f(x+nh)$.

  \item Taylor/Maclaurin($a=0$) expansion an analytic function around  point $a$
    $\iff$
    action of forward differences:
    \[
      f(a+h)
      =T_hf(a)
      =\sum_{k=0}^\infty\frac{f^{(k)}(a)}{k!}h^k
      =\sum_{k=0}^\infty\frac{(hD)^k}{k!}f(a)
      = e^{hD}f(a); \quad
      % \;\rightsquigarrow\; T_h = e^{hD};
      \Delta_h=T_h-I=e^{hD}-I.
      % , \quad \Delta_hf(a)= f(a+h)-f(a).
    \]

  \item The Mean Value Thm in terms of $T_h$ acting on $C^{n+1}$ functions:
    $
     T_h = \sum_{k=0}^n\frac{(hD)^k}{k!}
      + \Ordo(h^{n+1})D^{n+1}
    $.

  \item Forward and backward
    differences operators and correponding
    approximation of the derivative:
    \[
      D_h=\frac{\Delta_h}{h}=\frac{T_h-I}{h}=\frac{e^{hD}-I}{h},\quad
      D_{-h} =\frac{I-T_h}{h} = T_{-h}D_h=e^{-hD}D_h,
      \quad
      D_{\pm h}f(a)=f'(a) + \Ordo(h).
    \]

  \item Central
    differences differential operators and corresponding
    approximation of $f'$ and $f''$:
    \begin{align*}
      \bar D_h
      &= \frac{D_h+D_{-h}}{2}
      = \frac{T_h-T_{-h}}{2h},
      &
      \bar D_hf(a)
      &= \frac{f(a+h)-f(a-h)}{2h}
      = f'(a) + \Ordo(h^2);
      \\
      \bar D_{\frac h2}^2
      % \Bigl(\frac{D_{\frac h2}+D_{-\frac h2}}{2}\Bigr)^2
      &= \frac{(T_{\frac h2}-T_{-\frac h2})^2}{h^2}
       = \frac{T_{h}-2I+T_{-h}}{h^2},
      &
      \bar D_{\frac h2}^2f(a)
       &= \frac{f(a+h)-2f(a)+f(a-h)}{h^2}
        = f''(a) + \Ordo(h^2).
    \end{align*}

  \item Forward/backward approximations for
    $f''\!: D_{\pm h}^2f(a) = f''(a) + \Ordo(h)$.
    Non-symmetric ditto for $f'''$:
    \[
       \bar D_{\pm h}\bar D_{\frac h2}^2 f(a)
     = f'''(a) + \Ordo(h^2);
     \qquad
     \{
       D_h^3,
       D_h^2D_{-h},
       D_hD_{-h}^2,
       D_{-h}^3
     \}f(a)
     =
     f'''(a) + \Ordo(h).
    \]

  \item Divided differences (DD)
    of $f$ on even grid: $x_k=a+kh$:
    $
    [f_0\!:\!f_k]
    =
     \frac{T_h-I}{kh}[f_0\!:\!f_{k-1}]
     = \cdots
     =\frac{D_h^kf(a)}{k!}
     \cdot
    $
    % $k=0,1,\ldots,n$

  \item Newton interpolation polynomial
    of $f\in C^{n+1}$ on regular meshes $x_i=a+kh$,
    $k=0,1,\ldots,n$:
    \[
      P_n(x)
      = \sum_{k=0}^n
         \frac{D_h^kf(a)}{k!}
        (x-a)^{\scriptsize\b kh},
        \quad
        (x-a)^{\scriptsize\b kh}
         = (x-a)(x-a-h)\cdots(x-a-(k-1)h)
    \]

\end{itemize}% >>>

\subsubsection*{Numerical integration}% <<<

\begin{itemize}
  \item (Open) Newton-Cotes approximations are
  weighted sums of integrand values at equally spaced points:
  \[
\lower3em
\hbox{\begin{array}[t]{l}
    \int_a^bf(x)\,dx
      = \sum_{i=0}^{n} w_if_i,
      \\[17pt]
      f_i =f(a+ih),
      \\[12pt]
      h = \frac{b-a}{n}
    \end{array}}
  \quad
    \begin{array}[t]{*{12}{|c}|}
      \hline n & h & \text{Rule name}& \text{Formula} & \text{Error term} \\ \hline
      \rule[0pt]{0pt}{18pt} 1 & b-a &  \text{Trapezoidal}
      & Q_hf = \frac{f_0+f_1}{2}\,h & -\frac{h^3}{12}f''(\xi)
      \\[5pt] \hline
      \rule[0pt]{0pt}{18pt} 2 & \frac{b-a}{2} &  \text{Simpson $\frac13$}
      & Q^{(2)}_{h}f=\frac{f_0+4f_1+f_2}{3}\,h & -\frac{h^5}{90}f^{(4)}(\xi)
      \\[5pt] \hline
      \rule[0pt]{0pt}{18pt} 3 & \frac{b-a}{3} &  \text{Simpson $\frac38$}
      & Q^{(3)}_{h}f=\frac{3(f_0+3f_1+3f_2+f_2)}{8}\,h & -\frac{3h^5}{80}f^{(4)}(\xi)
      \\[5pt] \hline
    \end{array}
  \]

  \item
  \textbf{The composite rules} on a larger interval $[a,b]$ subdivided evenly in
    $\{(a+ih,f_i)\}_0^n$, $x=\frac{b-a}{n}$:

    \[
      \int_a^bf(x)\,dx
      =
      \begin{cases}
        h(f_0+2f_1+\cdots+2f_{n-1}+f_n)/2 + \Ordo(h^2),
        & \text{(Trapezoidal)}
        \\
        h(f_0+4f_1+2f_2+4f_3+\cdots+2f_{n-2}+4f_{n-1}+f_n)/3 + \Ordo(h^4),
        ~n=2m,
        & \text{(Simpson $\frac13$)}
      \end{cases}
    \]

\end{itemize}% >>>

\subsubsection*{Log-scale wDD (weighted Div Diffs) to crank up convergence rate of finite differences schemes} %<<<

\begin{itemize}
\item
Approximation of $F(0)=\lim_{x\to0}F(x)$ when $F(0)=F(h) + \Ordo(h^k)$ with
\textbf{gap in powers $k$} in the Maclaurin expansion:
\[
  F(h) = F(0) + a_1h^k+a_2h^{2k}+\cdots+a_lh^{lk} + \Ordo(h^{(l+1)k}),\; h\to 0.
\]
The  wDD-table
\[
  \wDD[\bddelta:\bdF]
  =\wDD[(1,q^k,q^{2k},\ldots,q^{lk}):(F(h),F(qh),F(q^2h),\ldots,F(q^lh)]
\]
 with refinement ratio $0<q<1$, has last term
\[
  \boxed{\calR^lF(h)=\wDD[F(h):F(q^lh)]=F(0) + \Ordo(h^{(l+1)k}).}
\]

  \item \textbf{Richardson extrapolation}
    in approximating the derivatives $f'$, $f''$, etc., of a smooth function $f$:
    \[
    \begin{array}{c|*{5}{c}}
      F(h) & D_{\pm h}f & \bar D_hf & \bar D^2_{\frac h2}f & \cdots \\[5pt]
      \hline \rule[0pt]{0pt}{9pt}
      k & 1 & 2 & 2 & \cdots \\
    \end{array}
  \]
  \item \textbf{Romberg integration} = Richardson extrapolation
    when approximating $\int_a^b \!\! f(x)\,dx$ of a smooth integrand $f$ with the
    composite Trapezoidal rule:
    \[
      F(h)=Q_hf,\quad
      h=\frac{b-a}{n},\quad
      k=2,\quad
      ~\text{where the refinement ratio}~
      q=\frac1r,\quad
      n,r\in\Zone_+.
    \]
\end{itemize}

%>>>

\medskip\hrule

% -- Convex sets and functions:

\subsubsection*{Convex sets of $\Rone^n$}% <<<

Below
$\Omega\subset\Rone^n$,
$0\le \lambda\le 1$,
$\bdlambda=(\lambda_1,\ldots,\lambda_m)^T\ge\bdzero$
and form a discrete distribution, i.e.,
$\bdone^T\bdlambda=\sum_{i=1}^m \lambda_i=1$.

\begin{itemize}
  \item A set $\Omega\subset\Rone^n$ is convex if
    $(1-\lambda)\bdx^{(1)}+\lambda\bdx^{(2)}$
    for any
    $\bdx^{(1)},\bdx^{(2)}\in\Omega$ and
    $0\le\lambda\le1$.

  \item If $\Omega_i\subset\Rone^n$ are all convex, then their
    intersection (finite or infinite) $\cap_i \Omega_i$ is convex.

  \item Convex linear combinations (CLC):
    $\sum\lambda_i\bdx^{(i)}$, $\bdx^{(i)}\in\Omega$.
    A set is convex iff it contains all
    its CLC's.

  \item The Separation Hyperplan Thm: Let $\Omega\neq\varnothing$ be convex,
    closed and $\bdy\notin\Omega$.
    Then there exist
    a normal vector $\bdn$ and a scalar $b$, such that $\bdn^T\bdy>b$
    and $\bdn^T\bdx\le b$ for all $\bdx\in\Omega$.
    Moreover, if $\hat\bdx\in\partial\Omega$, then $\bdn^T(\bdx-\hat\bdx)<0$
    for all $\bdx\in\Omega$.
\end{itemize}
% >>>

\subsubsection*{Convex functions $\Rone^n\to\Rone^1$}% <<<

\begin{itemize}
  \item Let $\bdx=(1-\lambda)\bda+\lambda\bdb$.
    The function
    $f\!:\Rone^n\to\Rone$ is \textbf{convex} if
    $
    f(\bdx)
    \le
    (1-\lambda) f(\bda)+\lambda f(\bdb)
    $.
    Moreover,

    \[
      % \text{Moreover, the following inequality holds: }~
      \frac{f(\bdx)-f(\bda)}{\Norm{\bdx-\bda}}
      \le
      \frac{f(\bdb)-f(\bda)}{\Norm{\bdb-\bda}}
      \le
      \frac{f(\bdb)-f(\bdx)}{\Norm{\bdb-\bdx}},
      \quad 0 < \lambda < 1.
    \]
  Jensen's inequality generalises for any convex linear combination of points:
    $
    f(\sum_{i=1}^m\lambda_i\bdx^{(i)})
    \le
    \sum_{i=1}^m\lambda_i f(\bdx^{(i)}).
    $

   \item The function $f\!:\Rone^n\to\Rone$ is \textbf{concave} if $(-f)$ is convex, i.e.,
    $
    f[(1-\lambda)\bda+\lambda\bdb]
    \ge
    (1-\lambda) f(\bda)+\lambda f(\bdb)
    $.

  \item If both $f_{1,2}$ are convex and $\alpha_{1,2}\ge0$, then $\alpha_1f_1+\alpha_2 f_2$
    and $\max\{f_1,f_2\}$ are convex.

  \item If both
    $f\!:\Rone\to\Rone$
    and
    $g\!:\Rone^n\to\Rone$
    are convex and $f$ is increasing, then
    $f\circ g(\bdx)=f[g(\bdx)]$ is convex.

  \item (Graph above tangent) If $f$ is locally convex and differentiable at $\bda$, then
  $f(\bdx)\ge f(\bda)+\nabla f(\bda)^T(\bdx-\bda)$.

  \item If $f\!:\Rone\to\Rone\in C^2$ is locally convex at $a$, then $f''(a)>0$

  \item If $f\!:\Rone^n\to\Rone\in C^2$ is locally convex, then the
    Hessian matrix is \textbf{s}ymmetric and \textbf{p}ositive \textbf{d}efinite (\textbf{spd}):
    \[
      H(\bda)=\nabla\nabla^T f(\bda)
      =
      \left[\frac{\partial^2f(\bda)}{ \partial x_i\partial x_j}\right]_{i,j=1}^n
      =H^T(\bda)\succeq 0,
      ~\text{ i.e. }~
      \bdx^T\!H(\bda)\bdx\ge0,
      \quad \forall\bdx\in\Rone^n.
    \]

\end{itemize}
% >>>

% -- Non-linear programming and the KKT-conditions:

\subsubsection*{Non-linear programming (NLP). Convex optimization problems}% <<<

\begin{itemize}
  \item An optimization problem $\{\text{min}~f(\bdx), \; \bdx\in\Omega\subset\Rone^n\}$
    is \textbf{convex} if $f$ is convex
    and $\Omega$ is convex.
  \item If $\bdx^*\in\Omega$ is  a local optimum of \textbf{convex} problem
    $\{\text{min}~f(\bdx), \; \bdx\in\Omega\}$, then $f(\bdx)\ge f(\bdx^*)$ for all $\bdx\in\Omega$.
\end{itemize}% >>>

\subsubsection*{Lagrangian relaxation of non-linear problems (NLP) under all-equality constraints}% <<<

\begin{itemize}
  \item Given is constrained NLP (opt = max/min):
    \hfil
\(
  P\!:
  \begin{cases}
    \text{opt} & f(\bdx),\; \bdx\in S, \\
      \hfill
      \text{s.t.} & S\!: \bdg(\bdx)=\bdzero,
  \end{cases}
    \quad
    \bdg(\bdx)
    =
    \begin{bmatrix}
      g_1(\bdx)\\[-2pt] \vdots\\ g_m(\bdx)
    \end{bmatrix}
    \mkern-6mu,
    \quad
    \bdx\in\Rone^n.
\)

\item Lagrangian relaxation of the constrained NLP
  $P$ to an unconstrained $L$ in $\Rone^{n+m}$:
\[
  L\!: \calL(\bdx,\bdlambda)
  = f(\bdx)-\bdlambda^T\bdg(\bdx)
  = f(\bdx)-\sum_{i=1}^m \lambda_ig_i(\bdx),
  \quad
  \bdlambda\in\Rone^m.
\]
\item Lagrange Theorem: If $f,\bdg\in C^1$, $(\bdx^*,\bdlambda^*)$ is a critical point of $L$ and
$\{\nabla g_1(\bdx^*),\ldots,\nabla g_1(\bdx^*)\}$ are linearly independent, then
$\bdx^*$ is also critical for $f_{|_S}$ from $P$ and their value values coincide:
\[
  \nabla \calL(\bdx^*,\bdlambda^*)
  =
  \tvavektor[\nabla_\bdx,\nabla_{\bdlambda}] \calL(\bdx^*,\bdlambda^*)
  =
  \begin{bmatrix}
    \nabla f(\bdx^*)-\nabla\bdg^T\!(\bdx)\,\bdlambda \\
    - \bdg(\bdx^*)
  \end{bmatrix}
  =
  \begin{bmatrix}
    \bdzero_{n\times1} \\
    \bdzero_{m\times1}
  \end{bmatrix}
  \;\ergo\;
  \begin{cases}
    \hfill
    \nabla f(\bdx^*)&=\hfil \bdzero, \\
    \calL(\bdx^*,\bdlambda^*)&=f(\bdx^*).
  \end{cases}
\]
\Obs! $\bdx^*$ being a critical point for $f_{|_S}$ is necessary,
but not sufficient for opt
of smooth NLP's.
\end{itemize}
% >>>

\subsubsection*{Optimization of normal non-linear problems (NLP) under all-inequality constraints}% <<<

  All NLP's below are formulated in \textbf{normal form}:
  \[
    P_{\min}\!:
    \begin{cases}
      \min        & f(\bdx),\; \bdx\in\Rone^n, \\
      \hfill
      \text{s.t.} & \bdg(\bdx)\ge\bdzero,
    \end{cases}
    \quad
    \text{or}
    \quad
    P_{\max}\!:
    \begin{cases}
      \max        & f(\bdx),\; \bdx\in\Rone^n, \\
      \hfill
      \text{s.t.} & \bdg(\bdx)\le\bdzero,
    \end{cases}
    \quad
    \text{where}
    \quad
    \bdg(\bdx)
    =
    \begin{bmatrix}
      g_1(\bdx)\\[-2pt] \vdots\\ g_m(\bdx)
    \end{bmatrix}
    \mkern-6mu.
  \]
 \Obs! There are no implicit assumptions about non-negativity of the decision variables $\bdx$.
 If present, these have to be included explicitly as constraints, e.g., $g_i(\bdx)=x_i\ge0$.

 \begin{itemize}
   \item Lagrangian relaxation:
    $
      \calL(\bdx,\bdmu)
      = f(\bdx)
      -
      \bdmu^T\bdg(\bdx)
  $,
  $
  (\bdx,\bdmu)\in\Rone^{n+m},\;
    \bdmu\ge\bdzero.
   $

  \item The point $(\bdx^*,\bdmu^*)\in\Rone^{n+m}$ is a saddle point for the Lagrangian, if
    \[
      \text{in}~P_{\min}:
      \begin{cases}
          \calL(M^*) = \min_{\bdx}\max_{\bdmu} \calL(\bdx  ,\bdmu),~\text{i.e.}, \\
          \calL(\bdx^*,\bdmu  ) \le \calL(M^*) \le \calL(\bdx  ,\bdmu^*),
        \end{cases}
        \hspace{1em}
        -\text{and}-
        \hspace{2em}
      \text{in}~P_{\max}:
        \begin{cases}
          \calL(M^*) = \max_{\bdx}\min_{\bdmu} \calL(\bdx  ,\bdmu),~\text{i.e.}, \\
          \calL(\bdx  ,\bdmu^*) \le \calL(M^*) \le \calL(\bdx^*,\bdmu).
      \end{cases}
    \]

    \item If a normal NLP has a saddle point $(\bdx^*,\bdmu^*)$, then $\bdx^*$
    is an optimum for the corresponding
    $P_{\min}$ or
    $P_{\max}$.

 \end{itemize}

% >>>

 \subsubsection*{The Karush-Kuhn-Tucker (KKT) conditions for $\bdx^*$ to be a local opt of a smooth normal NLP}% <<<


    \medskip
    \Thm.~For $x^*$ to be a local optimum of a normal min or max NLP $P$
    with $f,\bdg\in C^1$ and linearly independent gradients $\nabla g_i(\bdx^*)$
    of the constraints active at $\bdx^*$, it is necessary that the following
    KKT conditions hold:

    \[
        \boxed{
        \begin{array}[t]{rl}
        \textbf{K1:}
        & \text{Primal Feasibility:} \\[4pt]
        \multicolumn{2}{r}{\text{in}~P_{\min}\!:\; \bdg(\bdx^*)\ge\bdzero} \\[5pt]
        \multicolumn{2}{r}{\text{in}~P_{\max}\!:\; \bdg(\bdx^*)\le\bdzero} \\
        \end{array}}
        \hspace{2em}
        \boxed{
        \begin{array}[t]{l}
        \textbf{K2: }
        \bdx^*~\text{is stationary:} \\[3pt]
         \nabla f(\bdx^*)=\sum_{i=1}^m\mu_i\nabla\bdg(\bdx^*) \\
        \end{array}}
        \hspace{1em}
        \begin{array}[t]{l}
          \boxed{
            \begin{array}[t]{rl}
              \textbf{K3:} & \text{Dual feasibility: }~\bdmu \ge \bdzero
          \end{array}}
          \\[5pt]
          \boxed{
            \begin{array}[t]{rl}
              \textbf{K4:}&\text{Complementary slackness} \\
                          &\mu_ig_i(\bdx^*) = 0,\; 1\le i\le m.
          \end{array}}
        \end{array}
    \]

    The constant vector $\bdmu\in\Rone^m_+$ is the vector of \textbf{dual} variables for the
    corresponding constraints.
% >>>

% -- LP / The Simplex method:

\subsubsection*{The Simplex Method for solving Linear Programming problems:}%<<<
  \begin{tabular}{ll}
Parameters: &
  $\bdb_{m\times1}$,
  $A_{m\times n}$,
  $\bdc_{n\times1}$,
  % $\bdzero=(0,\ldots,0)^T$,
  % $I_m$ (unit matrix),
  (intermediate values: $\hat \bdb$, $\hat\bdc$)
\cr
Variables: &
  $z$, $w$ (primal and dual objective~functions),
  $\bdx_{n\times1}$ (primal decision vars),
  $\bdy_{m\times1}$ (dual vars),
\cr
  &
  $z^*$, $\bdx^*$,
  $w^*$, $\bdy^*$
  (values at opt);
  $\bds_{m\times1}=\bds^+\cup \bds^-$ (slack/surplus),
  $\bda^-$ (artificial). % , $|\bds^-|=|\bda^-|$
\cr
\end{tabular}
%>>>

%%\subsubsection*{Normal max LP problem:}%<<<
\[
  \begin{array}[m]{c|c|c}
    \textbf{Normal max LP problem:}
    & \textbf{Normal min LP:}
    & \textbf{Generic LP:}
  \\ \hline
    \begin{array}[m]{ll}
      \max & z = c_1x_1+\cdots+c_nx_n, \;\bdx\ge0, \cr
      \mbox{s.t.} &
      \mkern-15mu
      \begin{cases}
        a_{11}x_1+\cdots+a_{1n}x_n &\le b_1 \\[-3pt]
        \hfil\vdots &  \hfil\vdots          \\[-3pt]
        a_{m1}x_1+\cdots+a_{mn}x_n &\le b_m \cr
       %\hfill x_1,x_2,\ldots,x_n  &\ge 0.
      \end{cases}
    \end{array}
  \;\leftrightsquigarrow\;
  \boxed{
    \begin{array}[m]{ll}
      \max & z=\bdc^T\bdx\cr
      \mbox{s.t.} &
      \mkern-15mu
      \begin{cases}
      A\bdx\le\bdb\cr
      \hfill\bdx\ge\bdzero
      \end{cases}
    \end{array}
  } % boxed
  &
  \boxed{
    \begin{array}[m]{ll}
      \min & z=\bdc^T\bdx\cr
      \mbox{s.t.} &
      \mkern-15mu
      \begin{cases}
      A\bdx\ge\bdb\cr
      \hfill\bdx\ge\bdzero
      \end{cases}
    \end{array}
  } % boxed
  &
  \boxed{
    \begin{array}[m]{ll}
      \text{opt} & z=\bdc^T\bdx\cr
      \mbox{s.t.} &
      \mkern-15mu
      \begin{cases}
      A\bdx\lessgtr\bdb\cr
      \hfill\bdx\ge\bdzero
      \end{cases}
    \end{array}
  } % boxed
  \end{array}
\]
The LP: \fbox{$\min z$, s.t. conditions}
  $\iff$
  LP: \fbox{$\max \;-z$, subject to the \textbf{same} conditions.}
%>>>

\subsubsection*{Normal max LP in standard form. Initial and Final Simplex Tableau at opt (actual layout):}%<<<
\[
  \begin{cases}
    \begin{array}[m]{*{5}{lc}}
    z &-& \bdc^T\bdx &+& \bdzero^T\bds &=&  0   \cr
      & &   A\bdx    &+&   I_m\bds     &=& \bdb \cr
    \end{array}
  \end{cases}
  \mkern-16mu\rightsquigarrow\;
    \begin{array}[m]{c|ccc|c}
 \perp &     z   & \bdx^T & \bds^T     & \hat\bdb \cr\hline\rule{0pt}{10pt}
    z  &     1   &-\bdc^T & \bdzero^T  &   0    \cr %\hline
 \bds  & \bdzero & A      & I_m        & \bdb   \cr
    \end{array}
  \quad\sim\cdots\sim\quad
    \begin{array}[m]{c|ccc|c}
 \perp &     z   &     \bdx^T &     \bds^T   & \hat\bdb \cr\hline\rule{0pt}{10pt}
 z  &     1   &-\hat\bdc^T &     \bdy^{*T} & z^*\cr %\hline
 \bdx_B& \bdzero &     B^{-1}A&     B^{-1}   & \bdx_B^*   \cr
    \end{array}
\]
Split into Basic/Non-basic variables,
$(\bdx,\bds)\to(\bdx_B,\bdx_N)$,
 and matrices:
    $B=\cols(\bdx_B)$,
    $N=\cols(\bdx_N)$
%>>>

\subsubsection*{The Final Tableau of the Normal max LP at opt, layout split in (B)asic/(N)on-basic columns:}%<<<
\[
  \begin{cases}
    \begin{array}[m]{*{5}{lc}}
      z^* &+& \bdzero^T\bdx_B^* &-& \hat\bdc_N^T\bdx_N &=&  \bdc_B^TB^{-1}\bdb \cr
    %%%          &&&&                       &=& \bdc_B^T\bdx_B^* = \bdb^T\bdy^*\cr
      & &    I_m\bdx_B^*  &+& B^{-1}N\bdx_N     &=& B^{-1}\bdb \cr
    \end{array}
  \end{cases}
  \;\rightsquigarrow\quad\;
    \begin{array}[m]{c|ccc|c}
   \perp &     z   & \bdx_B^T  & \bdx_N^T & \hat\bdb \cr\hline\rule{0pt}{10pt}
      z^*&     1   & \bdzero^T &-\hat\bdc_N^T & \bdc_B^TB^{-1}\bdb \cr %\hline
  \bdx_B & \bdzero &  I_m  & B^{-1}N      &  B^{-1}\bdb   \cr
    \end{array}
\]
Here \fbox{$\bdy^{*T}=\bdc_B^TB^{-1}$} are the dual vars and
\fbox{$\hat\bdc_N^T=\bdc_N^T-\bdc_B^TB^{-1}N=\bdc^T-y^{*T}N\le0$} the reduced costs at max.
%>>>

% Weak + Strong Duality<<<
\bigskip
\textbf{Weak duality}: For any feasible $\bdx$ in (P) and $\bdy$ in (D) holds:
\fbox{$
  z = \bdc^T\bdx \le \bdy^TA\bdx \le \bdy^T\bdb = w.
$}
\begin{itemize}
  \item If (P) is unbounded $\ergo$ (D) is infeasible.
        If (D) is unbounded $\ergo$ (P) is infeasible
  \item If both $\bdx^*$ is feasible in (P) and
                $\bdy^*$ is feasible in (D) and
                \[
                  \boxed{z^*=\bdc^T\bdx^* = \bdy^{*T}A\bdx^*=\bdb^T\bdy^*=w^*,}
                  \eqno(\star)
                \]
                then
                $\bdx^*$ is optimal in (P)
           and  $\bdy^*$ is optimal in (D).
\end{itemize}

\textbf{Strong duality}: If either (P) or (D) has a bounded optimal solution,
so has the other and $(\star)$ holds.%>>>

\medskip\textbf{Complementary slackness at optimum}: %<<<
$
\begin{cases}
  \bdy^{*T}(\bdb-A\bdx^*)=0
  \iff y_i^*(b_i-\sum_{j=1}^{n} a_{ij}x_j^*) = 0, \;
   i=1,\ldots,m
\cr
  (\bdy^{*T}A-\bdc)\bdx^*=0 \iff x_j^*(\sum_{i=1}^m y_i^*a_{ij}-c_j)=0,
  \; j=1,\ldots, n.\cr
\end{cases}
$%>>>

\vspace{-1em}\subsubsection*{Duality} %<<<
\(
\boxed{
  \begin{array}[m]{l}
   \textbf{Normal LP's have:}  \\
   \bullet~\mbox{normal constraints:}~\max\!:\,\le,\,\min\!:\,\ge \\
   \bullet~\mbox{normal decis.~vars:}~\bdx\ge\bdzero,\;\bdy\ge\bdzero \\
  \end{array}
}
\;\rightsquigarrow\;
\boxed{
  (P)\!:
  \begin{cases}
    \max~z=\bdc^T\bdx \cr
    A\bdx\le\bdb,\;  \bdx\ge\bdzero
  \end{cases}
}
   \;\overset{1:1}\leftrightsquigarrow\;
   \boxed{
     (D)\!:\begin{cases}
       \min~w=\bdb^T\bdy\cr
       A^T\bdy\ge\bdc,\;\bdy\ge\bdzero.
     \end{cases}
   }
\)%>>>

\bigskip\textbf{(P)rimal-to-(D)ual correspondence rules:}%<<<
\begin{itemize}
  \item Involution property: (P) $\overset{1:1}\leftrightsquigarrow$(D),
    i.e., the Dual of the Dual is the Primal
  \item \fbox{Normal (P)} $\overset{1:1}\leftrightsquigarrow$ \fbox{Normal (D)}
  \item \fbox{Constraints of reversed inequality} $\overset{1:1}\leftrightsquigarrow$
    \fbox{Non-positive decision~variables}
  \item \fbox{'='-Constraints} $\overset{1:1}\leftrightsquigarrow$
    \fbox{Non-restricted decision variables}
\end{itemize}%>>>

%%% \bigskip \textbf{Involution property}: %<<<
%%% The dual of the dual is the
%%% primal problem. In particular, the dual of a min problem is a max problem and vice versa;
%%% a variable in (P) corresponds to a constraint in (D) and vice versa: %>>>

\end{document}

% -- Överbliven LinAlg-stoff:

\subsubsection*{Linear Algebra}% <<<
All matrices below are square $n\times n$.
$I$ is the unit matrix and $\bdzero$ is the zero matrix.
\begin{itemize}
  \item
    For a non-singular $A$, denote by $A^{-T}=(A^{T})^{-1}=(A^{-1})^{T}$.
  \item
    Conjugation: $A^*=PAP^{-1}$ is the $P$-conjugate of $A$.
    Notation:
    $A^{-*}\defeq (A^{-1})^*=PA^{-1}P^{-1}=(PAP^{-1})^{-1}=(A^*)^{-1}$.
  \item
    If $A=A^T$ is symmetric, then all its eigenvalues are real.
    If
    $\lambda_i\neq \lambda_j$,
    $A\bdv_i=\lambda_i\bdv_i$
    and
    $A\bdv_j=\lambda_j\bdv_j$
    then
    $\bdv_i\perp\bdv_j$.
  \item
    The Spectral theorem: If $A=A^T$, then it is orthogonally diagonalizable:
    $A=VDV^T$,
    where
    $D=\diag[\lambda_1,\cdots,\lambda_n]$
    and
    $VV^T=I$.

  \item
    A matrix $A$ is: \textbf{diagonally dominant}
    if \/
    $
    |a_{ii}|
    \ge
    \sum_{j\neq i} |a_{ij}|
    = |a_{i1}|+\cdots+|a_{i,j-1}|+ |a_{i,j+1}|+\cdots+ |a_{in}|+
    $,
    $1\le i\le n$.
  \item
    A matrix $A$ is
    \textbf{positive definite}, $A\succeq 0$, if
    $\Angled{\bdx,A\bdx}=\bdx^T\!A\bdx\ge0$, $\forall\bdx\in\Rone^n$.
    $A$ is \textbf{spd} (symmetric positive definite)
    if
    $A=A^T\succeq0$.

  \item
    Transposition matrices: $T_{ij} = I$ with swapped rows $i$ and $j$.
    $T_{ij}A$ swaps rows $i$ and $j$ of $A$, while
    $T_{ij}A$ swaps cols $i$ and $j$ of $A$.
  \item
    Permutation matrices:
    $
    P=
    T_{i_1j_1}
      \cdots
    T_{i_kj_k}
    $
    ($I$ with rows arbitrarily transposed).
    Permutation matrices are orthogonal:
    $P^{-1}=P^T$.
\end{itemize}% >>>

\subsubsection*{LU-decomposition}% <<<

\begin{itemize}
  \item
    The partial upper triangular $U_k$ has elements $u_{ij}=0$ for $1\le j<j\le k$.
    Thus $U=U_n$ is upper triangular.
  \item
    $\overset\circ L_k$ has
    non-zero elements only in col $k$ strict under the main diagonal, i.e.,
    $\overset\circ l_{ij}=0$ for $\overset\circ l_{ik}=0$, $i\le j$ or $j\neq k$.
    $\overset\circ L_i\overset\circ L_j=\bdzero_{n\times n}$, $i\le j$
  \item
    $L_k = I+\overset\circ L_k$.
    If
    $l_{ik}=\frac{u_{ik}}{u_{kk}}$,
    then
    $U_k=L_k^{-1}U_{k-1}$,
    $k< i\le n$.
    $L_iL_j=I+L_i+L_j$ for $i\le j$.
    $(I+\overset\circ L_k)(I-\overset\circ L_k)=1$
    $\ergo$
    $L_k^{-1} = I-\overset\circ L_k$.
  \item
    Forward row reduction on $A$,
    $\det(A)\neq 0$,
    is recursive left multiplication:
    $U_1=L_1^{-1}A$,
    \ldots,
    $U_{k}=L_{k}^{-1}U_{k-1}$,
    $1\le k\le n-1$.
  \item
    LU-decomposition  based on Gaussian elimination without pivoting
    $A=LU$,
    $U=U_{n-1}$ and
    $L=L_1L_2\cdots L_{n-1}=I+\sum_{k=1}^{n-1}\overset\circ L_k$.
  \item Pivoting before
    step $k$ of row red.:
    $U_k=L_k^{-1}T_{k}U_{k-1}$,
    where $T_k:=T_{km}$ transposes
    row $k\leftrightarrow m$ with
    $m: |u_{mk}|=\max_{k\le i\le n}|u_{ik}|$.
  \item LU-decomposition:
    $PA=LU$,
    where
    $P=P_1$, \/
    $
    P_k
    =
    \begin{cases}
      T_{n-1}\cdots T_k,& k < n-1, \\
      \hfill I\hfill,& k = n-1, \\
    \end{cases}
    $
    \/ and
    $
    \begin{cases}
      L=L_{n-1}^*\cdots L_1^*, \\
      L_k^* = P_{k+1} L_k P_{k+1}^T.
    \end{cases}
    $
  \item Linear systems through LU-decomposition
    $PA=LU$:
    $
    AX=B
    \;\iff\;
    \begin{cases}
      LY=PB \\
      UX=Y.
    \end{cases}
    $

\end{itemize}% >>>

\subsubsection*{LDL decomposition of spd matrices: $A=A^T\succ0$}% <<<
% Below $A=A^T\succ 0$ is \textbf{spd} and non-singular, i.e.,
% $\Angled{\bdx,A\bdx}>0$ for all $\bdx\neq\bdzero$.
%  $L_k$, $U_k$ are the ones from the LU-decomposition of $A$.
\begin{itemize}
  \item
    LU-decomposition of $LU=A=A^T\succ 0$ is numerically stable and accurate without pivoting.
  \item
    LDL-decomposition:
    $0\prec A^T=A=LU=LDL^T$, where $D\succ 0$ is diagonal,
    $L=L_1\cdots L_{n-1} = I + \sum_{k=1}^{n-1}\overset\circ L_k$.
  \item
    LDL-incrementally:
    $D_k = (L_k\cdots L_1)A(L_k\cdots L_1)^T
    =
    \begin{bmatrix}
      \widehat{D}_k & \bdzero \\
      \bdzero       &\widehat{U}_{n-k} & \\
    \end{bmatrix}
    $,
    $k=1,\cdots,n-1$,
   where
   $\widehat U_{n-k}=[U_k]_{k+1:n,k+1:n}$
   is the lower-right
   %$(n-k)\times(n-k)$
   submatrix
   of $U_k$.
 \item Cholesky decomposition: For spd
   $
   0\prec A=A^T
   =LDL^T
   =L(\sqrt{D})^2L^{-T}
   =L_cL_c^T
   $,
   where
   $
   L_c =L\sqrt{D}.
   $
\end{itemize}
% >>>

% vim: foldmethod=marker:spelllang=en:spell
