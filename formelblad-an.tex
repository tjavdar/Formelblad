% vim: foldmethod=marker
% CHANGELOG:
% Ändring:    IvTj, 2012-01, Dln(x) = 1/x --> Dln(|x|) = 1/x
% Ändring:    IvTj, 2009-01, Lagt till Eulers formler, nya headers
% Ändring:    IvTj, 2008-02, Tagit bort JTH-logot från s. 1
% Rättelse:   IvTj, 2008-02, atan(x)+atan(x) = pi/2 --> atan(x)+atan(1/x) = pi/2
% Rättelse:   IvTj, 2008-02, Vinkel 145^\circ --> 135^circ
% Version 1.1 IvTj, 2007-12, tilläg, ommöblering, latin1->UTF-8
% Version 1.0 FrAb, 2007-10
\voffset-1.5cm
\documentclass{article}
\usepackage[swedish]{babel}
% \usepackage{t1enc,graphicx,a4wide,amsfonts,amsmath,fancyhdr}
\usepackage[utf8]{inputenc} % <-- IvtTj: Changed encoding to UTF-8
\usepackage{graphicx,a4wide,amsfonts,amsmath,fancyhdr}
\everymath{\displaystyle}   % <-- Ändring IvTj
\usepackage{ae,aecompl} %% nice pdf output! <-- Ändring IvTj

\parindent=0pt

\input Lang.h % En-Sv settings outsourced


% Macros <<<
\addtolength{\topmargin}{-35mm}
\addtolength{\headheight}{25ex}
\addtolength{\headsep}{-3ex}
\addtolength{\textheight}{20mm}
\addtolength{\textwidth}{12mm}
%\footheight{ 0ex}
\footskip 3ex
\pagestyle{fancy}

\newcommand{\one}{\marginpar{(1p)}}
\newcommand{\two}{\marginpar{(2p)}}
\newcommand{\three}{\marginpar{(3p)}}
\newcommand{\four}{\marginpar{(4p)}}
\newcommand{\Le}{\Leftrightarrow}

%\newcommand{\D}{\frac{d}{dx}}

\newcommand{\D}{D}

\newtheorem{theorem}{Sats}
\begin{document}

%% <-- 2009-01-21: ändring IvTj, nya headers/footers
%\lhead{\ifnum\thepage=1  \includegraphics[scale=1.3]{jth-logo.eps}\fi} % <-- Ändring IvTj
\chead{\ifnum\thepage=1 {} \else \Tr{Cheatsheet Calculus}{Formelblad Envariabelanalys}\fi}
\rhead{\ifnum\thepage=1 \textbf{\Tr{CHEATSHEET CALCULUS}{FORMELBLAD ENVARIABELANALYS}}
               \else sid. \thepage{} av \pageref{LastPageNo}  \fi}

\lfoot{\ifnum\thepage=1\small\today\fi}
\cfoot{}
\rfoot{\ifnum\thepage=1\small File: \texttt{\jobname.pdf}\fi}
\parindent 0 cm
\normalsize
%% \begin{flushright}\ifnum\thepage=1
%% %{\today}
%% \fi\end{flushright}



\let\iff\Leftrightarrow % <-- ändring IvTj
\let\ob\overline        % <-- ändring IvTj
\newcommand\Rone{\mathbb{R}}
\def\EndRow{\\*[3pt]}

%\psfig{figure=CTH-logo-4cm.eps%,width=19mm}
 %\psfig{figure=k-ing.eps,width=29mm}} <-- ändring IvTj
%>>>

\vspace{-1.5cm}

\subsection*{\Tr{The square-, conjugate- and binomial formulas}{Kvadrerings-, konjugat- och binomialformlerna}}%<<<


\begin{tabular}[m]{|c|c|}
  \hline
  \Tr{The binomial coeffients}{Binomiala koefficienter:}
    $\binom nk = \frac{n!}{k!(n-k)!}$
&
$n! = \begin{cases}
                \hfil1,& n=0\\
                1\cdot2\cdot3\cdots\cdot n,& n=1,2,\dots
                \end{cases}$
  \\\hline
\end{tabular}

\medskip

\begin{tabular}[m]{|c|c|c|}
      \hline
      &&\\*[-11pt]
    $(a + b)^2 = a^2 + 2ab+b^2$
    &
    $(a+b)(a-b) = a^2-b^2$
    &
    $(a+b)^n =
    \sum_{k=0}^n\binom nk a^{n-k}b^k$\\
      \hline
\end{tabular}
%
% Pascals triangel:
% \begin{verbatim}
%       1
%     1 2 1
%    1 3 3 1
%   1 4 6 4 1
% \end{verbatim}
%>>>

\subsection*{\Tr{Powers, exponentials and logarithms}{Potenser, exponentialfunktioner, logaritmer}}%<<<


$\begin{array}{|l|l|l|l|l|}\hline
  &&&&\\*[-11pt]
  a^0=1 &a^xa^y=a^{x+y} &
  %\frac{a^x}{a^y}=a^{x-y}&(a^x)^y=a^{xy} <-- ändring IvTj
  a^x/a^y=a^{x-y}&(a^x)^y=a^{xy}
  & (ab)^x=a^xb^x
\EndRow
  \ln(1)=0 &\ln(xy)=\ln(x)+\ln(y) &\ln\big(\frac{x}{y}\big)
  =\ln(x)-\ln(y) &
  \ln(\frac1x) = -\ln(x) &
  \ln(x^y)=y\ln(x)
\EndRow
  e^{\ln(x)}=x & \ln(e^x)=x & \log_a(x)
  =\frac{\ln(x)}{\ln(a)} & a^x=e^{x\ln(a)} &
  (\log_ba)(\log_ab) = 1
\EndRow
\hline\end{array}$%>>>

\subsubsection*{\Tr{The trigonometric functions}{De trigonometriska funktionerna}}%<<<

% Standarda värden för sin/cos/tan + symmetrirelationer t höger %<<<
{% Local scope def's
\let\F\frac
\newcommand\CC[1]{#1^\circ}
\def\vPad{&&&&&\\*[-10pt]}
\def\myEnd{\\\hline\vPad}
\newcommand\tW{{\sqrt2}}
\newcommand\tH{{\sqrt3}}
$
\begin{array}{|c|c|c|c|c|c|}
\hline\vPad
\mbox{(Std.~\Tr{angles}{vinklar})}^\circ
  & \CC{0} &\CC{30} &\CC{45} &\CC{60} &\CC{90}   % &\CC{120} &\CC{135} &\CC{150} &\CC{180}
 \myEnd
 x \mbox{ (\textbf{rad})}
        & 0     & \pi/6 & \pi/4 &  \pi/3 & \pi/2 % & 2\pi/3 & 3\pi/4 & 5\pi/6 &\pi
 \\\hline\hline\vPad
 \cos(x)&   1   & \tH/2 & 1/\tW &  1/2  &   0    % & -1/2   & -1/\tW & -\tH/2 & -1
 \myEnd
 \sin(x)&   0   &   1/2 & 1/\tW & \tH/2 &   1    % & \tH/2  &  1/\tW &   1/2  &  0
 \myEnd
 \tan(x)&   0   & 1/\tH &   1   & \tH   &
                                    \mbox{ej def}
                                                 % &-\tH  &    -1  & -1/\tH &  0
 \\\hline
\end{array}
\quad
\begin{array}[m]{|l|l|}
\hline
  \cos(-x) = \cos(x)    & \cos(x\pm\pi) = -\cos(x) \\
  \sin(-x) = -\sin(x)   & \sin(x\pm\pi) = -\sin(x) \\
\hline
\hline
  \sin(\pi-x) = \sin(x) &  \cos(x+2\pi) = \cos(x) \\
  \cos(\pi-x) = -\cos(x)&  \sin(x+2\pi) = \sin(x) \\
\hline
\hline
 \tan(-x) = -\tan(x)     & \tan(x\pm\pi) = \tan(x) \\
\hline
\end{array}
$
}%>>>

% Trig.~ekv: %<<<
\medskip
\textbf{Trig.~\Tr{eqns}{ekv}}:
\raise8pt\hbox{
$
\begin{array}[t]{|c|c|c|}
  \hline
%  \multicolumn{3}{|c|}{\mbox{Trigonometriska ekvationer}}\\ % <-- Ändring IvTj
%  \cline{1-3}
  \sin(x)=c\,, \;|c|\leq 1,      \mbox{ löses av:} &
  \cos(x)=c\,, \;|c|\leq 1,      \mbox{ löses av:} &
  \tan(x)=c\,, \;c\in\mathbb{R}, \mbox{ löses av:}
\\[2pt]
x=
\begin{cases}
     x_0+2n\pi,   & n\in\mathbb{Z}\cr
    \pi-x_0+2k\pi,& k\in\mathbb{Z}
\end{cases}
&
x=\pm x_0+n2\pi\,,\;n\in\mathbb{Z}
&x=x_0+n\pi\,,\;n\in\mathbb{Z}
\\[11pt]
\hline
\end{array}
$} % raised hbox
%>>>

% Trigrormlerna%<<<
\medskip
$\begin{array}{|l|l|}\hline &\\*[-11pt] % <-- ändring IvTj
 %\sin(-x)=-\sin(x) &\cos(-x)=\cos(x) \EndRow
 \cos^2(x)+\sin^2(x)=1  &
 \cos(2x)=\cos^2(x)-\sin^2(x) \EndRow %% <-- ändring IvTj
 \sin(2x)=2\sin(x)\cos(x)&\cos(2x)=2\cos^2(x)-1=1-2\sin^2(x) \EndRow
 \sin^2(x) = (1-\cos(2x))/2 &
 \cos^2(x) = (1+\cos(2x))/2 \EndRow % <-- ändring IvTj
 \sin(x)=\sin(\pi-x)=\cos(\frac{\pi}{2}-x) &
 \cos(x)=\sin(\frac{\pi}{2}-x)=-\cos(\pi-x)\EndRow
 % \sin(x\pm\pi) = -\sin(x) & \cos(x\pm\pi) = -\cos(x) \EndRow % <-- ändring IvTj
 \sin(x\pm y)=\sin(x)\cos(y)\pm\sin(y)\cos(x) &
 \cos(x\pm y)=\cos(x)\cos(y)\mp\sin(x)\sin(y)\EndRow
% \sin(2x)=2\sin(x)\cos(x)&\cos(2x)=2\cos^2(x)-1=1-2\sin^2 x \EndRow
 \sin(x)\sin(y)=\frac{1}{2}(\cos(x-y)-\cos(x+y)) &
 \sin(x)\cos(y)=\frac{1}{2}(\sin(x-y)+\sin(x+y)) \EndRow
 \cos(x)\cos(y)=\frac{1}{2}(\cos(x-y)+\cos(x+y)) &
 \tan(x\pm y)=\dfrac{\tan(x)\pm\tan(y)}{1\mp\tan(x)\tan(y)}\\*[8pt]
 \cline{1-2}\multicolumn{2}{|c|}{     % <-- ändring IvTj
   a\cos x+b\sin x=A\sin(x+\varphi) = A\cos(x-\psi),\;
              A=\sqrt{a^2+b^2}
     \mbox{ \ och \ }
              \left\{ \begin{array}{l}
              \sin\varphi=\cos\psi=a/A \\
              \cos\varphi=\sin\psi=b/A
\end{array}\right.}
 \EndRow\hline\end{array}$ %>>>

%>>>

\subsubsection*{\Tr{The inverse trigonometric functions}{De inversa trigonometriska funktionerna}}%<<<

$\begin{array}{|lll|}\hline && \\*[-9pt]
  y=\sin(x),\hspace{2 mm} x\in[-\frac{\pi}{2},\frac{\pi}{2}]&
  \iff &
  x=\arcsin(y),\hspace{2 mm} y\in[-1,1] \EndRow
  y=\cos(x),\hspace{2 mm} x\in[0,\pi]&
  \iff &
  x=\arccos(y),\hspace{2 mm} y\in[-1,1] \EndRow
  y=\tan(x),\hspace{2 mm} x\in(-\frac{\pi}{2},\frac{\pi}{2})&
  \iff &
  x=\arctan(y), \hspace{2 mm} y\in\mathbb{R}\\*[6pt]\hline
\end{array}$

\vspace{1 mm}

$\begin{array}{|l|l|l|}\hline & & \\*[-11pt]
\arcsin(-x)=-\arcsin(x) & \arccos(-x)=\pi-\arccos(x) &
  \arctan(x)+\arctan(\tfrac1x)=\frac\pi2, \; x>0 \EndRow % <-- ändring IvTj
\arctan(-x)=-\arctan(x) &\arccos(x)=\frac{\pi}{2}-\arcsin(x) &
  \arctan(x)+\arctan(\tfrac1x)=-\frac\pi2, \; x < 0 \\*[6pt] % <-- ändring IvTj
\hline
\end{array}$%>>>

\subsection*{\Tr{The complex numbers}{Komplexa tal}}%<<<

$  % <-- Ändring IvTj
\begin{array}{|l|l|}
  \hline & \\*[-8pt]
  z=x+iy    =|z|\,e^{ i\varphi}=|z|(\cos\varphi+i\sin\varphi) &
 x=\text{Re}(z)\in\Rone,\;
 y=\text{Im}(z)\in\Rone,\;
 |z| = |\ob z| = \sqrt{x^2+y^2}
  \EndRow
  \ob z=x-iy=|z|\,e^{-i\varphi}=|z|(\cos\varphi-i\sin\varphi) &
  \varphi = \arg (z),
  \;
  %\text{ där }
  \cos\varphi = x/|z|,
  %\text{ och }
  \;
  \sin\varphi = y/|z|  ;
  \;
  \arg(\ob z) = -\varphi
  \EndRow
  \ob{\ob z} = z ;
  \;\,
  \ob{zw} = \ob z\,\,\ob w ;
  \;\,
  \ob{z / w} = \ob z/\ob w ;
  \;\,
  z\ob z = |z|^2
  &
  |zw| = |z|\,|w| ;
  \quad
  |z/w| = |z|/|w| ;
  \quad
  |z+w| \le |z| + |w|
  \EndRow \cline{1-2} \multicolumn{2}{|l|}{}\\*[-9pt]
  \multicolumn{2}{|c|}{
    \text{\Tr{The Euler's formulae}{Eulers formler}: \ }
  \left\{\begin{matrix}
  e^{i\varphi}   =  \cos \varphi + i \sin \varphi\\
  e^{-i\varphi}  =  \cos \varphi - i \sin \varphi\\
  \end{matrix}\right.
  \hfil
  \left\{\begin{matrix}
    \cos\varphi = (e^{i\varphi}+e^{-i\varphi})/2\\
    \sin\varphi = (e^{i\varphi}-e^{-i\varphi})/2i\\
  \end{matrix}\right.
  }
  \\*[9pt]
  \cline{1-2}
  \multicolumn{2}{|l|}{}\\*[-7pt]
  \multicolumn{2}{|c|}{\text{\Tr{The de Moivre's formula }{de Moivres formel}: \ }
  (\cos\varphi+i\sin\varphi)^n
  =(e^{i\varphi})^n
  =e^{in\varphi}
  =\cos(n\varphi)+i\sin(n\varphi)
  }
  \EndRow
  \hline
\end{array}
$

%% Ett komplext tal $z$ har formen $z=x+iy$ där $x,y\in\mathbb{R}$ och $i^2=-1$.
%%
%% $\text{Re}(z)=x$ --- realdelen av $z$; $\text{Im}(z)=y=$imaginärdelen av $z$
%%
%% $\vert z\vert=\sqrt{x^2+y^2}$
%%
%% $z=x+iy\Rightarrow \text{ komplexkonjugatet }\overline{z}=x-iy$
%%
%% \vspace{5 mm}
%%
%% $\begin{array}{|l|l|}
%% \hline \overline{\overline{z}}=z &\vert z\vert\geq 0 \text{ för alla }z \text{ med likhet }\Le z=0
%% \\ & \\ \overline{zw}=\overline{z}\cdot\overline{w}&\vert zw\vert=\vert z\vert\cdot\vert w\vert
%% \\ & \\\overline{\left(\frac{z}{w}\right)}=\frac{\overline{z}}{\overline{w}}&\left\vert\frac{z}{w}\right\vert=\frac{\vert z\vert}{\vert w\vert}
%% \\ & \\z\overline{z}=\vert z\vert^2&\vert z+w\vert\leq\vert z\vert+\vert w\vert
%% \\\hline
%% \end{array}$
%%
%%\vspace{1 cm}
%%
%%Polär form: $$z=x+iy=r(\cos\theta+i\sin\theta)=re^{i\theta}\Le\left\{\begin{array}{l}r^2=x^2+y^2=\vert z\vert^2\\r\cos\theta=x\\r\sin\theta=y\end{array}\right.$$
%%
%%De Moivres formel: $$\text{Om } n\in\mathbb{Z} \text{ så gäller att } \left(e^{i\theta}\right)^n=e^{in\theta} \;\text{ eller, } \left(\cos \theta+i\sin\theta\right)^n=\cos n\theta+i\sin n\theta\,.$$
%>>>

\subsection*{\Tr{The polynomials and their roots}{Polynom och deras nollställen}} %<<<
%% <-- änding IvTj
\Tr{Polynomial of degree}{Polynom av grad} $n$:\, $P_n(z)
   =a_nz^n+a_{n-1}z^{n-1}+\dots+a_1z+a_0
   =\sum_{k=0}^n a_kz^k$,
   $a_n\neq 0$

 \Tr{The factor thm}{Faktorsatsen}: $P_n(z_0)=0 \iff P_n(z) = (z-z_0)Q_{n-1}(z)$,
 \Tr{where}{där} $Q_{n-1}$
 \Tr{is a polynomial of degree}{är polynom av grad} $n-1$

\bigskip

\Tr{Polynomials with real coeffients}{Polynom med reella koefficienter}:
\Tr{If}{Om}
$P_n(z_0)=0$
\Tr{and}{och}
$\text{Im}(z_0)\neq0$,
\Tr{then}{då är}
$P_n(\ob z_0)=0$,
\Tr{and hence}{och därmed}
   $$
   P_n(z)
    =(z-z_0)(z-\ob z_0)Q_{n-2}(z)
    =(z^2-2\text{Re}(z_0)z +|z_0|^2)Q_{n-2}(z)
   $$
  \Tr{Completing squares in second-degree polynomials}
   {Kvadratkomplettering och andragradsekvationer}:
\[
z^2+pz+q
  =\left(z+\frac{p}{2}\right)^2-\frac{p^2}{4}+q
  =(z-z_1)(z-z_2)
  ,\quad
  z_{1,2}=-\frac{p}{2}\pm\sqrt{\frac{p^2}{4}-q}.
  %%\; \mbox{ (med } \sqrt{-1} = i).
\]

\vspace{-1em}
% Andragradsekvationer:
% % \subsubsection*{Andragradsekvationer}
% $x^2+px+q=(x-x_1)(x-x_2)=0\iff
% x_{1,2}
% =\begin{cases}
%   -\frac{p}{2}\pm\sqrt{\frac{p^2}{4}-q},&\frac{p^2}{4}-q\geq 0
%   \\*[9pt]
%   -\frac{p}{2}\pm i\sqrt{q-\frac{p^2}{4}},&\frac{p^2}{4}-q< 0
% \end{cases}
% $
%>>>


\vspace{-5pt}
\subsection*{\Tr{Derivatives}{Derivator}}%<<<

$$Df(x)=\frac{df(x)}{dx}=f'(x)
=\lim_{h\rightarrow 0}\frac{f(x+h)-f(x)}{h};
\qquad D=\frac d{dx}
$$


% Tangent och Normal <-- Ändring IvTj
     \begin{tabular}[m]{|l|}
        \hline  \\*[-11pt]
      \Tr{Tangent line to the curve}
        {Tangentlinje till kurvan}
        $y=f(x)$
        \Tr{at the point}{i punkten}
        $(a,f(a))$:
        \; $y=f(a)+f'(a)(x-a)$\\
       \hline  \\*[-10pt]
       \Tr{Normal line to the curve}{Normallinje till kurvan} $y=f(x)$
        \Tr{at the point}{i punkten}
      $(a,f(a))$:
       $\begin{cases}
         y=f(a)-\dfrac1{f'(a)}(x-a),&
         f'(a)\neq0\\
         \hfill x=a\hfill,&  f'(a)=0
         \end{cases}$ \\*[8pt]
       \hline
     \end{tabular}

\medskip

$\begin{array}{|c|c|c|}\hline & &\\*[-8pt]
(\alpha f(x)+\beta g(x))'=\alpha f'(x)+\beta g'(x)&
\D f(g(x))=f'(g(x))g'(x)
&\Bigl(\dfrac{f}{g}\Bigr)'
 =\dfrac{f'g-fg'}{g^2}
\EndRow
(fg)'=f'g+fg' & \D e^x = e^x &
(f^{-1})'(b) = \dfrac1{f'(a)}\,\mbox{ om }
\begin{cases}
f(a)=b\\f'(a)\neq0    %% <-- ändring IvTj
\end{cases}
\EndRow
\D x^r=rx^{r-1}
&   \D\ln(|x|)=\dfrac{1}{x} &\D a^x=a^x\ln(a)
\EndRow
 \D\sin(x)=\cos(x) & \D\cos(x)=-\sin(x) &
  \D\tan(x)=\dfrac{1}{\cos^2(x)}=1+\tan^2(x)
\EndRow
\D\arcsin(x)=\dfrac{1}{\sqrt{1-x^2}} &
\D\arccos(x)=-\dfrac{1}{\sqrt{1-x^2}} &
\D\arctan(x)=\dfrac{1}{1+x^2}
\\*[9pt]
\hline
\end{array}$

\medskip

\fbox{\parbox{105ex}
{\textbf{L'Hôpital\Tr{'s rools}{sregler}}:
\Tr{If}{Om}
$\lim_{x\to a}\dfrac{f(x)}{g(x)}=\Bigl[\dfrac00\Bigl]$
\Tr{or}{eller}
$\lim_{x\to a}\dfrac{f(x)}{g(x)}=\Bigl[\dfrac\infty\infty\Bigl]$,
\Tr{both $f$ and $g$ have continuous derivatives in some neighbourhood of $a$, then the existence of}
{både $f$ och $g$ har kontinuerliga derivator i en omgivning av $a$, då ett existerande gränsvärde}
$$
\lim_{x\to a}\dfrac{f'(x)}{g'(x)}=A
\quad\Rightarrow\quad
 \lim_{x\to a}\dfrac{f(x)}{g(x)}
=\lim_{x\to a}\dfrac{f'(x)}{g'(x)}=A.
$$
\Tr{The rules of L'Hôpital can be used recursively and are in vigour also when}
{L'Hôpitalsreglerna kan användas rekursivt och är uppfyllda även då}
$a=\infty$ \Tr{or}{eller}
$a=-\infty$.
}}


\medskip

\fbox{\parbox{104ex}{\textbf{\Tr{Oblique asymptotes}{Sneda asymptoter}}:
\Tr{The line}{Linjen}
$y=kx+m$
\Tr{is an oblique asymptote to the function}
{är en sned asymptot till funktionen}
$f(x)$ \Tr{for}{då} $x\to\infty$ \Tr{if}{då}
$$
\lim_{x\to\infty}(f(x)-kx-m) = 0.
\quad\mbox{\Tr{Then}{Då är} \ }
  k=\lim_{x\to\infty} \frac{f(x)}{x}
  \mbox{ \ \Tr{and}{och} \ }
  m=\lim_{x\to\infty}(f(x)-kx).
$$
}
} % fbox

%>>>

\subsection*{\Tr{Indefinite integrals}{Obestämda integraler}}%<<<

%% 2007-11-30: Bytt dx --> \,dx (Tjavdar)

$
\begin{array}{|l|l|}\hline   & \\*[-5pt] % <-- ändring IvTj

 \mbox{Beteckning: }F'(x) = f(x) &
 \int (\alpha f(x)+\beta g(x))\,dx
 =\alpha\int f(x)\,dx
 +\beta\int g(x)\,dx \\
 & \\
 \int f(ax)\,dx = \frac1a F(ax) + C &
 \int f(x)g(x)\,dx=F(x)g(x)-\int F(x)g'(x)\,dx
 \\*[9pt]
 \cline{1-2}
 \multicolumn{2}{|l|}{}\\*[-6pt]
 \multicolumn{2}{|l|}{
   \mbox{\Tr{Change of variables}{Variabelbyte}: }
\int f(g(x))g'(x)\,dx=
\genfrac{\{}{\}}{0pt}{}{t=g(x)}{dt=g'(x)\,dx}
  = \int f(t)dt = F(g(x)) + C
  }
\\*[12pt]\hline
\end{array}
$

\medskip

$\begin{array}{|l|l|}\hline   & \\*[-5pt]
  %\int_a^b f(x)g(x)\,dx=\Big[F(x)g(x)\Big]_a^b-\int_a^b F(x)g'(x)\,dx &  F \text{ är en primitiv funktion till }f \\ & \\ \int_a^bf(x)\,dx=F(b)-F(a) &\ F \text{ är en primitiv funktion till }f  \\
\int x^a\,dx=\dfrac{x^{a+1}}{a+1}+C, \hspace{2 mm} a\neq -1 & \int\frac{1}{x}\,dx=\ln\vert x\vert +C \\ & \\ \int e^x\,dx=e^x+C &\int a^x\,dx=\frac{a^x}{\ln(a)}+C, \hspace{2 mm} 0<a\neq 1 \\ & \\ \int \sin(x)\,dx=-\cos(x)+C &\int\cos(x)\,dx=\sin(x)+C \\ & \\ \int \frac{1}{\cos^2(x)}\,dx=\tan(x)+C &\int\frac{1}{\sin^2(x)}\,dx=-\frac{1}{\tan(x)}+C \\ & \\ \int\frac{1}{\sqrt{a^2-x^2}}\,dx=\arcsin\Bigl(\frac xa\Bigr)+C,\;\, a>0 &\int\frac{1}{x^2+a^2}\,dx=\frac{1}{a}\arctan\left(\frac{x}{a}\right)+C\\ &\\ \int\frac{1}{\sqrt{x^2+a}}\,dx=\ln\vert x+\sqrt{x^2+a}\vert+C,\hspace{2 mm}a\neq 0 &\int\frac{f'(x)}{f(x)}\,dx=\ln\vert f(x)\vert +C
\\*[10pt]
\cline{1-2}
\multicolumn{2}{|l|}{}\\*[-7pt]
\multicolumn{2}{|l|}{

\int\frac{(ax+b)\,dx}{(x-c)(x-d)}
=\{\text{\Tr{partial fractions}{partialbråksuppdelning}}\}=
\begin{cases}

A\int\frac{dx}{x-c} +B\int\frac{dx}{x-d} &, c\neq d \\*[10pt]

A\int\frac{dx}{x-c} +B\int\frac{dx}{(x-c)^2} &, c=d
\end{cases}
}\\
%%\multicolumn{2}{l}{}\\*[-7pt]
\hline
\end{array}
$
%>>>

\subsection*{\Tr{Definite integrals}{Bestämda integraler}}%<<<

\Tr{The fundamental theorem of Calculus}{Analysens huvudsats}:
\Tr{If}{om} $f$
\Tr{is continuous, then}{är kontinuerlig, då är} \/
$\frac d{dx}\Bigl(\int_a^x f(t)\,dt\Bigr)=f(x)$.

%% Insättningsformeln:  $\int_a^b f(x)\,dx
%%                      = \bigl[F(x)\bigr]_{a}^b = F(b)-F(a)$, där
%% $F(x)$ är en primitiv funktion till $f$

\medskip

$
\begin{array}{|l|l|}\hline   & \\*[-5pt] % <-- ändring IvTj

  \mbox{\Tr{Notation}{Beteckning}: }F'(x) = f(x) &
 \int_a^b (\alpha f(x)+\beta g(x))\,dx
 =\alpha\int_a^b f(x)\,dx
 +\beta\int_a^b g(x)\,dx \\
 & \\
 \int_a^b f(x)\,dx = \bigl[F(x)\bigr]_{a}^b = F(b)-F(a)&
 \int_a^b f(x)\,dx = \int_a^c f(x)\,dx + \int_c^b f(x)\,dx\\
 & \\
 \int_a^b f(x)\,dx = -\int_b^a f(x)\,dx &
 \int_a^b f(x)g(x)\,dx=\bigl[F(x)g(x)\bigr]_{a}^b-\int_a^b F(x)g'(x)\,dx
 \\*[9pt]
 \cline{1-2}
 \multicolumn{2}{|l|}{}\\*[-6pt]
 \multicolumn{2}{|l|}{
  \mbox{\Tr{Change of variables}{Variabelbyte}: }
\int_a^b f(g(x))g'(x)\,dx=
\genfrac{\{}{\}}{0pt}{}{t=g(x)}{dt=g'(x)\,dx}
= \int_{g(a)}^{g(b)} f(t)dt = \bigl[F(t)\bigr]_{g(a)}^{g(b)}
  }
\\*[12pt]\hline
\end{array}
$

\medskip

\textbf{\Tr{Improper integrals}{Generaliserade integraler}:}\\
\begin{tabular}{|l|l|}
\hline & \\*[-9pt]
$\int_a^\infty f(x)\,dx
 \stackrel{\mbox{\tiny def}}{=}
 \lim_{N\to\infty}\int_a^N f(x)\,dx$
&
$\int_a^b f(x)\,dx
 \stackrel{\mbox{\tiny def}}{=}
 \lim_{\varepsilon\to0^+}\int_{a+\varepsilon}^b f(x)\,dx$,
 \Tr{if $f$ is not finite at}{om $f$ ej begränsad i}
 $a$\\*[8pt]
\hline
\end{tabular}
%>>>

\subsection*{\Tr{Some standard limits}{Standardgränsvärden}:}%<<<

\def\EndRow{\\*[6pt]}
$\begin{array}{|ll|ll|l|}\hline& &&&\\*[-8pt]
  \lim_{x\rightarrow\infty}\frac{1}{x^p}=0, \hspace{2 mm} p>0 & &\lim_{x\rightarrow\infty}x^p=\infty,\hspace{2 mm} p>0&& \lim_{x\to\pm\infty}\arctan(x)=\pm\dfrac{\pi}{2}
\EndRow
  \lim_{x\rightarrow\infty}a^x=0,\hspace{2 mm} 0<a<1 & &\lim_{x\rightarrow\infty}a^x=\infty, \hspace{2 mm}a>1 &&\lim_{x\to0}\frac{\sin(x)}x=1
\EndRow
\lim_{x\rightarrow\infty}(1+\frac{t}{x})^x=e^t & &\lim_{x\rightarrow 0}\frac{\ln(1+x)}{x}=1 &&\lim_{x\rightarrow 0}\frac{1-\cos(x)}{x^2}=\frac12
\EndRow \lim_{x\rightarrow \infty}\frac{x^p}{a^x}=0, \hspace{2 mm} a>1 & &\lim_{x\rightarrow\infty}\frac{\ln(x)}{x^p}=0, \hspace{2 mm}p>0&&\lim_{x\rightarrow 0}\frac{\arcsin(x)}{x}=1
\EndRow
\lim_{x\rightarrow0^+}x^p\ln(x)=0, \hspace{2 mm}p>0 \hspace{10 mm}& & \lim_{x\rightarrow 0}\frac{e^x-1}{x}=1 && \lim_{x\rightarrow 0}\frac{\arctan(x)}{x}=1
\\*[9pt]\hline
\end{array}$
%>>>

\Tr{% No ODEs in En-version
}{
\subsection*{Några resultat om differentialekvationer}%<<<

\textbf{1:a ordningens linjär ODE:} % <-- Ändring IvTj
\fbox{$y'+g(x)y=f(x) \iff \bigl(y(x)e^{G(x)}\bigr)' = f(x)\,e^{G(x)}$, där $G'(x)=g(x)$}

Lösningsmetod: Multiplicera ekvationen med den integrerande faktorn $e^{G(x)}$
där $G(x)$ är en primitiv funktion till $g(x)$. Integrera sedan ekvationen som
då erhålls.

\vspace{5 mm}

\textbf{1:a ordningens separabel ODE:} \fbox{$g(y)y'=h(x)$}

Lösningsmetod: Separera variablerna enligt $g(y)dy=h(x)dx$ och integrera sedan
ekvationen.

\vspace{5 mm}

\textbf{2:a ordningens linjära ODE med konstanta koefficienter:}
\fbox{$y''(x)+ay'(x)+by(x)=f(x)$}

Lösningsmetod: $y=y_h+y_p$ där $y_h$ är den allmäna lösningen till motsvarande
homogena ekvation $y''+ay'+by=0$ (se satsen nedan) och $y_p$ är en
partikulärlösning till ekvationen $y''+ay'+by=f(x)$.

\begin{theorem}
  Låt $a,b\in\mathbb{R}$ och betrakta den homogena differentialekvationen
  $y''+ay'+by=0.$ Antag att $r_1$ och $r_2$ är rötterna till den
  karakteristiska ekvationen $r^2+ar+b=0.$ Då gäller
  \begin{itemize}

  \item[i)] Om $r_1, r_2\in\mathbb{R}$ och $r_1\neq r_2$ så kan varje  lösning
    skrivas $$y(x)=C_1e^{r_1x}+C_2e^{r_2x},$$där $C_1$ och $C_2$ är godtyckliga
    konstanter.

\item[ii)] Om $r_1=r_2$ så kan varje lösning skrivas
  $$y(x)=(C_1+C_2x)e^{r_1x},$$där $C_1$ och $C_2$ är godtyckliga konstanter.

\item[iii)] Om $r_1=\alpha+i\beta,$ $r_2=\alpha-i\beta,$ där $\beta\neq 0$ så
  kan varje lösning skrivas $$y(x)=e^{\alpha x}(C_1\cos(\beta x)+C_2\sin(\beta
  x)),$$där $C_1$ och $C_2$ är godtyckliga konstanter.

\end{itemize}
\end{theorem}

\textbf{Ansatser till partikulärlösning för olika högerled}:
\fbox{$y''+ay'+by=f(x)$}


\medskip

$\begin{array}{|c|c|l|}\hline %% <-- Ändring IvTj
  \text{\textbf {Typ}}&
  \text{\bf Högerled} \ f(x)&
  \text{\bf Ansats för partikulära lösningen $y_p(x)$}\\
  \hline
  \text{I}&f(x)=P_n(x) & y_p(x) =
  \begin{cases}
    \hfill Q_n(x),       & b\neq0\\
    \hfill x\, Q_n(x),   & b=0, \; a\neq0\\
    \hfill x^2\, Q_n(x), & a=b=0\\
  \end{cases}
  \\
  \hline & & \\*[-9pt]
  \text{II}&f(x)=P_n(x)\,e^{kx}&y_p=z(x)e^{kx}\,,\;
   \text{ insättning ger ekvation av typ I}\EndRow
  \hline
  \text{III}&f(x)=\alpha\cos(kx)+\beta\sin(kx)
  &y_p(x) =
  \begin{cases}
    \hfill A\cos(kx)+B\sin(kx),                 & r_{1,2}\neq\pm ik  \\
    \hfill x\,\left(A\cos(kx)+B\sin(kx)\right), & r_{1,2}=\pm ik
  \end{cases} \\
  \hline\multicolumn{3}{|c|}{
  \text{\small Beteckningar: $P_n$, $Q_n$ -- polynom av grad $n$; $r_{1,2}$
   -- rötterna till den karakteristiska ekvationen}
  } \\
\hline
\end{array}$
%>>>
}

\label{LastPageNo}

\end{document}
