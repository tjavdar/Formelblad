% Changes:
% Ver. 0.91 (Fri Mar 16 14:38:00 CET 2007) isolatin --> utf-8
% Ver. 0.9  (Tue Aug  8 16:01:48 CEST 2006) \voffset-0.9in --> -1.1in
% Ver. 0.8  (Thu Apr 27 21:41:58 CEST 2006) \bdy --> \bdX i för att ha
%                                  samma beteckningar som OH-bilderna
%
\documentclass{article}
\voffset-1.1in
\hoffset-.9in
\usepackage[swedish]{babel}
\usepackage[utf8]{inputenc}
\usepackage{amssymb}
\pagestyle{empty}
\usepackage{graphicx} % \includegraphics does both eps && pdf!
\textwidth=38pc\textheight=734pt\parindent=0pt

%% restore good old TeX \eqalign:
\makeatletter
\def\eqalign#1{\null\,\vcenter{\openup\jot\m@th
  \ialign{\strut\hfil$\displaystyle{##}$&$\displaystyle{{}##}$\hfil
      \crcr#1\crcr}}\,}
\makeatother

\def\ppmatrix{\protect\pmatrix}
\def\pcases{\protect\cases}
\newcommand\pdet[1]{\protect\left|\protect\matrix{#1}\protect\right|}
\newcommand\Ordo{\mathcal O}
\let\ergo\Longrightarrow
\let\vaxer\nearrow
\let\avtar\searrow
\newcommand\conj[1]{{\overline #1}}
\let\ob\conj
\newcommand\binom[2]{{#1\choose #2}}

%% the boldface guys %%

\newcommand\bdF{\mathbf F}
\newcommand\bdG{\mathbf G}
\newcommand\bdN{\mathbf N}
\newcommand\bdP{\mathbf P}
\newcommand\bdQ{\mathbf Q}

\newcommand\bda{\mathbf a}
\newcommand\bdb{\mathbf b}
\newcommand\bdc{\mathbf c}
\newcommand\bde{\mathbf e}
\newcommand\bdf{\mathbf f}
\newcommand\bdg{\mathbf g}
\newcommand\bdi{\mathbf i}
\newcommand\bdj{\mathbf j}
\newcommand\bdk{\mathbf k}
\newcommand\bdm{\mathbf m}
\newcommand\bdn{\mathbf n}
\newcommand\bdp{\mathbf p}
\newcommand\bdq{\mathbf q}
\newcommand\bdr{\mathbf r}
\newcommand\bdu{\mathbf u}
\newcommand\bdv{\mathbf v}
\newcommand\bdx{\mathbf x}
\newcommand\bdy{\mathbf y}
\newcommand\bdw{\mathbf w}
\newcommand\bdz{\mathbf z}
\newcommand\bdzero{\mathbf 0}


\newcommand\bdX{\mathbf X}
\newcommand\bdZ{\mathbf Z}

% - sets -
\def\Rone{{\mathbb R}}

% - operators -

\def\ddn{\partial_n}
\def\dist{\mathop{\mathrm{dist}}}
\def\divergence{\mathop{\mathrm{div}}}
\def\rot{\mathop{\mathrm{rot}}}
\def\norm#1{{\Vert #1\Vert}}

% - förkortnngar

\newcommand\mha{med hjälp av }
\newcommand\bis{^{\prime\prime}}
\newcommand\triss{^{'''}}
\newcommand\PartInt[2]{\left\lceil\matrix{\mbox{\small #1}\cr
                                       \mbox{\small #2}}\right\rceil}

%adjust row height in tables #1 - overall height #2 - depth:
\newcommand\TblHeight[2]{\lower#2em\vbox to#1em{\hsize=0pt}}

%--------- End Preambule -------------

\everymath{\displaystyle}

\begin{document}

\section*{Linjära system av ODE
med konstanta koefficienter}

Konventioner och beteckningar:
\itemsep=1pt
\begin{itemize}
\item $\bdX(t)=(x_1(t),\cdots,x_n(t))^T$ --- kolumnvektorn av de
  obekanta funktioner;
\item  $A=[a_{ij}]_{i,j=1}^n$ -- $n\times n$
  matris vars element är reella konstanter;

\item Egenvärden $\lambda_k$ och egenvektorer $\bdv_k$ till $A$, dvs
$A\bdv_k=\lambda_k\bdv_k$,\, $k=1,2,\ldots,p$,\, $p\le n$.

\item Om $A$ är diagonaliserbar: $A=TDT^{-1}$ där
$T=\pmatrix{
| &       & | \cr
\bdv_1 & \cdots & \bdv_n\cr
| &       & | \cr
}$ och
$D=\pmatrix{
\lambda_1 &  0 & \cdots    & 0 \cr
0 & \lambda_1 &  \cdots    & 0 \cr
\vdots&\vdots &  \ddots    & \vdots \cr
0 &   0       &  \cdots    & \lambda_n\cr
}$.
%  $T=[\bdv_1,\ldots,\bdv_n]$ och
%$D=\mbox{diag}[\lambda_1,\ldots,\lambda_n]$.

\end{itemize}

\subsection*{Första ordningens homogena system: $\bdX'=A\bdX$}

Till varje homogent system går att hitta en sk fundamental upsättning (FU)
av linjärt oberoende lösningar $\{\phi_1(t),\ldots,\phi_n(t)\}$ så att
den allmänna lösningen $\bdX_h(t)$ (dvs varje lösning) till $\bdX'=A\bdX$
är en linjär\-kom\-bi\-na\-tion av funktionerna i den FU
$$
\bdX_h(t) = \sum_{k=1}^nC_k\phi_k(t)=C_1\phi_1(t)+\cdots+C_n\phi_n(t),
$$
där $C_1,...,C_n$ är godtyckliga konstanter. Följande fall kan uppstå:

\bigskip
\textbf{Fall 1}: $A$ är diagonaliserbar och alla dess egenvärden är
reella.  Då gäller:
$$
\bdX'=A\bdX\iff
\bdX'=TDT^{-1}\bdX\iff
T^{-1}\bdX'=DT^{-1}\bdX\iff
\bdZ'=D\bdZ
\mbox{ med } \bdZ=T^{-1}\bdX \iff \bdX=T\bdZ.
$$
Det sista systemet består av $n$-st oberoende ekvationer:
$$
\bdZ'=D\bdZ\iff
\cases{
     z_1'=\lambda_1 z_1\cr
     \hfil\vdots\cr
     z_n'=\lambda_n z_n\cr
       }
\ergo \,
\cases{
     z_1'=C_1e^{\lambda_1t}\cr
     \hfil\vdots\cr
     z_n'=C_ne^{\lambda_nt}\cr
       }
\quad\ergo\quad
\bdX_h=T\bdZ
= \sum_{k=1}^n C_k\underbrace{e^{\lambda_kt}\bdv_k}_{\phi_k(t)}.\eqno(\star)
$$
Med andra ord kan man välja fundamentalt system
$\{\phi_1(t),\ldots,\phi_n(t)\}=
 \{e^{\lambda_1t}\bdv_1,\ldots,e^{\lambda_nt}\bdv_n\}$.

%%
\bigskip
\textbf{Fall 2}: $A$ är diagonaliserbar men har komplext
konjugerade egenvärden. Antag, t.ex., att $\lambda_{k+1}=\ob\lambda_k$ för något
$k$ mellan $1$ och $n-1$. Då gäller visserligen framställningen
$(\star)$, men då är resp. egenvektorer också komplext konjugerade,
dvs $\bdv_{k+1}=\ob\bdv_k$. Den delen av $\bdX_h$ som formeras
av dessa två egenvärden kan skrivas som
%%
$$
C_ke^{\lambda_k t}\bdv_k+C_{k+1}e^{\lambda_{k+1}t}\bdv_{k+1}
=C_ke^{\lambda_k t}\bdv_k+C_{k+1}e^{\ob\lambda_{k}}\ob\bdv_{k}
=A\,\mbox{Re\,}[e^{\lambda_k t}\bdv_k]+B\,\mbox{Im}\,[e^{\lambda_k t}\bdv_k]
$$
där $A$ och $B$ också är godtyckliga konstanter. Notera att
om $\bdv_k=\bdp+i\bdq$ (och därmed $\ob\bdv_k=\bdp-i\bdq$),
$\mbox{Re\,}\lambda_k=x$
och $\mbox{Im\,}\lambda_k=y$
så är $e^{\lambda_k t}=e^{xt+iyt}=e^{xt}(\cos yt + i \sin yt)$, och då är
$$
\mbox{Re\,}[e^{\lambda_k t}\bdv_k] = e^{xt}[\cos(yt)\,\bdp-\sin(yt)\,\bdq]
\quad\mbox{ och }\quad
\mbox{Im\,}[e^{\lambda_k t}\bdv_k] = e^{xt}[\sin(yt)\,\bdp+\cos(yt)\,\bdq].
$$


%%
\bigskip
\textbf{Fall 3}: $A$ är ej diagonaliserabar. Då finns det
minst ett egenvärde vars multiplicitet (som rot till det
karakteristiska ekvationen $|A-\lambda I|=0$) är högre än det största
antalet linjärt oberoende egenvektorer det genererar.  Det går
ändå att ta fram lika många lösningar som ska ingå i det fundamentala systemet
som rotens multiplicitet.

Vi betraktar endast det speciella fallet då
$\lambda_k=\lambda_{k+1}=\lambda$ är \emph{dubbelrot} och genererar
endast \emph{en} egenvektor $\bdv$, så att $A\bdv=\lambda\bdv$. Då ges
de två lösningarna $\phi_k(t)$ och $\phi_{k+1}(t)$ i det fundamentala
systemet som genereras av denna dubbelrot av $$
\phi_k(t)=e^{\lambda
  t}\bdv \quad\mbox{ och }\quad \phi_{k+1}(t)=e^{\lambda t}(\bda +
\bdb t), $$
där $\bda$ och $\bdb$ är obekanta vektorer.  Substitutionen
$\bdX=\phi_{k+1}$ i det homogena systemet $\bdX'=A\bdX$ ger: $$
[e^{\lambda t}(\bda + \bdb t)]' =e^{\lambda t}[\lambda(\bda + \bdb
t)+\bdb] =Ae^{\lambda t}(\bda + \bdb t) \iff \lambda(\bda + \bdb
t)+\bdb = A(\bda + \bdb t) \iff\cases{A\bdb=\lambda\bdb\cr(A-\lambda
  I)\bda=\bdb.}
$$
De sista två likheterna ovan följer från kravet att om två vektorpolynom,
ett i VL och ett i HL, ska vara lika för varje $t$, då måste
koefficienterna framför de respektive potenserna av $t$ vara lika.
Notera att eftersom $\bdb$ är egenvektor till $A$, då är $\bdb=\bdv$.
Slutligen är $\bda$ en
godtycklig lösning till systemet $(A-\lambda I)\bda=\bdv$.

\subsection*{Icke-homogena första ordningens linjära system}
Det icke-homogena första ordningens linjära system
$
\bdX'=A\bdX + \bdF
$
har allmän lösning
$
\bdX(t) = \bdX_h(t) + \bdX_p(t)
$.
%%
Här är $\bdX_h$ det homogena systemets ($\bdX'=A\bdX$) allmänna
lösning och $\bdX_p$ är en (vilken som helst) lösning till det
icke-homogena systemet.

Vi betraktar endast det speciella fallet då $\bdF$ är ett
vektorpolynom $\bdP_m(t)$. Man får en partikulär lösning genom att
ansätta $\bdX_p$ till ett polynom av samma grad $m$ som $\bdP_m$ med
obestämda vektorer $\{\bdb_k\}_{k=1}^m$ som koefficienter:
%%
$$
\bdF=\bdP_m(t)=\bda_0+\bda_1t+\cdots\bda_mt^m \ergo
\bdX_p=\bdQ_m(t)=\bdb_0+\bdb_1t+\cdots\bdb_mt^m.  $$
Substitution av
$\bdX_p$ i det icke-homogena systemet $\bdX'=A\bdX + \bdP_m$
resulterar i höger- och vänsterled som är båda polynom av samma grad
$m$.  Att HL = VL för varje $t$ medför att koefficienterna framför de
respektive potenserna av $t$ i HL och VL måste vara lika.

\textit{Exempel}:
Låt $m=1$ så att ekvationsytemet av $\bdX'=A\bdX + \bda_0+\bda_1t$.
Ansättning av $\bdX_p=\bdb_0+\bdb_1t$\, in i systemet och jämförelse av
koefficienterna i polynomen från HL och VL ger:
$$
(\bdb_0+\bdb_1t)'=\bdb_1=A(\bdb_0+\bdb_1t) + \bda_0+\bda_1t
\ergo
\cases{t^1: A\bdb_1=-\bda_1 & $\to\bdb_1$ löses ut\cr
       t^0: A\bdb_0=\bdb_1-\bda_0 & $\to\bdb_0$ löses ut.\cr}
         % & ($\bdb_1$ är redan bestämt från ovan)
$$

\subsection*{Begynnelsevärdesproblem (BVP): \
  $\cases{\bdX'=A\bdX + \bdF\cr \bdX(0)=\bdX_0.}$}

Man tar fram den allmänna lösningen av systemet som ovan:
$\displaystyle\bdX(t)=\bdX_h(t)+\bdX_p(t)=\sum_{k=1}^nC_k\phi_k(t)+\bdX_p(t)$.
De obekanta konstanterna $C_1,...,C_n$ beräknas genom att sätta in
begynnelsevillkoren för $t=0$ och lösa det (algebraiska) linjära systemet:
$$
\sum_{k=1}^nC_k\phi_k(0)
=\underbrace{\pmatrix{
| & \cdots & | \cr
\phi_1(0) & \cdots & \phi_n(0)\cr
| & \cdots & | \cr
}}_{\mbox{$T$, om $A$ är d-bar}}
\pmatrix{C_1\cr\vdots\cr C_n}
=\bdX_0-\bdX_p(0).
$$

%% \section*{Andra ordningens linjära system av ODE
%%   av typen: $\bdX''=A\bdX$.}
%%
%% Vi betraktar endast fallet då $A$ är diagonaliserbar och alla
%% egenvärden är reella. Då gäller:
%% $$
%% \bdX''=A\bdX\!\iff\!
%% \bdX''=TDT^{-1}\bdX\!\iff\!
%% T^{-1}\bdX''=DT^{-1}\bdX\!\iff\!
%% \bdZ''=D\bdZ
%% \mbox{ med } \bdZ=T^{-1}\bdX \!\iff\! \bdX=T\bdZ.
%% $$
%% Det sista systemet består av $n$-st oberoende ekvationer:
%% $$
%% \bdZ''=D\bdZ\iff
%% \cases{
%%      z_1''=\lambda_1 z_1\cr
%%      \hfil\vdots\cr
%%      z_n''=\lambda_n z_n\cr
%%        }
%% \ergo \,z_k(t)
%% =\cases{
%% C_{k1}e^{\sqrt\lambda_kt}+C_{k2}e^{-\sqrt\lambda_kt}, & $\lambda_k>0$;\cr
%% C_{k1}\cos(\sqrt{-\lambda}_kt)+C_{k2}\sin(\sqrt{-\lambda}_kt), &
%%                                            $\lambda_k<0$;\cr
%% C_{k1} + C_{k2}t, & $\lambda_k=0$.
%% }
%% $$
%% Därmed är
%% $$
%% \bdX=T\bdZ = \sum_{k=1}^n z_k(t)\bdv_k.
%% $$
%%
%% \hrule
%% \smallskip
%% \hrule
%%



\section*{Linjära kombinationer, summor och snitt av stokastiska variabler}
Antag att $\{\xi_k\}_{k=1}^n$ är (diskreta eller kontinuerliga) stokastiska variabler med
väntevärden $E(\xi_k)=\mu_k$ och varianser $V(\xi_k)=\sigma_k^2$, $k=1,...,n$, samt
 $\{a_k\}_{k=1}^n$ är konstanter. Då gäller
$$
\eqalign{
E(a_1\xi_1+\cdots+a_n\xi_n)& =a_1\mu_1+\cdots a_n\mu_n\cr
V(a_1\xi_1+\cdots+a_n\xi_n)& =a_1^2\sigma_1^2+\cdots a_n\sigma_n^2
   \quad(\mbox{om de är oberoende}).
}
$$
Om $\{\xi_k\}_{k=1}^n$ är \textbf{normalfördelade} och oberoende,
så är deras linjärkombinationer också normalfördelade:
$$
\xi_k\in N(\mu_k,\sigma_k),\;
k=1,...,n,
\quad\ergo\quad
a_1\xi_1+\cdots+a_n\xi_n\in N(a_1\mu_1+\cdots a_n\mu_n,\sqrt{a_1^2\sigma_1^2+\cdots a_n\sigma_n^2}).
$$
I synnerhet, för \textbf{summor} och \textbf{snitt av likafördelade} och oberoende
variabler gäller:
$$
\xi_k\in N(\mu,\sigma),\;
k=1,...,n,
\quad\ergo\quad
\cases{
\xi_1+\cdots+\xi_n \in N(n\mu,\sigma\sqrt{n})\cr\noalign{\vskip5pt}
\frac{\xi_1+\cdots+\xi_n}{n} = \ob\xi \in N(\mu,\frac{\sigma}{\sqrt{n}})\cr
}
$$



\section*{Väntevärde och varians för vissa fördelningar.
             Approximationer}

\hbox to \hsize{
\def\EspaceAuDessus{&&\\*[-10pt]}\def\FromBelow{\\*[2pt] \hline}
\begin{tabular}{|c|c|c|}
\hline
  \EspaceAuDessus $\xi$        & $\mu=E(\xi)$ & $\sigma^2=V(\xi)$\FromBelow
\hline
  \EspaceAuDessus $Hyp\,(N,n,p)$ & $np$ & $\frac{N-n}{N-1}\,np(1-p) $\FromBelow
  \EspaceAuDessus $Bin\,(n,p)$   & $np$ & $np(1-p)$\FromBelow
  \EspaceAuDessus $Po\,(\lambda)$& $\lambda$ & $\lambda$\FromBelow\hline
  \EspaceAuDessus $Exp\,(\lambda)$& $1/\lambda$ & $1/\lambda^2$\FromBelow
  \EspaceAuDessus $N(\mu,\sigma)$ & $\mu$ & $\sigma^2$\FromBelow
\end{tabular}
 \hfill
   \raise-1.6cm\hbox{\includegraphics[scale=0.35]{approx}}
}

\bigskip


\end{document}
