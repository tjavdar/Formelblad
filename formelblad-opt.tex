% 2019-12-06: Primal<->Dual section total revamp

\documentclass[a4paper]{article}

\usepackage[utf8]{inputenc}
\usepackage{mathtools,graphicx,amssymb}
\usepackage{array}   % nicer tables

% instead of \usepackage{fullpage} <<<
\advance \topmargin by -4.5\headheight
\advance \textheight by 125pt
\oddsidemargin 0pt
\evensidemargin \oddsidemargin
\marginparwidth 0.5in
\textwidth 6.6in
\parindent=0pt
\advance\parskip by 1pt
% >>>

\pagestyle{headings}
\everymath{\textstyle}

%-------------
%- LaTeX def's
%-------------

\everymath{\displaystyle}

% Header matter %<<<
\makeatletter
\renewcommand{\@oddhead}{}
\renewcommand{\@evenhead}{}
\renewcommand{\@oddfoot}
{\ifnum\thepage=1
  \today\hfill file:~\small\texttt{\jobname.pdf}
\else
  LP Formula Sheet
  \hfill
  \Tr{p}{s}.~\thepage{} \Tr{of}{av} \pageref{LastPageNo}
\fi}

  \renewcommand{\@evenfoot}{\small
        \texttt{\jobname.pdf},\hfill}
\makeatother%>>>

\input ../bz/StdLaTeXdef.h

\begin{document}

\section*{LP Formula Sheet}

\subsubsection*{Notation:}%<<<
  \begin{tabular}{ll}
Parameters: &
  $\bdb_{m\times1}$,
  $A_{m\times n}$,
  $\bdc_{n\times1}$,
  % $\bdzero=(0,\ldots,0)^T$,
  % $I_m$ (unit matrix),
  (intermediate values: $\hat \bdb$, $\hat\bdc$)
\cr
Variables: &
  $z$, $w$ (primal, dual objective~f-n),
  $\bdx_{n\times1}$ (primal decision),
  $\bdy_{m\times1}$ (dual ditto),
\cr
  &
  $z^*$, $\bdx^*$,
  $w^*$, $\bdy^*$
  (values at optimum);
  $\bds_{m\times1}$ (slack/surplus),
  $\bda$ (artificial)
\cr
Partitioning:
  & $(A,I_m)\rightsquigarrow(B_{m\times m},N_{m\times n})$,
    $(\bdx_B,\bdx_N)$,
    $(\bdc_B,\bdc_N)$
     $\rightsquigarrow$
  (Basic, Non-basic) columns \cr
%  & $(\bdx_B,\bdx_N)$,
%    $(\bdc_B,\bdc_N)$
%     $\rightsquigarrow$
%       (Basic, Non-basic)
%     variables/coefficients.\cr
\end{tabular}
%>>>

%%\subsubsection*{The normal max LP problem:}%<<<
\[
  \begin{array}[m]{|c|c|}
  \hline
  \textbf{The normal max LP problem:} &
  \textbf{The normal min LP problem:}
  \\ \hline
    \begin{array}[m]{ll}
      \max & z = c_1x_1+c_2x_2+\cdots+c_nx_n \cr
      \mbox{s.t.} &
      \mkern-15mu
      \begin{cases}
        a_{11}x_1+\cdots+a_{1n}x_n &\le b_1 \cr
        \hfil\vdots &  \hfil\vdots          \cr
        a_{m1}x_1+\cdots+a_{mn}x_n &\le b_m \cr
        \hfill x_1,x_2,\ldots,x_n  &\ge 0.
      \end{cases}
    \end{array}
  \;\leftrightsquigarrow\;
  \boxed{
    \begin{array}[m]{ll}
      \max & z=\bdc^T\bdx\cr
      \mbox{s.t.} &
      \mkern-15mu
      \begin{cases}
      A\bdx\le\bdb\cr
      \hfill\bdx\ge\bdzero
      \end{cases}
    \end{array}
  } % boxed
  &
  \boxed{
    \begin{array}[m]{ll}
      \min & z=\bdc^T\bdx\cr
      \mbox{s.t.} &
      \mkern-15mu
      \begin{cases}
      A\bdx\ge\bdb\cr
      \hfill\bdx\ge\bdzero
      \end{cases}
    \end{array}
  } % boxed
  \cr\hline
  \end{array}
% \;\leftrightsquigarrow\;
% \boxed{\{\max\; z=\bdc^T\bdx; A\bdx\le\bdb,\, \bdx\ge\bdzero\}}
\]
%>>>

\subsubsection*{The max LP in standard form, initial and final Simplex Tableau at optimum:}%<<<
\[
  \begin{cases}
    \begin{array}[m]{*{5}{lc}}
    z &-& \bdc^T\bdx &+& \bdzero^T\bds &=&  0   \cr
      & &   A\bdx    &+&   I_m\bds     &=& \bdb \cr
    \end{array}
  \end{cases}
  \rightsquigarrow\;
    \begin{array}[m]{c|ccc|c}
 \perp &     z   & \bdx^T & \bds^T     & \hat\bdb \cr\hline\rule{0pt}{10pt}
    z  &     1   &-\bdc^T & \bdzero^T  &   0    \cr %\hline
 \bds  & \bdzero & A      & I_m        & \bdb   \cr
    \end{array}
  \quad\sim\;\cdots\;\sim\quad
    \begin{array}[m]{c|ccc|c}
 \perp &     z   &     \bdx^T &     \bds^T   & \hat\bdb \cr\hline\rule{0pt}{10pt}
 z  &     1   &-\hat\bdc^T &     \bdy^{*T} & z^*\cr %\hline
 \bdx_B& \bdzero &     B^{-1}A&     B^{-1}   & \bdx_B^*   \cr
    \end{array}
\]
%>>>

\subsubsection*{The final max LP Simplex Tableau (at optimum) partitioned after (B)asic/(N)on-basic columns:}%<<<
\[
  \begin{cases}
    \begin{array}[m]{*{5}{lc}}
      z^* &+& \bdzero^T\bdx_B^* &-& \hat\bdc_N^T\bdx_N &=&  \bdc_B^TB^{-1}\bdb \cr
    %%%          &&&&                       &=& \bdc_B^T\bdx_B^* = \bdb^T\bdy^*\cr
      & &    I_m\bdx_B^*  &+& B^{-1}N\bdx_N     &=& B^{-1}\bdb \cr
    \end{array}
  \end{cases}
  \;\rightsquigarrow\quad\;
    \begin{array}[m]{c|ccc|c}
   \perp &     z   & \bdx_B^T  & \bdx_N^T & \hat\bdb \cr\hline\rule{0pt}{10pt}
      z^*&     1   & \bdzero^T &-\hat\bdc_N^T & \bdc_B^TB^{-1}\bdb \cr %\hline
  \bdx_B & \bdzero &  I_m  & B^{-1}N      &  B^{-1}\bdb   \cr
    \end{array}
\]
Here \fbox{$\bdy^*=B^{-1}\bdc_B$} are the dual vars and
\fbox{$\hat\bdc_N=\bdc_N-\bdc_B^TB^{-1}N=\bdc-y^{*T}N\le0$} the reduced costs at opt.
%>>>

% Weak + Strong Duality<<<
\bigskip
\textbf{Weak duality}: For any feasible $\bdx$ in (P) and $\bdy$ in (D) holds:
\fbox{$
  z = \bdc^T\bdx \le \bdy^TA\bdx \le \bdy^T\bdb = w.
$}
\begin{itemize}
  \item If (P) is unbounded $\ergo$ (D) is infeasible.
        If (D) is unbounded $\ergo$ (P) is infeasible
  \item If both $\bdx^*$ is feasible in (P) and
                $\bdy^*$ is feasible in (D) and
                \[
                  \boxed{z^*=\bdc^T\bdx^* = \bdy^{*T}A\bdx^*=\bdb^T\bdy^*=w^*,}
                  \eqno(\star)
                \]
                then
                $\bdx^*$ is optimal in (P)
           and  $\bdy^*$ is optimal in (D).
\end{itemize}

\textbf{Strong duality}: If either (P) or (D) has a bounded optimal solution,
so has the other and $(\star)$ holds.%>>>

\medskip\textbf{Complementary slackness at optimum}: %<<<
$
\begin{cases}
  \bdy^{*T}(\bdb-A\bdx^*)=0
  \iff y_i^*(b_i-\sum_{j=1}^{n} a_{ij}x_j^*) = 0, \;
   i=1,\ldots,m
\cr
  (\bdy^{*T}A-\bdc)\bdx^*=0 \iff x_j^*(\sum_{i=1}^m y_i^*a_{ij}-c_j)=0,
  \; j=1,\ldots, n.\cr
\end{cases}
$%>>>


\vspace{-1em}
\subsubsection*{Duality}
\(
\boxed{
  \begin{array}[m]{l}
   \textbf{Normal LP's have:}  \\
   \bullet~\mbox{normal constraints:}~\max\!:\,\le,\,\min\!:\,\ge \\
   \bullet~\mbox{normal decis.~vars:}~\bdx\ge\bdzero,\;\bdy\ge\bdzero \\
  \end{array}
}
\;\rightsquigarrow\;
\boxed{
  (P)\!:
  \begin{cases}
    \max~z=\bdc^T\bdx \cr
    A\bdx\le\bdb,\;  \bdx\ge\bdzero
  \end{cases}
}
   \;\overset{1:1}\leftrightsquigarrow\;
   \boxed{
     (D)\!:\begin{cases}
       \min~w=\bdb^T\bdy\cr
       A^T\bdy\ge\bdc,\;\bdy\ge\bdzero.
     \end{cases}
   }
\)%>>>

\bigskip\textbf{(P)rimal-to-(D)ual correspondence rules:}
\begin{itemize}
  \item Involution property: (P) $\overset{1:1}\leftrightsquigarrow$(D),
    i.e., the Dual of the Dual is the Primal
  \item \fbox{Normal (P)} $\overset{1:1}\leftrightsquigarrow$ \fbox{Normal (D)}
  \item \fbox{Constraints of reversed inequality} $\overset{1:1}\leftrightsquigarrow$
    \fbox{Non-positive decision~variables}
  \item \fbox{'='-Constraints} $\overset{1:1}\leftrightsquigarrow$
    \fbox{Non-restricted decision variables}
\end{itemize}

%%% \bigskip \textbf{Involution property}: %<<<
%%% The dual of the dual is the
%%% primal problem. In particular, the dual of a min problem is a max problem and vice versa;
%%% a variable in (P) corresponds to a constraint in (D) and vice versa: %>>>
%%%
%%% \medskip\hfil % \textbf{Correspondence}: %<<<
%%% \begin{tabular}[m]{|c|l|c|c|c|}
%%%   \hline
%%%   Corresp.~$(P)\leftrightsquigarrow (D)$
%%%   & Constraint${}_i$ in (P):
%%%   & $\textstyle\sum a_{ij}x_j\le b_i$
%%%   & $\textstyle\sum a_{ij}x_j\ge b_i$
%%%   & $\textstyle\sum a_{ij}x_j =  b_i$
%%%   \cr\hline
%%%   $(P)_{\min}\iff(D)_{\max}$
%%%     & Sign of $y_i$ in $(D)_{\max}$:
%%%     & $y_i\le 0$
%%%     & $y_i\ge 0$
%%%     & $y_i$ free \cr
%%%   $(P)_{\max}\iff(D)_{\min}$
%%%     & Sign of $y_i$ in $(D)_{\min}$:
%%%     & $y_i\ge 0$
%%%     & $y_i\le 0$
%%%     & $y_i$ free
%%%   \cr\hline
%%% \end{tabular}



\label{LastPageNo}
\end{document}%>>>

% vim: foldmethod=marker spelllang=en
